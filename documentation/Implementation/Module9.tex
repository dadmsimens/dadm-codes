\section{Module 9. Segmentation}

\textbf{\textit{Input and output data}} \\
An input data is NxM brain MRI image after skull stripping and with whole slices (Fig. \ref{fig:figures/m09_567}).\\

\begin{figure}[H]
	\centering
	\begin{subfigure}[b]{0.25\linewidth}
		\includegraphics[width=\linewidth]{figures/Module_09/m09_5}
	\end{subfigure}
		\begin{subfigure}[b]{0.25\linewidth}
		\includegraphics[width=\linewidth]{figures/Module_09/m09_6}
	\end{subfigure}
	\begin{subfigure}[b]{0.25\linewidth}
		\includegraphics[width=\linewidth]{figures/Module_09/m09_7}
	\end{subfigure}
	\caption{Example of input data} 
	\label{fig:figures/m09_567}
\end{figure} \\

An output data is NxM segmentation mask of image witheout nonbarin tissues (\ref{fig:figures/m09_8910}). \\

\begin{figure}[H]
	\centering
	\begin{subfigure}[b]{0.25\linewidth}
		\includegraphics[width=\linewidth]{figures/Module_09/m09_8}
	\end{subfigure}
		\begin{subfigure}[b]{0.25\linewidth}
		\includegraphics[width=\linewidth]{figures/Module_09/m09_9}
	\end{subfigure}
	\begin{subfigure}[b]{0.25\linewidth}
		\includegraphics[width=\linewidth]{figures/Module_09/m09_10}
	\end{subfigure}
	\caption{Example of output data} 
	\label{fig:figures/m09_8910}
\end{figure} \\

\textbf{\textit{Libraries}}
In module, the basic python libraries are used:
\begin{itemize}
	\item math
	\item scipy 
	\item numpy
\end{itemize}

\textbf{\textit{Implementation}}
This module is used to brain MRI segmentation based on skull stripping images from Module 8. It consists of four functions:
\begin{itemize}
	\item imHist (image),
	\item imPart (skullFreeImage, firstPitch, lastPitch, rows, columns, pitches)
	\item gmm (x,mu,v,p),
	\item segmentation (skullFreeImage).
\end{itemize}


\textit{imHist function}
It is a function used for create image histogram. An image histogram is a chart that shows the distribution of intensities in an indexed or grayscale image. This is the first step of segmentation in every single slice of brain MRI image. In the beginning, to calculate the total number of image pixels, an array of image value is changing to vector. Next, by iterating by vector length, histogram is creating step by step (Fig. \ref{fig:figures/m09_11}). 
30/5000
The function argument is the \textbf{NxM single image}

\begin{figure}[H]
	\centering{}\includegraphics[width=0.7\textwidth]{figures/Module_09/m09_11}
	\caption{Image histogram  
	\label{fig:figures/m09_11}}
\end{figure} 

\begin{lstlisting}[language=Python, caption = Create image histogram]
    for i in range(lengthImage):										# create histogram of non-zero image values
        f = int(floor(image[i]))										# round floor
        if f>0:
            if f<maxValue:
                odds=image[i]-f             							# difference between image and round floor image value
                a1=1-odds
                imageHistogram[f-1]=imageHistogram[f-1]+a1
                imageHistogram[f]=imageHistogram[f]+odds
                ...
                imageHistogram=np.convolve(imageHistogram,[1,2,3,2,1])# smoothing the histogram
\end{lstlisting}

At the end, histogram is smoothing by convolution with window [1,2,3,2,1] and noramlize by the sum of histogram values (\ref{fig:figures/m09_12}).

\begin{figure}[H]
	\centering{}\includegraphics[width=0.7\textwidth]{figures/Module_09/m09_12}
	\caption{Image histogram  
	\label{fig:figures/m09_12}}
\end{figure} 
\\

\textit{imPart function}
It is a small function to get only image slices, which dosn't contain nonbrain tissues. In ths case, funtion gets 70 slices, from 70 to 140. Geting greater amonut can make issues. 
The function argument are: \textbf{image after skull stripping, number of first geting slice, number of last geting slice, image rows, image columns, number of slices}
\\
\begin{lstlisting}[language=Python, caption = PGeting part of image to segmentation]
    for pitch in range (pitches):
        if pitch < firstPitch:
            continue
        if pitch >= firstPitch:
            if pitch < lastPitch:
                imageToSeg [:,:,rPit] = skullFreeImage [:,:,pitch]
                rPit = rPit+1
            else:
                break
\end{lstlisting}


\textit{gmm function}
The gmm function calculate mixture model gaussian distribution and and probability for each cluster of image data, based on mixture model initialize parameters
The function argument are: \textbf{x - histogram values vestor, mu - expectation values vector, v - variance vector, p - probability}

\begin{lstlisting}[language=Python, caption = PGeting part of image to segmentation]
for i in range(mu.size):
        differ=x-mu[i]
        amplitude = p[i]/(math.sqrt(2*math.pi*v[i]))
        app = amplitude*(np.exp((-0.5*(differ*differ))/v[i]))
        probab[:,i] = app	
\end{lstlisting}


\textit{segmentation function}
The segmentation function is the most important function in this module. It's utomatic algorithm, matching the Gaussian mixture model to the image histogram, in order to group pixels into proper cluster. To achieve the match, the expectation-maximization algorithm is used. 

The first step is inicialize Gaussian mixture model parameters vectors. They need to have the same numbers of element as numer of clusters. The initial parameters are choosen arbitrarily. It is important, to set different expected values to provide various algorithm starting points. 

\begin{lstlisting}[language=Python, caption = PGeting part of image to segmentation]
# alghoritm inicialization parameters

        clustersNum = 4										# number of clusters
        mu = arange(1,clustersNum+1)*imMax/(clustersNum+1)	# expected value of each clusters
        v = np.ones(clustersNum)*imMax						# variation of each clusters
        p = np.ones(clustersNum)/clustersNum				# probability of each clusters
\end{lstlisting}

After inicialization, the algorithm starts its operation. It consists of two steps. The first one - expectation (step E) - use gmm function to deremine the gaussian distribution and check relative density using inicialized (in the first time) or updated distribution parameters. It calculate also the log-likelihood based on histogram values. 

\begin{lstlisting}[language=Python, caption = PGeting part of image to segmentation]
#Expectation Step

            probab = gmm(x,mu,v,p)							# use gmm function - get probability
            distrDens = probab.sum(axis=1)					# relative density 
            llh = (hx*np.log(distrDens)).sum()				# the log-likelihood base on histogram data
\end{lstlisting}\\

The second step - maximization (step M) -  in which the model parameters are restimated using the distribution from step E



\begin{lstlisting}[language=Python, caption = PGeting part of image to segmentation]
#Maximization Step

            for j in range(clustersNum):
                hxsh = hx.shape
                probabxh = probab.shape
                distsh=distrDens.shape
                resp = hx*probab[:,j]/distrDens				# compute the responsibilities
                p[j]=resp.sum()
                mu[j]=(x*resp/p[j]).sum()					# compute the weighted of expected values
                differ=x-mu[j]
                v[j]=(differ*differ*resp/p[j]).sum()		# compute the weighted varainces	
\end{lstlisting}\\

The algorithm's operation is shown below 

\begin{figure}[H]
	\centering{}\includegraphics[width=0.7\textwidth]{figures/Module_09/m09_13}
	\caption{Image histogram  
	\label{fig:figures/m09_13}}
\end{figure} 

\begin{figure}[H]
	\centering{}\includegraphics[width=0.7\textwidth]{figures/Module_09/m09_14}
	\caption{Image histogram  
	\label{fig:figures/m09_14}}
\end{figure} 

\begin{figure}[H]
	\centering{}\includegraphics[width=0.7\textwidth]{figures/Module_09/m09_15}
	\caption{Image histogram  
	\label{fig:figures/m09_15}}
\end{figure} 


The last part of segmentation functiom is part, which creates image maske after segmenation (). Mask can be divided into 3 separated binary mask, represent white matter, gray matter and cerebrospinal fluid (Fig. \ref{fig:figures/m09_1678}).


\begin{lstlisting}[language=Python, caption = PGeting part of image to segmentation]
# Creat image mask        
        for i in range(rows):
            for j in range(columns):
                for k in range(clustersNum):
                    c[k] = gmm(image[i,j],mu[k],v[k],p[k])
                a = (c==c.max()).nonzero()
                imageMask[i,j]=a[0]
\end{lstlisting}

\begin{figure}[H]
	\centering
	\begin{subfigure}[b]{0.25\linewidth}
		\includegraphics[width=\linewidth]{figures/Module_09/m09_16}
	\end{subfigure}
		\begin{subfigure}[b]{0.25\linewidth}
		\includegraphics[width=\linewidth]{figures/Module_09/m09_17}
	\end{subfigure}
	\begin{subfigure}[b]{0.25\linewidth}
		\includegraphics[width=\linewidth]{figures/Module_09/m09_18}
	\end{subfigure}
	\caption{Example of output data} 
	\label{fig:figures/m09_1678}
\end{figure} \\