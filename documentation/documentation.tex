%% LyX 2.2.1 created this file.  For more info, see http://www.lyx.org/.
%% Do not edit unless you really know what you are doing.
\documentclass{aghdpl}
\usepackage[T1]{fontenc}
\usepackage[utf8]{inputenc}
\setcounter{secnumdepth}{3}
\setcounter{tocdepth}{3}
\usepackage{verbatim}
\usepackage[unicode=true,pdfusetitle,
 bookmarks=true,bookmarksnumbered=true,bookmarksopen=true,bookmarksopenlevel=3,
 breaklinks=false,pdfborder={0 0 1},backref=false,colorlinks=false]
 {hyperref}
\usepackage{breakurl}

\makeatletter

%%%%%%%%%%%%%%%%%%%%%%%%%%%%%% LyX specific LaTeX commands.
%% Because html converters don't know tabularnewline
\providecommand{\tabularnewline}{\\}

%%%%%%%%%%%%%%%%%%%%%%%%%%%%%% User specified LaTeX commands.
\usepackage{polski} 



\usepackage{mathtools}
\usepackage{amsfonts}
\usepackage{amsmath}
\usepackage{amsthm}

\usepackage[
style=numeric,
sorting=none,
language=autobib,
autolang=other,
urldate=iso8601,
backref=false,
isbn=true,
url=false,
maxbibnames=3,
backend=biber,
]{biblatex}
\usepackage{csquotes} 
\DeclareQuoteAlias{croatian}{polish} 


\addbibresource{bibliografia.bib}

\AtBeginDocument{

}

\author{SIMENS Healthkare}
\shortauthor{SIMENS}
\titlePL{DADM project documentation - draft}
\titleEN{}
\shorttitlePL{DADM project documentation}
\shorttitleEN{DADM project documentation}
\thesistype{}
\supervisor{dr inż. Tomasz Pięciak}
\degreeprogramme{Biomedical engineering}
\date{2017}
\department{Katedra Automatyki i Inżynierii Biomedycznej}
\faculty{Wydział Elektrotechniki, Automatyki,\protect\\[-1mm] Informatyki i Inżynierii Biomedycznej}
\acknowledgements{}
\setlength{\cftsecnumwidth}{10mm}



\makeatother

\usepackage[english]{babel}
\begin{document}
\titlepages

\fancypagestyle{plain} {
	\fancyhf{}
	\renewcommand{\headrulewidth}{0pt}
	\renewcommand{\footrulewidth}{0pt}
}

\setcounter{tocdepth}{2}

\tableofcontents{}

\clearpage{}

\chapter{List of changes}

\begin{tabular}{|c|c|c|}
\hline 
Name & Date & Details\tabularnewline
\hline 
\hline 
Sylwia Mól & 19-Nov-2017 & Document created\tabularnewline
\hline 
 &  & \tabularnewline
\hline 
 &  & \tabularnewline
\hline 
 &  & \tabularnewline
\hline 
\end{tabular}

\chapter{Assumptions}

about the app - the aim, what you can do here etc. 

\chapter{Structure}

dependences (tree), modules` descriptions

\chapter{Requirements}

what you need to use the app 

\chapter{Module 1 }

\section{Description}

-ofirst module description

\section{Implementation}

-code

\section{Tests}

-tests

+gdzie powinien znalezc się opis gui czyli jak korzystać z apki? w
każdym module czy osobny rozdzial? jesli tak to jaki?

\chapter{Tests}

-whole app tests

\newpage{}

\listoffigures

\begin{comment}
\bibliographystyle{plain}
\bibliography{bibliografia}
\end{comment}

\end{document}
