%% LyX 2.2.1 created this file.  For more info, see http://www.lyx.org/.
%% Do not edit unless you really know what you are doing.
\documentclass{aghdpl}
\usepackage[T1]{fontenc}
\usepackage[utf8]{inputenc}
\setcounter{secnumdepth}{3}
\setcounter{tocdepth}{3}
\usepackage{verbatim}
\usepackage[unicode=true,pdfusetitle,
 bookmarks=true,bookmarksnumbered=true,bookmarksopen=true,bookmarksopenlevel=3,
 breaklinks=false,pdfborder={0 0 1},backref=false,colorlinks=false]
 {hyperref}
\usepackage{breakurl}

\makeatletter

%%%%%%%%%%%%%%%%%%%%%%%%%%%%%% LyX specific LaTeX commands.
%% Because html converters don't know tabularnewline
\providecommand{\tabularnewline}{\\}

%%%%%%%%%%%%%%%%%%%%%%%%%%%%%% User specified LaTeX commands.
\usepackage{polski} 



\usepackage{mathtools}
\usepackage{amsfonts}
\usepackage{amsmath}
\usepackage{amsthm}

\usepackage[
style=numeric,
sorting=none,
language=autobib,
autolang=other,
urldate=iso8601,
backref=false,
isbn=true,
url=false,
maxbibnames=3,
backend=biber,
]{biblatex}
\usepackage{csquotes} 
\DeclareQuoteAlias{croatian}{polish} 


\addbibresource{bibliografia.bib}

\AtBeginDocument{

}

\author{SIMENS Healthkare}
\shortauthor{SIMENS}
\titlePL{DADM project documentation - draft}
\titleEN{}
\shorttitlePL{DADM project documentation}
\shorttitleEN{DADM project documentation}
\thesistype{}
\supervisor{dr inż. Tomasz Pięciak}
\degreeprogramme{Biomedical engineering}
\date{2017}
\department{Katedra Automatyki i Inżynierii Biomedycznej}
\faculty{Wydział Elektrotechniki, Automatyki,\protect\\[-1mm] Informatyki i Inżynierii Biomedycznej}
\acknowledgements{}
\setlength{\cftsecnumwidth}{10mm}



\makeatother

\usepackage[english]{babel}
\begin{document}
\titlepages

\fancypagestyle{plain} {
	\fancyhf{}
	\renewcommand{\headrulewidth}{0pt}
	\renewcommand{\footrulewidth}{0pt}
}

\setcounter{tocdepth}{2}

\tableofcontents{}

\clearpage{}

\chapter{List of changes}

\begin{tabular}{|c|c|c|}
\hline 
Name & Date & Details\tabularnewline
\hline 
\hline 
Sylwia Mól & 19-Nov-2017 & Document created\tabularnewline
\hline 
 Sylwia Mól &  20-Nov-2017 & Structure changed\tabularnewline
\hline 
 Sylwia Mól &  21-Nov-2017 & Chapter "Authors" added, in-out table added\tabularnewline
\hline 
 &  & \tabularnewline
\hline 
\end{tabular}

\chapter{Assumptions}

about the app - the aim, what you can do here etc. 

\chapter{Structure}

dependences (tree), modules` descriptions
\\ 
\begin{tabular}{|c|c|c|c|}
\hline 
Module & Input  & Output & Before that module\tabularnewline
\hline 
\hline 
1& e.g. raw signal & e.g. JPEG image 320x240 & e.g. modules 3,4\tabularnewline
\hline 
2& & &\tabularnewline
\hline 
3 &  & &\tabularnewline
\hline 
 4&  & &\tabularnewline
\hline
 5&  & &\tabularnewline
\hline 
 6&  & &\tabularnewline
\hline 
 8&  & &\tabularnewline
\hline 
 9&  & &\tabularnewline
\hline  
10&  & &\tabularnewline
\hline 
 11&  & &\tabularnewline
\hline  
\end{tabular}


\chapter{User guide}

\section{Requirements}
what user need to use this app - e.g. windows version etc

\section{Instruction}

instructions for user - GUI screens etc
\chapter{Detailed description}

\section{Module 1}

-detailed description of module, algorithm etc, NO CODES, only theoretical!
\section{Module 2}

-detailed description of module, algorithm etc, NO CODES, only theoretical!
\section{Module 3}

-detailed description of module, algorithm etc, NO CODES, only theoretical!
\section{Module 4}

-detailed description of module, algorithm etc, NO CODES, only theoretical!
\section{Module 5}

-detailed description of module, algorithm etc, NO CODES, only theoretical!
\section{Module 6}

-detailed description of module, algorithm etc, NO CODES, only theoretical!
\section{Module 8}

-detailed description of module, algorithm etc, NO CODES, only theoretical!
\section{Module 9}

-detailed description of module, algorithm etc, NO CODES, only theoretical!
\section{Module 10}

-detailed description of module, algorithm etc, NO CODES, only theoretical!
\section{Module 11}

-detailed description of module, algorithm etc, NO CODES, only theoretical!

\chapter{Implementation}

\section{Tools}

-python version, libraries

\section{Module 1}

-code

\section{Module 2}

-code
\section{Module 3}

-code
\section{Module 4}

-code
\section{Module 5}

-code
\section{Module 6}

-code
\section{Module 8}

-code
\section{Module 9}

-code
\section{Module 10}

-code
\section{Module 11}

-code

\chapter{Tests}

\section{Module 1}
-module 1 tests
\section{Module 2}
\section{Module 3}
\section{Module 4}
\section{Module 5}
\section{Module 6}
\section{Module 8}
\section{Module 9}
\section{Module 10}
\section{Module 11}
\section{Application}
-whole app tests

\chapter{Authors}
Authors of this project are students of Biomedical Engineering, AGH UST, Krakow, Poland. 
\\
\begin{center}
\begin{tabular}{|c|c|}
\hline 
Name & Role \tabularnewline
\hline 
\hline 
Sylwia Mól & Project Manager\tabularnewline
\hline 
 Jacek Fidos &  Software architect\tabularnewline
\hline 
Maciej Gryczan &  GUI engineer\tabularnewline
\hline 
Adrian Stopiak &  Vizualization engineer \tabularnewline
\hline 
Malwina Molendowska &  1st module developer \tabularnewline
\hline 
Klaudia Gugulska &  2nd module developer \tabularnewline
\hline 
Kacper Turek &  3rd module developer \tabularnewline
\hline 
Magdalena Rychlik &  4th module developer \tabularnewline
\hline 
Alicja Martinek &  5th module developer \tabularnewline
\hline 
Mateusz Pabian &  6th module developer \tabularnewline
\hline 
Anna Grzywa &  8th module developer \tabularnewline
\hline 
Magdalena Kucharska &  9th module developer \tabularnewline
\hline 
Eliza Kowalczyk &  9th module developer \tabularnewline
\hline 
Karolina Gajewska &  10th module developer \tabularnewline
\hline 
Michał Kotarba &  11th module developer \tabularnewline
\hline 
\end{tabular}
\end{center}
\newpage{}

\listoffigures

\begin{comment}
\bibliographystyle{plain}
\bibliography{bibliografia}
\end{comment}

\end{document}
