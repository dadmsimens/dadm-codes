%% LyX 2.2.1 created this file.  For more info, see http://www.lyx.org/.
%% Do not edit unless you really know what you are doing.
\documentclass[english]{aghdpl}
\usepackage[T1]{fontenc}
\usepackage[utf8]{inputenc}
\setcounter{secnumdepth}{3}
\setcounter{tocdepth}{3}
\usepackage{babel}
\usepackage{verbatim}
\usepackage{float}
\usepackage{mathtools}
\usepackage{amsmath}
\usepackage{listings}
\usepackage{graphicx}
\usepackage[unicode=true,pdfusetitle,
 bookmarks=true,bookmarksnumbered=true,bookmarksopen=true,bookmarksopenlevel=3,
 breaklinks=false,pdfborder={0 0 1},backref=false,colorlinks=false]
 {hyperref}
\usepackage{breakurl}

\definecolor{mygreen}{rgb}{0,0.6,0}
\definecolor{mygray}{rgb}{0.5,0.5,0.5}
\definecolor{mymauve}{rgb}{0.58,0,0.82}

\lstset{ %
	backgroundcolor=\color{white},   % choose the background color
	basicstyle=\footnotesize,        % size of fonts used for the code
	breaklines=true,                 % automatic line breaking only at whitespace
	captionpos=b,                    % sets the caption-position to bottom
	commentstyle=\color{mygreen},    % comment style
	escapeinside={\%*}{*)},          % if you want to add LaTeX within your code
	keywordstyle=\color{blue},       % keyword style
	stringstyle=\color{mymauve},     % string literal style
}

\makeatletter

%%%%%%%%%%%%%%%%%%%%%%%%%%%%%% LyX specific LaTeX commands.

\newcommand*\LyXZeroWidthSpace{\hspace{0pt}}
%% Because html converters don't know tabularnewline
\providecommand{\tabularnewline}{\\}

%%%%%%%%%%%%%%%%%%%%%%%%%%%%%% User specified LaTeX commands.
\usepackage{babel}
\usepackage{polski}





\usepackage{amsfonts}
\usepackage{amsthm}



\usepackage[
style=numeric,
sorting=none,
language=autobib,
autolang=other,
urldate=iso8601,
backref=false,
isbn=true,
url=false,
maxbibnames=3,
backend=bibtex8,
]{biblatex}
\usepackage{csquotes}
\DeclareQuoteAlias{croatian}{polish}


\addbibresource{bibliografia.bib}

\AtBeginDocument{

}

\author{SIMENS Healthkare}
\shortauthor{SIMENS}
\titlePL{DADM project documentation - draft}
\titleEN{}
\shorttitlePL{DADM project documentation}
\shorttitleEN{DADM project documentation}
\thesistype{}
\supervisor{PhD. Tomasz Pięciak}
\degreeprogramme{Biomedical engineering}
\date{2017}
\department{Department of  Automatics and Biomedical Engineering}
\faculty{AGH University of Science and Technology\protect\\[-2mm]
Faculty of Electrical Engineering, Automatics,\protect\\[-2mm] Computer Science and Biomedical Engineering}
\acknowledgements{}
\setlength{\cftsecnumwidth}{10mm}
\usepackage{makecell}

\makeatother

\begin{document}
\titlepages

\fancypagestyle{plain} { \fancyhf{} \global\long\def\headrulewidth{0pt}
 \global\long\def\footrulewidth{0pt}
 }

\setcounter{tocdepth}{2}

\tableofcontents{}

\clearpage{}

\chapter{List of changes}

\begin{tabular}{|c|c|c|}
\hline
Name  & Date  & Details\tabularnewline
\hline
\hline
Sylwia Mól  & 19-Nov-2017  & Document created\tabularnewline
\hline
Sylwia Mól  & 20-Nov-2017  & Structure changed\tabularnewline
\hline
Sylwia Mól  & 21-Nov-2017  & Chapter \textquotedbl{}Authors\textquotedbl{} added, in-out table
added\tabularnewline
\hline
Malwina Molendowska  & 26-Nov-2017  & Description of 1st module added\tabularnewline
\hline
Eliza Kowalczyk  & 27-Nov-2017  & Description of 9th module added\tabularnewline
\hline
Karolina Gajewska  & 27-Nov-2017  & Description of 11th module added\tabularnewline
\hline
Eliza Kowalczyk  & 28-Nov-2017  & Description of 10th module changed\tabularnewline
\hline
Alicja Martinek  & 28-Nov-2017  & Description of 5th module changed\tabularnewline
\hline
Mateusz Pabian  & 29-Nov-2017  & Description of 6th module changed\tabularnewline
\hline
Jacek Fidos  & 29-Nov-2017  & Tools description added\tabularnewline
\hline
Anna Grzywa  & 29-Nov-2017  & Description of 8th module added\tabularnewline
\hline
Magdalena Rychlik  & 29-Nov-2017  & Description of 4th module \tabularnewline
\hline
Michał Kotarba  & 29-Nov-2017  & Description of 12th module added\tabularnewline
\hline
Magdalena Kucharska  & 29-Nov-2017  & Description of 9th module added\tabularnewline
\hline
Klaudia Gugulska  & 30-Nov-2017  & Description of 2nd module added\tabularnewline
\hline
Sylwia Mól & 2-Dec-2017  & Titles added, numeration changed\tabularnewline
\hline
Jacek Fidos  & 3-Dec-2017  & Files separation\tabularnewline
\hline
Eliza Kowalczyk  & 9-Dec-2017  & Enhancement of description of 10th module\tabularnewline
\hline
Michał Kotarba &9-Dec-2017 & Changed in-out for my module\tabularnewline
\hline
Kacper Turek &9-Dec-2017 & Description of 3rd module changed and in-out for my module added\tabularnewline
\hline
Anna Grzywa &9-Dec-2017 & In-out for 8th module added\tabularnewline
\hline
Sylwia Mól &10-Dec-2017 & "Structure" \& "Assumptions" chapters changed\tabularnewline
\hline
Klaudia Gugulska &10-Dec-2017 &  Update of description of 2nd module\tabularnewline
\hline
Jacek Fidos &4-Jan-2018 &  Short SciPy description\tabularnewline
\hline
Eliza Kowalczyk &14-Jan-2018 &  Upsampling documentation update\tabularnewline
\hline
Sylwia Mól &17-Jan-2018 &  Structure and assumptions changed\tabularnewline
\hline
Eliza Kowalczyk &17-Jan-2018 &  Upsampling documentation update\tabularnewline
\hline
Sylwia Mól &19-Jan-2018 &  Logo added\tabularnewline
\hline
Malwina Molendowska &22-Jan-2018 &  Reconstruction documentation update\tabularnewline
\hline
Alicja Martinek &22-Jan-2018 &  mod5 documentation + lib\tabularnewline
\hline
Karolina Gajewska &22-Jan-2018 &  Brain3D implementation description added \tabularnewline
\hline
Karolina Gajewska &22-Jan-2018 &  Brain3D test description added. \tabularnewline
\hline
\end{tabular}
\newpage
\begin{tabular}{|c|c|c|}
\hline
Name  & Date  & Details\tabularnewline
\hline
Mateusz Pabian &23-Jan-2018 &  DTI description added\tabularnewline
\hline
Anna Grzywa &23-Jan-2018 & M08 description added\tabularnewline
\hline
Magdalena Rychlik &23-Jan-2018 &  Final Module 4 documentation added\tabularnewline
\hline
Malwina Molendowska &23-Jan-2018 &  Final Module 1 documentation added\tabularnewline
\hline
Kacper Turek &23-Jan-2018 &  Final Module 3 documentation added\tabularnewline
\hline
\end{tabular}

\chapter{Assumptions}

The aim of this project is to create the system for MRI data pre-processing and post-processing using Python, SciPy, Cython and Qt5 (detailed description of tools available in chapter {\it Implementation}, section {\it Tools}). The project is divided into 11 modules (more: chapter {\it Structure}). \\

The application is called SieMRI (the abbreviation for Siemens MRI). It allows to recontruct MRI images, correct intensity inhomogenities, estimate non-stationary noise and remove it using one of two methonds, show diffusion tensors, strip the skull, show specific parts of human brain, upsample the image or visualize brain in 3D or using specific plane. MRI image reconstruction starts automatically when user choose the file with data. Other functionalities are non-obligatory - the user decide if specific functionality should be used.

\chapter{Structure}

As mentioned in chapter {\it Assumptions} the project is divided into 11 modules. Each module includes one system`s functionality, as follows:
\begin{itemize}
\item Module 1.: MRI images reconstruction using Cartesian SENSE algorithm.
\item Module 2.: Intensity inhomogenity correction on MRI reconstructed images.
\item Module 3.: Non-stationary noise estimation (a homomorphic approach)
\item Module 4.: Non-stationary noise filtering \#1 with LMMSE filter.
\item Module 5.: Non-stationary noise filtering \#2 using unbiased non-local means filter.
\item Module 6.: Diffusion tensor imaging (WLS, NLS algorithms)
\item Module 8.: Skull stripping
\item Module 9.: MRI segmentation of the human brain
\item Module 10.: Upsampling based on non-local means approach
\item Module 11.: Brain 3D visualization using marching cubes method
\item Module 12.: Oblique imaging
\end{itemize}
Module 7. is excluded due to limited human resources and time.

The input and output signals for each module are listed in the table. \\
\begin{tabular}{|c|c|c|c|}
\hline  Module  & Input  & Output  & Before that module\tabularnewline \hline  \hline  1 & \textbf{k}-space signals  &  \textbf{x}-space fully reconstructed data  & ——–\tabularnewline \hline  2 & reconstruccted image & image with intensity correction & image reconstruction \tabularnewline \hline  3  & reconstructed image  & estimated noise map  & reconstruction \tabularnewline \hline  4 & reconstructed, normalized image & Rician noise-free image & intensity correction \tabularnewline \hline  5 & image, noise map & Rician noise-free image & \makecell{reconstruction, \\ noise estimation} \tabularnewline \hline  6 & \makecell{reconstructed, normalized image;\\ gradient-sequence vectors} & \makecell{estimated diffusion tensor 3D \\ 6-channel image}& modules 1-5, 8* \tabularnewline \hline  8 & reconstructed, normalized, filtered image & image without non-brain tissues & modules 1- 4 or 5 \tabularnewline \hline  9 & image without non-brain tissues & segmentated image & \tabularnewline \hline  10 & 320x240 image  & 640x480 image  & denoised data \tabularnewline \hline  11  &segmentated image & new window of application & segmentation \tabularnewline \hline  12& many images from one examination & one image from non-standard perspective & denoised data\tabularnewline \hline  \end{tabular}\\

Dependencies between modules are shown on the picture. 11th module is divided into two parts, because every part of module requires other pre-procesing activities.
\begin{figure}[H]
\centering{}\includegraphics[scale=0.7]{figures/drzewo2}\caption{Dependencies between modules}
\label{fig:figures/drzewo }
\end{figure}


\chapter{User guide}

\section{Requirements}
\begin{itemize}
\item Windows 7 or newer \textbf{or} Linux (Ubuntu 16.04 tested for now, however the others are also likely to be sufficient)
\item This is highly dependent on the data size, but \textbf{at least} 4 GB of RAM is recommended.
\item Processor - Intel Core 2 duo at minimum, however at least Sandy Bridge (Intel core ''second generation'') is recommended
\end{itemize}


\section{Instruction}

To run the program:

\begin{enumerate}
\item Unzip the project files
\item Navigate to siemri folder and run the siemri.exe file by double clicking
\end{enumerate}



\chapter{Detailed description}

\section{Module 1. MRI reconstruction}

The aim of this module is to formulate mathematical algorithm, which
enables proper data reconstruction for images obtained with parallel
MRI scans. The reconstruction is performed with use of Sensitivity
Encoding (SENSE) algorithm in least squares (LS) solution context
and Tikhonov regularization method.

Generally, parallel MRI acquisitions are targeted to diminish time
needed for data sampling. The usage of multiple coils enabled simultaneous
acquisition of signals. A further step, which is acquiring partial
data from \textbf{k}-space, leads to craved time savings, meanwhile
maintaining full spatial resolution as well as contrast at the same
time. However, the approach of omitting lines in acquisition step
results in data aliasing, i.e. folded images that need further data
processing.

To clearly mark out how data is processed in this module, we list
following reconstruction steps: i) the application of 2D Fourier Transform
transform (2D FFT) to \textbf{k}-space data (acquired raw signals)
from multiple coils. The result is a set of \textbf{x}-space images
with folded pixels, ii) the sensitivity maps estimation of coils profiles
(the information is needed to properly unfold subsampled data) and
iii) the proper unfolding data process with usage of SENSE reconstruction
algorithm and its alterations.

The most crucial step in processing is estimation of sensitivity coil
profiles as a successful image reconstruction with use of pMRI algorithms
highly depends on accurate sensitivity coil assessment. As sensitivity
information varies from scan to scan it is impossible to obtain absolute
maps. To obtain reliable knowledge, reference scans have to be conducted
each time an examination is performed. These low-resolution information
helps to estimate coil profiles with use of the many methods i.e.
dividing each component coil image by a 'sum of squares' image.

It basic formulation, SENSE algorithm is applied to Cartesian MRI
data undersampled uniformly by a factor $r$~(i.e. $r=2$ means that
every other line in \textbf{k}-space is skipped). After Fourier transformation,
each pixel in \textbf{x}-space image received in \textit{l}-th coil
can be seen as weighted sum of $r$ pixels from full FOV, each multiplied
by corresponding localized values of maps. The distance between those
'aliased' points in the full FOV is always equal to the desired FOVy
value divided by subsampling rate. Obviously, depending on subsampling
rate the number of folded pixels changes. Basically, the signal in
one pixel at a certain location $(x,y)$ received from $l$-th component
coil image $D_{l}^{S}$ with chosen subsampling rate $r$ can be written
as: 
\begin{equation}
D_{l}^{S}(x,y)=S_{l}(x,y_{1})D^{R}(x,y_{1})+S_{l}(x,y_{2})D^{R}(x,y_{2})+...+S_{l}(x,y_{r})D^{R}(x,y_{r}),\label{Eq:wzor1}
\end{equation}
where index $l$ counts from 1 to $L$ (number of coils) and index
$i$ counts from 1 to $r$. Eq.(\ref{Eq:wzor1}) can be rewritten
as:

\begin{equation}
D_{l}^{S}(x,y)=\sum_{i=1}^{r}S_{l}(x,y_{i})D^{R}(x,y_{i})\quad\text{for}\quad l=1,...,L.\label{Eq:wzor2}
\end{equation}

Including all $L$ coils the above equation can be rewritten in a
matrix form:

\begin{equation}
\textbf{D}^{S}(\textbf{x})=\textbf{S}(\textbf{x})\textbf{D}^{R}(\textbf{x}),\label{Eq:wzor3}
\end{equation}

The vector $\textbf{D}^{S}(\textbf{x})$ denotes the aliased coil
image values at a specific location \textbf{x} = $(x,y_{i})$ and
has a length of $L$, $\textbf{S}(\textbf{x})$ is a $L$x$R$ matrix
and represents the sensitivities values for each coil at the $r$~superimposed
positions and $\textbf{D}^{R}(\textbf{x})$ lists the $r$ pixels
from full FOV image to be reconstructed. The closed-form solution
of the problem is as follows: 
\begin{equation}
\widehat{\textbf{D}^{R}(\textbf{x})}=(\textbf{S}^{H}(\textbf{x})\textbf{S}(\textbf{x}))^{-1}\textbf{S}^{H}(\textbf{x})\textbf{D}^{S}(\textbf{x}),\label{Eq:wzor4}
\end{equation}
where $\widehat{\textbf{D}^{R}(\textbf{x})}=[\widehat{D^{R}(x,y_{1})},...,\widehat{D^{R}(x,y_{r})}]^{T}$
and $\textbf{S}^{H}(\textbf{x})$ is the conjugate transpose of the
$\textbf{S}(\textbf{x})$ matrix. The final reconstruction image is
defined as: 
\begin{equation}
M(\textbf{x})=\left|\widehat{\textbf{D}^{R}(\textbf{x})}\right|.\label{Eq:wzor5}
\end{equation}

The `unfolding' process can be performed as long as inversion of $\textbf{S}(\textbf{x})$
matrix is possible. Therefore, we cannot set the value of subsampling
rate exceeding the number of coils $L$. To restore full FOV data,
SENSE algorithm has to be recalled for each pixel in aliased \textbf{x}–space
image.

A regularization approach is defined as an~inversion method that
introduces additional information in order to stabilize the solution.
This method is beneficial as it roughly matches the desired solution
and is less sensitive to perturbations of the data. Tikhonov regularization
is a common approach to obtain an inexact solution to a~linear system
of equations. In particular, the Tikhonov regularized estimate reads
as follows:

\begin{equation}
\widehat{\textbf{D}_{reg}^{R}}=\text{arg}\underset{\textbf{D}^{R}}{\text{min}}\left\{ \left\Vert \textbf{D}^{S}-\textbf{S}\textbf{D}^{R}\right\Vert ^{2}+\lambda^{2}\left\Vert \textbf{A}(\textbf{D}^{R}-\textbf{D})\right\Vert ^{2}\right\} ,\label{Eq:wzor6}
\end{equation}

where $\lambda$ is a regularization parameter ($\lambda>0$) and
$\textbf{D}$ is a~prior image known as a regularization image. Selection
of the parameter $\lambda$ and $\textbf{D}$ can be performed using
different procedures. In this module $\textbf{A}$~is assumed to
be an identity matrix. The first term provides fidelity to the data
and the second introduces prior knowledge (e.x. median filtered initial
guess of LS SENSE) about the expected behaviour of $\textbf{D}^{R}$.
The Tikhonov regularization problem is given by:

\begin{equation}
\widehat{\textbf{D}_{reg}^{R}}=\textbf{D}+(\textbf{S}^{H}\textbf{S}+\lambda\textbf{A}^{H}\textbf{A})^{-1}\textbf{S}^{H}(\textbf{D}^{S}-\textbf{S}\textbf{D}).\label{Eq:wzor7}
\end{equation}

A reasonable value for $\lambda$~can be picked using many technique,
i.e. the L-curve criterion or generalized cross-validation.

\textbf{\emph{Module input}}: Synthetic MR images are brain MRI slices
coming from BrainWeb are normalized to {[}0-255{]} (all with intensity
non-uniformity INU=0). Only T1- and T2-weighted data is used. The
dataset is free of noise and the background areas are set to zero.
The slice thickness equals 1 mm. These images are used then to simulate
synthetic noisy accelerated parallel Cartesian SENSE MRI data according
to following steps (the data simulation is performed with use of eight
receiver coils ($L=8$)): i) simulated sensitivity maps (divided into
the ratio 3:1 for real and imaginary parts, respectively) are added
to fully-sampled \textbf{x}-space data, ii) correlated complex Gaussian
noise with different values of standard deviations is added to each
coil image, iii) 2D FFT and data subsampling with chosen reduction
factor $r$ is performed and iv) 2D iFFT is applied to recover data
in \textbf{x}-space. Then, data reconstruction process is conducted.

\textbf{\emph{Module output}}: The output is full resolution reconstructed
data performed with two different algorithms: SENSE (LSE) and Tikhonov
regularization. \\

\section{Module 2. Intensity inhomogeneity correction}
The aim of this module is to remove intensity inhomogeneity from MR image. Usually after MR imaging there are brighter and darker spots in the image of brain. The bias is very undesirable especially for the next modules such as segmentation and 3D imaging. The expected result of this module is that every kind tissue would be presented by approximately one shade of gray. As we never know how the bias map signal will look like, a method of removing it, which is be able to adapt to any shape of the inhomogeneity map is required. To achieve the goal, the following steps are taken. 
Firstly, the reconstructed image with intensity inhomogeneity is read (Fig. \ref{fig: Module2_1}).
\begin{figure}[H]
\centering{}\includegraphics[scale=0.7]{figures/Module_02/m2_3}\caption{The original image with visible intensity inhomogeneity} 
\label{fig: Module2_1}
\end{figure}
The next step is extracting the background of the image with Gaussian Filter of a size equal 2/3 the size of MRI image. In this step all the details corresponding to high frequency components are filtered out. The gaussian filter used in this part takes two variables. The first one is the size of the bandwidth and the second is the standard deviation sigma. For a picture of size 256x256 the size of the filter is set as 170x170 and the sigma is set as 20. Many tests have shown that too large sigma would reduce too much details from the background while too little sigma would leave too much details (Fig. \ref{fig: Module2_5}). The appropriate background extraction is shown in the Fig. \ref{fig: Module2_2}. 
\begin{figure}[H]
\centering{}\includegraphics[scale=0.7]{figures/Module_02/m2_5}\caption{The filtered image with sigma = 5 (left) and sigma = 150 (right)}
\label{fig: Module2_5}
\end{figure}
\begin{figure}[H]
\centering{}\includegraphics[scale=0.7]{figures/Module_02/m2_2}\caption{The image after Gaussian filtering with sigma = 20}
\label{fig: Module2_2}
\end{figure}
Next there are 150 datapoints randomly selected from the background image. Every datapoint has 3 dimensions: x and y coordinates and z as a level of gray. They are used in next steps to fit a surface to the background. Following 3rd degree polynomial was set as the parametric equation of the surface:
\begin{equation}
\begin{aligned}
f(x, y) = a+(b*x)+(c*y)+(d*x^{2})+(e*y^{2})+(f*x*y)+(g*x^{3})+(h*y^{3})+(i*x^{2}*y)+(i*x*y^{2})
\end{aligned}
\end{equation}
The fitting of the surface was performed with an $curve_fit$ function from the scipy module. $*describe in a few words how the function works*$ The equation is used to generate an image of bias field signal. The result is shown in the fig. \ref{fig: Module2_3}.
\begin{figure}[H]
\centering{}\includegraphics[scale=0.7]{figures/Module_02/m2_1}\caption{The surface fitted to the filtered image}
\label{fig: Module2_3}
\end{figure}
The last step is dividing each element of the original image by the corresponding element of the image of bias field performed with numpy function divide(). To avoid dividing by 0, all the pixels in the bias field image that were equal 0 are changed into value 1. The final result of all the following operations is shown in the fig. \ref{fig: Module2_4}.
\begin{figure}[H]
\centering{}\includegraphics[scale=0.7]{figures/Module_02/m2_4}\caption{The result with intenstity bias reduced}
\label{fig: Module2_4}
\end{figure}

\section{Module 3. Non-stationary noise estimation}

Magnetic Resonance Imaging (MRI) is known to be affected by several
sources of quality deterioration, due to limitations in the hardware,
scanning times, movement of patients, or even the motion of molecules
in the scanning subject. Among them, noise is one source of degradation
that affects acquisitions. The presence of noise over the acquired
MR signal is a problem that affects not only the visual quality of
the images, but also may interfere with further processing techniques
such as registration or tensor estimation in Diffusion Tensor MRI.

The aim of this module is to estimate the spatial dependent pattern
of the variance of noise in SENSE reconstructed images. For this to
work some additional information must be known beforehand, such as
the sensitivity maps of each receiver coil. In the background of a
SENSE MR image, where the SNR is zero, the Rician PDF (Probability
density function) simplifies to a (non-stationary) Rayleigh distribution,
whose second order moment is defined as:

\begin{equation}
\begin{aligned}E\left\{ M^{2}\left(x\right)\right\} =2\cdot\sigma_{R}^{2}\cdot\left(x\right)\end{aligned}
\end{equation}

Since $\sigma_{R}^{2}$ is \textit{x}-dependent, $E\left\{ M^{2}\left(x\right)\right\} $
will also show a different value for each \textit{x} position.

Let us assume that each coil in the \textit{x} -space is initially
corrupted with uncorrelated Gaussian noise with the same variance
$\sigma_{n}^{2}$ and there is a correlation between coils $\rho$
so that matrix $\sigma$ becomes:

\begin{equation}
\begin{aligned}\Sigma=\sigma_{n}^{2}\begin{pmatrix}1 & \rho & \cdots & \rho\\
\rho & 1 & \cdots & \rho\\
\vdots & \vdots & \ddots & \vdots\\
\rho & \rho & \cdots & 1
\end{pmatrix}=\sigma_{n}^{2}\left(I+\rho\left[1-I\right]\right)\end{aligned}
\end{equation}
with \textit{I} the $L\times L$ identity matrix and \textit{1} a
$L\times L$ matrix of 1's. For each \textit{x} value, we define the
global map

\begin{equation}
\begin{aligned}G_{w_{i}}=W_{i}^{*}\left(I-\rho\left[1-I\right]\right)W_{i},\quad i=1,\cdots,r\end{aligned}
\end{equation}

Global map $G_{w}(x)$ can be easily inferred from $G_{w_{i}}$ values.
Note that $G_{w}(x)$ is strongly related fo the g-factor, so the
first equation becomes

\begin{equation}
\begin{aligned}E\left\{ M^{2}\left(x\right)\right\} =2\cdot\sigma_{n}^{2}G_{w}\left(x\right)\end{aligned}
\end{equation}

and

\begin{equation}
\begin{aligned}\sigma_{n}^{2}=\frac{E\left\{ M^{2}\left(x\right)\right\} }{2G_{w}\left(x\right)}\end{aligned}
\end{equation}

By using this regularization we can assure a single $\sigma_{n}^{2}$
value for all the points in the image. We can now define a noise estimator
based on the local sample estimation of the second order moment:

\begin{equation}
\begin{aligned}\left\langle M^{2}(x)\right\rangle _{x}=\frac{1}{\left|\eta(x)\right|}\sum_{p\in\eta(x)}M^{2}(p)\end{aligned}
\end{equation}
with $\eta(x)$ a neighborhood centered in x. $\left\langle M^{2}(x)\right\rangle _{x}$
is know to follow a Gamma distribution whose mode is $\sigma_{n}^{2}(\left|\eta(x)\right|-1)/\left|\eta(x)\right|$.
Then

\begin{equation}
\begin{aligned}mode\left\{ \frac{\left\langle M_{L}^{2}\right\rangle _{x}}{G_{w}(x)}\right\} =2\sigma_{n}^{2}\frac{\left|\eta(x)\right|-1}{\left|\eta(x)\right|}\approx2\sigma_{n}^{2}\end{aligned}
\end{equation}
when $\left|\eta(x)\right|\gg1$. The estimator is the defined as

\begin{equation}
\begin{aligned}\widehat{\sigma_{n}^{2}}=\frac{1}{2}mode\left\{ \frac{\left\langle M_{L}^{2}(x)\right\rangle _{x}}{G_{w}(x)}\right\} \end{aligned}
\end{equation}
and consequently the noise in each pixel is estimated as

\begin{equation}
\begin{aligned}\widehat{\sigma_{R}^{2}}=\frac{1}{2}mode\left\{ \frac{\left\langle M_{L}^{2}(x)\right\rangle _{x}}{G_{w}(x)}\right\} G_{w}(x)\end{aligned}
\end{equation}

This estimator is only vaild over the background pixels. However,
no segmentation of these pixels is needed: the use of the mode allows
us to work with the whole image. On the other hand, to carry out the
estimation, the sensitivity map of each coil and the correlation between
coils must be known beforhand. These parameters are needed for the
SENSE encoding, and thus, they can be easily obtained.

\section{Module 4. Non-stationary noise filtering 1}

The unit tests was implemented to check how the programme behaves in different conditions. The few tests

\begin{itemize}

	\item \textbf{Test\_empty\_input\_data} - test checks how the code behave in case receiving empty dataset,
	\item \textbf{Test\_incorrect\_input\_data} - test checks how the code behave, when incorrect data is passed to the function,
	\item \textbf{Test\_size\_input\_data} - test checks if the size of data is changed on the output of code,
	\item \textbf{Test\_none\_noise\_map} - if code received empty data of noise maps, the function should rise an error,
	\item \textbf{Test\_LMMSE\_function\_changes} - test checks if the input data was changed in the LMMSE-filter function.

\end{itemize}

The results of tests were as it was expected and code behaves in correct way.

\section{Module 5. Non-stationary noise filtering 2}

Unbiased Non-Local Means algorithm was implemented based on \cite{5a1}. Some, it is believed, improvements to original idea were introduced and described below. Implementation of the UNLM algorithm was divided into several steps:
\begin{itemize}
	\item rewriting NLM prototype from MATLAB to Python,
	\item refactoring to UNLM,
	\item testing different Gaussian kernels,
	\item optimization of algorithm in order to reduce execution time,
	\item adjusting the algorithm to fit both types of data (structural and diffusion weighted).
\end{itemize}
Because some works and further improvements were done in parallel it is hard to describe implementation in chronological order.

\subsection*{Comparison of different Gaussian kernels}
Once the main body of algorithm was ready, it was decided to run some tests to examine the influence of various Gaussian kernels on the output of the UNLM function. The main requirement for the Gaussian kernel was the fact that the overall sum of all elements within kernel should be equal to 1. Another, for sake of sanity, requirement built on the shape of the kernel surface - the slope should be gentle in order to avoid 'overfitting' to the central pixel.

\begin{figure}[H]
\centering{}
\includegraphics[scale=0.7]{figures/module05/gk1}
\caption{First tested Gaussian kernel.} 
\end{figure}

\begin{figure}[H]
	\centering{}
	\includegraphics[scale=1]{figures/module05/gk1results}
	\caption{Results obtained for first kernel.} 
\end{figure}

First kernel was presented in the article describing the algorithm \cite{5a1}. However, kernel was considered as not optimal due to its two-level shape. To overcome the problem of wrong shape another attempt took place. Second generated kernel had desired shape, yet values didn't meet the requirement of summing to 1.

\begin{figure}[H]
	\centering{}
	\includegraphics[scale=0.7]{figures/module05/gk2}
	\caption{Second tested Gaussian kernel.} 
\end{figure}

\begin{figure}[H]
	\centering{}
	\includegraphics[scale=0.8]{figures/module05/gk2results}
	\caption{Results obtained for second kernel.} 
\end{figure}  

Third kernel, again was disappointing. It had a shape of first kernel, but represented different values, since it was generated in other manner.

\begin{figure}[H]
	\centering{}
	\includegraphics[scale=0.7]{figures/module05/gk3}
	\caption{Third tested Gaussian kernel.} 
\end{figure}

\begin{figure}[H]
	\centering{}
	\includegraphics[scale=0.8]{figures/module05/gk3results}
	\caption{Results obtained for third kernel.} 
\end{figure}

Fourth kernel, had all properties that were desired. It was decided to use this particular one in implementation of the UNLM algorithm.

\begin{figure}[H]
	\centering{}
	\includegraphics[scale=0.7]{figures/module05/gk4}
	\caption{Fourth, and final, tested Gaussian kernel.} 
\end{figure}

\begin{figure}[H]
	\centering{}
	\includegraphics[scale=1]{figures/module05/gk4results}
	\caption{Results obtained for fourth kernel.} 
\end{figure}

Someone with an eye for detail can spot the differences between results achieved whilst using all kernels. Most eye-catching distinction lies in boundaries of the brain and skull.   


Following code snippet presents the implementation of Gaussian window, that was finally chosen after running some tests. This particular window has a special characteristic of being shallow - the central pixel's value doesn't stand out from values in close vicinity. 

\begin{lstlisting}[language=Python, caption = Code used for Gaussian kernel generation.]
def gk4(rsim):

# function that prepares Gaussian window used to penalize
# pixels based on distance within window
kerlen = 2 * rsim + 1
nsig = 2
interval = (2*nsig+1.)/kerlen
x = np.linspace(-nsig-interval/2., nsig+interval/2., kerlen+1)
kern1d = np.diff(st.norm.cdf(x))
kernel_raw = np.sqrt(np.outer(kern1d, kern1d))
kernel = kernel_raw/kernel_raw.sum()
return kernel
\end{lstlisting}

\subsection*{Reducing execution time}
After line-by-line translation of code from MATLAB to Python filtration of single slice lasted for 180 seconds. That was unacceptable, so few flaws in the code had to be identified. It's worth mentioning that single slice filtering in MATLAB took 19 seconds on average.

The first faced incorrectness was the fact of not fully using the numpy library utilities. Changing 'pure python' functions like \textit{sum} to numpy equivalents allowed reduction of time to 80 seconds per slice. That run time was still insufficient. After code profiling with \textit{cProfile} library it was identified, that nested for loops used to move around image were most time consuming. As a result it was decided to get rid off loops and change tedious loop-based architecture to multi-dimensional matrices operations. That operation resulted in achieving 3.5 seconds on average. Involving matrices operations allows to fully exploit numpy implementation (which itself is based on C, so it is faster than Python).

Furthermore, an attempt to use threads in order to speed up calculations on multi-dimensional data took place. For that test, a synthetic 3D data was generated. Processing each slice in a for loop takes 3.5 x \textit{number of slices} seconds. Involving threads should reduce the time, but unfortunately it was doubled. So a tough decision of processing slices in classic loop has been made. Other possibilities of parallel computing were not taken into consideration due to the compatibility issues with GUI.

\subsection*{Unbiased Non-Local Means algorithm}
Having reduced the computation time and having chosen suitable penalty function (Gaussian kernel) algorithm was considered finished. Some of the most important code features are presented in snippets below.

\begin{lstlisting}[language=Python, caption = Overcoming computation of same matrices in every iteration.]
# precompute possible window2
window2_precomp = np.zeros((m, n, rsearch, rsearch))
for i in range(2, m):
for j in range(2, n):
window2_precomp[i, j, :, :] = img_ext[i - rsim:i + rsim + 1, j - rsim:j + rsim + 1]
\end{lstlisting}

During thorough analysis it was discovered that q pixel windows can be calculated only once. Fact of precomputing windows allowed to save resources, which are really important in computationally complex problems.

Moving widows, which are visualized later in this section are changing positions in each iteration. Code snippet below shows how p pixel window and search space window are determined.

\begin{lstlisting}[language=Python, caption = Calculation of p pixel window and search space area.]
# p pixel loop
for i in range(0, m):
for j in range(0, n):
ii = i + rsim
jj = j + rsim
window1 = np.asarray(img_ext[ii - rsim:ii + rsim + 1, jj - rsim:jj + rsim + 1])

# limit the search space
pmin = max(ii - rsearch-1, rsim)
pmax = min(ii + rsearch - 1, m)
qmin = max(jj - rsearch-1, rsim)
qmax = min(jj + rsearch - 1, n)
\end{lstlisting}

Using precomputed set of all possible q pixel widows and using only those that fit the search space was a smart trick presented below.

\begin{lstlisting}[language=Python, caption = Calculation of q pixel windows within search space area.]
window2_slice = window2_precomp[pmin:pmax, qmin:qmax, :]
\end{lstlisting}

To explain the code and present it in more picturesque way, one can look at the image filtering as moving set of
2D windows on image and calculating some metrics, according to their positions and values. There are two types of windows: neighbourhood and search space. Size of neighbourhood window is indirectly defined by rsim parameter, which was assigned by default to 2. That results in neighbourhood window of size 5x5, becasue 5 = 2 * rsim +1. Search space window is dependent ot rsearch parameter, which was assigned to 5 giving 11x11 window. The values of these parameters were set based on analysis conducted in \cite{5a1}. Authors discovered that these sizes give best results in filtration.

Neighbourhood windows are centered around pixels p and q, which are pixel being filtered and filtering pixel correspondingly. Figure below presents the idea of moving windows.

\begin{figure}[H]
	\centering{}
	\includegraphics[scale=0.7]{figures/module05/m5windows}
	\caption{Graphical representation of moving windows.} 
\end{figure}

Given figure has to be explained. Gray area represents 2D picture, the grid on in resembles pixels. Dark green square (pixel) is the p pixel and the light green area around it is the neighbourhood window defined by rsim. Similarly, purple area and dark purple pixel resembles window of pixel q. The size of the image has to be extended by rsim in each direction to make calculations on edges of original image possible. This scenario is presented in the top left corner of picture. Pixel p is within area of image but its window doesn't fit into original image. As a consequence of extending the image, the orange area is temporarily added to the image. The values in the added space are assigned by symmetrical padding near edges. Last thing that has to be deciphered is blue square. It is the search space window. It narrows the possibilities of q pixel positions taken into consideration whilst calculating distances between p and q windows.

It is worth mentioning that there exist special case, when q pixel is in the same position as p pixel. If that situation would have been treated equally to other, what would introduce some bias towards 'self-distance'. According to the equation used to calculate distance (d) between windows, one has to calculate element-wise difference of windows. Having both windows in the same position would result in distance equal to 0, which further can have negative influence on the output. To overcome this problem, maximum weight found within the window is assigned to a weight calculated based on 'zero' distance.


\subsection*{Adjusting the algorithm for both types of data}
Denoising using UNLM method is used by both data types, structural and diffusion weighted. There are some differences in implementation of the algorithm for both types, however the same code is used for them in this module. It is caused by analysis of diagrams presented in \cite{5a2}. They showed that taking gradient information gives no significant information in filtering result. On this occasion, the only difference in handling both types of data lays in main function of the module. More precisely, after making decision which data is being processed loops work on different dimensions. Following code snippet presents the \textit{run\_module} function, where data is being accurately handled.

\begin{lstlisting}[language=Python, caption = run\_module function.]
def run_module(mri_input, other_arguments=None):

if isinstance(mri_input, smns.mri_diff):
[m, n, slices, grad] = mri_input.diffusion_data.shape
data_out = np.zeros([m, n, slices, grad])

for i in range(slices):
for j in range(grad):
data_out[:, :, i, j] = unlm(mri_input.diffusion_data[:, :, i, j], mri_input.noise_map[:, :, i, j])

mri_input.diffusion_data = data_out

elif isinstance(mri_input, smns.mri_struct):
[m, n, slices] = mri_input.structural_data.shape
data_out = np.zeros([m, n, slices])

for i in range(slices):
data_out[:, :, i] = unlm(mri_input.structural_data[:, :, i], mri_input.noise_map[:, :, i])

mri_input.structural_data = data_out

else:
return "Unexpected data format in module number 5!"

return mri_input
\end{lstlisting}

One decision must be finally justified. The whole bigger product is targeted (theoretically) at physicians that usually have no knowledge of image filtering. On this occasion it was decided to reduce number of parameters responsible for algorithm outcome to 0. That solution should reduce possibilities of getting badly-processed data, as optimal parameters are set by default.


\section{Module 6. Diffusion tensor imaging}

\textbf{Preprocessing and Module I/O}

In order to improve diffusion tensor estimation it is imperative to
remove artifacts. In addition to standard MRI pre-processing, one
needs to correct for artifacts arising the use of diffusion-gradient
pulse sequences and longer acquisition time. While hardware manufacturers
try to proactively diminish some of these effects, software processing
is still mandatory. 
\hfill\\

\textbf{Module Input}:
\begin{itemize}
	\item 
	3D structural data array of shape X x Y x Z, where XY - pixel image intensities, Z - chosen slice, which is the T1- or T2-weighted image corresponding the the given DWI acquisition
	
	\item 
	4D diffusion data array of shape X x Y x Z x M, where XY - pixel image intensities, Z - chosen slice, M - applied diffusion gradient direction
	
	\item 
	b\_value, a scalar value corresponding to applied diffusion gradient sequence magnitude
	
	\item 
	2D gradients matrix of shape M x 3, where each row corresponds to a normalized $(x,y,z)$ components of diffusion gradient sequence vectors
		
	\item 
	optionally - 3D binary mask of shape X x Y x Z, corresponding to the brain area detected by Module 8 (Skull Stripping); if not supplied, DTI is computed on each input data voxel

\end{itemize}
\hfill

\textbf{Module Output}:
\begin{itemize}
	\item
	list of size Z, corresponding to each slice; every list element is a dictionary of biomarker images: MD, RA, FA, VR of shape X x Y, and biomarker FA\_rgb of shape X x Y x 3
\end{itemize}

\subsection{Initialization}

In order to abstract DTI implementation from end-user, all classes and methods other than the main function \texttt{run\_module} are private to module source code script. It is important to note that prior to running the module one has to provide the module with input data object, as well as SOLVER and FIX\_METHOD parameters. SOLVER passed as an argument decides whether to use WLS or NLS estimation, whole FIX\_METHOD decides how to "fix" negative eigenvalues. 

As mentioned in the detailed description chapter, 'ABS' takes absolute value of each eigenvalue, while 'CHOLESKY' ensures that the estimated tensor is positive definite. Eigenvalues of positive definite matrices are always non-negative. 'ABS' is a post-estimation fix, meaning that it does not modify the default estimation algorithm (i.e. it is applied after WLS or NLS computation), while 'CHOLESKY' directly modifies the expressions for WLS and NLS cost function gradients and Hessian matrices.

After passing all required arguments to the \texttt{run\_module} function, they are reshaped internally in order to be compatible with module. Concretely, \texttt{DTISolver} class instance, computing the DTI proper, assumes that input data argument is a concatenated 3D array of both structural and diffusion images, which are stored separately in the original data structure. Moreover, b\_value and gradient fields are reshaped to be lists correpsonding to each slice of the new data array (that is: b\_value is repeated in length while both have zeros appended that correspond to structural images). Finally, all of the above is done separately for each slice and DTI module performs it's computation slice-by-slice due to memory constraints.

\subsection{WLS-ABS estimation}

WLS with the ABS fix method is the fastest yet simple method of module pipeline computation based on diffusion tensor estimation. As such these parameters were set as default for DTI.

Diffusion tensor estimate was computed by implementing the equation:
\begin{equation}
\begin{aligned}
\boldsymbol{\gamma}=\left(\boldsymbol{W}^T\boldsymbol{\omega}^T\boldsymbol{\omega}\boldsymbol{W}\right)^{-1}\boldsymbol{W}^T\boldsymbol{\omega}^T\boldsymbol{\omega y}
\end{aligned}
\label{Eq:m6_impl_eq_1}
\end{equation}

using NumPy matrix broadcasting operations, effectively abstracting away array reshaping. Weights vector $\boldsymbol{\omega}$ is calculated using a separate function in order to avoid changing every piece of code refering to WLS weights in case they change. The following implementation assumes the simplest of models presented in the Detail Description chapter, that is weights being equal to the measured signal.

\subsection{NLS-ABS estimation}

In case of NLS estimation, in addition to implementing gradient and Hessian matrix computation methods:

\begin{equation}
\begin{aligned}
\nabla{f_{NLS}}&=-\boldsymbol{W}^T\boldsymbol{\hat{S}}\boldsymbol{r} \\
\nabla^2{f_{NLS}}&=\boldsymbol{W}^T\left(\boldsymbol{\hat{S}^T\hat{S}-\boldsymbol{R\hat{S}}}\right)\boldsymbol{W}
\end{aligned}
\label{Eq:m6_impl_2}
\end{equation}

It is important to devise an iterative scheme because gradient result depends on NLS diffusion tensor estimate. For that reason an algorithm based on \cite{m6_koay2006a} has been implemented. The method itself is called a Modified Newton's Algorithm and can be summarised as in Fig.\ref{fig:m6_pic_1}.

\begin{figure}[H]
	\includegraphics[width=8cm]{figures/Module_06/mfn_simple}
	\centering
	\caption{Modified Newton's method for iterative computation of NLS estimate \vbox{(based on \cite{m6_koay2006a})}}.
	\label{fig:m6_pic_1}
\end{figure}

The following parameters (collectively known in code as MFN parameters) were set:
\begin{itemize}
	\item 
	MFN\_MAX\_ITER = 3 - iteration limit
	
	\item
	MFN\_ERROR\_EPSILON = 1e-5 - first convergence criterion (error change is small)
	
	\item
	MFN\_GRADIENT\_EPSILON = 1e-5 - second convergence criterion (vanishing gradient)
	
	\item
	MFN\_LAMBDA\_MATRIX\_FUN = 'identity'- regularization matrix added to Hessian matrix
	
	\item 
	MFN\_LAMBDA\_PARAM\_INIT = 1e-4 - initial regularization matrix multiplier
\end{itemize}

Delta estimate is calculated from the following formula:
\begin{equation}
\boldsymbol{\delta}=-\left(\nabla^2{f_{NLS}+\lambda I}\right)^{-1}\nabla{f_{NLS}}
\label{Eq:m6_impl_3}
\end{equation}


\section{Module 8. Skull stripping}

\section{Module 9. Segmentation}

Unity tests are designed to checked if smaller unit of module is running correctly. In this module, the unity tests are using to test each function in module for different conditions. implemented unity test provide information, what happens in module after changing the source code. 

\textbf{\textit{functionality of imHist function}} 
Unity test is checking functionality for different data type, data with only 0 values, empty dataset. Module returns correctly data only for double data type. Unexpected result should be shown to the user.

\textbf{\textit{functionality of gmm function}}
Unity test is checking functionality for different data type, data with only 0 values, empty dataset and missing arguments. Checking the function behavior for different data sizes is also very important. Valid, because of using matrix operation. 
Module returns correctly data only for double, nonzero array data, with the same size of expected values vector (mu), variation vector (v), probability vector (p). Unexpected result should be shown to the user.

\textbf{\textit{functionality of imPart function}}
\begin{itemize}
	\item Checking value of lastPitch, if the value is greater than pitches or smaller than firstPitch, function returns an error.
	\item Checking value of firstPitch, if the value is negative, greater than lastPitch or pitches, function returns an error.
	\item Checking the correct size of image data for the selected part. If the separated part of the data does not have a size equal to [rows x column x difference between the lastPitch and the firstPitch], the function should not return the image and returns an error
\end{itemize}


\textit{\textbf{functionality of segmentation function}}
\begin{itemize}
	\item Loading an image automatically starts the algorithm and calls all functions.
	\item An unexpected action of any of the functions immediately interrupts the program and the function returns an error.
	\item Checking the size of initialized parameters vectors, if are not the same, function returns an error. 
	\item Testing exit condition of while loop in function. If the loop operation repeats over 100 times, the function will be immediately interrupted and return an error. 
	\item Checking the correct size of segmentation image mask. It should be the same as the size of the data after calling the function partIm, in other way function returns an error.
	\item Checking mask contents after segmentation. The mask should contain four data clusters for: background, cerebrospinal fluid,  gray matter, white matter, with values, respectively, 1, 2, 3 and 4, which is corresponding to image histogram.
\end{itemize}



\section{Module 10. Upsampling}

-code

\section{Module 11. Brain 3D}

\indent To prepare tree dimension visualization of the cerebral cortex algorithm of marching cubes
is used.\\
 \indent The input data is multiple 2D slices of MR image. The marching cubes algorithm create a polygonal representation of constant density surfaces from a 3D array of data. To select the cerebral cortex is used output data from segmentation made in module 9. The structure of cortex is represented by value 3 in segmentation mask. 
\indent The space of the image is divided into a regular grid of cubes. In each iteration one cube is considered. At each vertex of cube is determined how the surface intersects this cube. The density value are compared with the limit value - surface constant. If the data value is bigger than suface constant, one is assinged to a cube’s vertex. There are 256 combinations of cube orientation relative to the surface, but we can distinguish 15 basic patterns, that repeat as symmetrical reflections, produces all possibilities (Fig. \ref{fig:figures/Marching cubes}). If all values are less than the constant value, then the cube does not form any polygon. Otherwise, the edges of the polygon are defined (by linear interpolation) at the edges that intersect the surface. Using central differences, a unit normal at each cube vertex is calculated and then normal to each trangle vertex is interpolated. The output of the algorithm is the triangle vertices and vertex normals.

\begin{figure}[H]
\centering{}\includegraphics[scale=0.7]{figures/MarchingCubes}\caption{Triangulation for the 15 patterns. \label{fig:figures/Marching cubes}}
\end{figure}

\indent To visualization the model, obtained by marching cubes, the
VTK library is used, which enables building the three-dimension model.
The VTK is object-oriented library. The classes of VTK is dedicated to processing and visualization data.
\\
 \indent The second part of this module includes visualization of the brain’s cross-section on arbitrarily defined plane. To enable selecting of intersection plane there was used object of VTK class, which allows set plane in elected direction by using computer mouse. When the plane is moved by user, in the real time the  three-dimensional model is clipped in the place of selected plane. To improve quality of visualization the cross-section image, there is also possibility to see the image imposed on the three-dimensional model. \\



\section{Module 12. Oblique imaging}

\indent The literature to this module is as useful as nipples on
men. Everything is about inventing how to realize point 1. of the
list. \\
 \indent Oblique Imaging is a technique to create non-perspective
projections from 3D or multiple 2D images.\\

\indent In order to create oblique image it is essential to: 
\begin{itemize}
\item choose two angles under which the plane will be inclined, 
\item create a matrix of points that this plane consists of, 
\item from existing points pick those, which will be used in the image, 
\item interpolate points that are not existing. 
\end{itemize}
\indent Type of interpolation can vary, but in this project interpolation
based on mean will be used. To interpolate one pixel mean of all pixels
around him with given proximity is taken.

\begin{figure}[H]
\centering{}\includegraphics[scale=0.7]{figures/m11_spherexyz}\caption{Visualization of pixels taken to interpolate}
\label{fig:figures/m11_spherexyz } 
\end{figure}



\chapter{Implementation}

\input{"Implementation/Tools"}

\section{Module 1. MRI reconstruction}

The aim of this module is to formulate mathematical algorithm, which
enables proper data reconstruction for images obtained with parallel
MRI scans. The reconstruction is performed with use of Sensitivity
Encoding (SENSE) algorithm in least squares (LS) solution context
and Tikhonov regularization method.

Generally, parallel MRI acquisitions are targeted to diminish time
needed for data sampling. The usage of multiple coils enabled simultaneous
acquisition of signals. A further step, which is acquiring partial
data from \textbf{k}-space, leads to craved time savings, meanwhile
maintaining full spatial resolution as well as contrast at the same
time. However, the approach of omitting lines in acquisition step
results in data aliasing, i.e. folded images that need further data
processing.

To clearly mark out how data is processed in this module, we list
following reconstruction steps: i) the application of 2D Fourier Transform
transform (2D FFT) to \textbf{k}-space data (acquired raw signals)
from multiple coils. The result is a set of \textbf{x}-space images
with folded pixels, ii) the sensitivity maps estimation of coils profiles
(the information is needed to properly unfold subsampled data) and
iii) the proper unfolding data process with usage of SENSE reconstruction
algorithm and its alterations.

The most crucial step in processing is estimation of sensitivity coil
profiles as a successful image reconstruction with use of pMRI algorithms
highly depends on accurate sensitivity coil assessment. As sensitivity
information varies from scan to scan it is impossible to obtain absolute
maps. To obtain reliable knowledge, reference scans have to be conducted
each time an examination is performed. These low-resolution information
helps to estimate coil profiles with use of the many methods i.e.
dividing each component coil image by a 'sum of squares' image.

It basic formulation, SENSE algorithm is applied to Cartesian MRI
data undersampled uniformly by a factor $r$~(i.e. $r=2$ means that
every other line in \textbf{k}-space is skipped). After Fourier transformation,
each pixel in \textbf{x}-space image received in \textit{l}-th coil
can be seen as weighted sum of $r$ pixels from full FOV, each multiplied
by corresponding localized values of maps. The distance between those
'aliased' points in the full FOV is always equal to the desired FOVy
value divided by subsampling rate. Obviously, depending on subsampling
rate the number of folded pixels changes. Basically, the signal in
one pixel at a certain location $(x,y)$ received from $l$-th component
coil image $D_{l}^{S}$ with chosen subsampling rate $r$ can be written
as: 
\begin{equation}
D_{l}^{S}(x,y)=S_{l}(x,y_{1})D^{R}(x,y_{1})+S_{l}(x,y_{2})D^{R}(x,y_{2})+...+S_{l}(x,y_{r})D^{R}(x,y_{r}),\label{Eq:wzor1}
\end{equation}
where index $l$ counts from 1 to $L$ (number of coils) and index
$i$ counts from 1 to $r$. Eq.(\ref{Eq:wzor1}) can be rewritten
as:

\begin{equation}
D_{l}^{S}(x,y)=\sum_{i=1}^{r}S_{l}(x,y_{i})D^{R}(x,y_{i})\quad\text{for}\quad l=1,...,L.\label{Eq:wzor2}
\end{equation}

Including all $L$ coils the above equation can be rewritten in a
matrix form:

\begin{equation}
\textbf{D}^{S}(\textbf{x})=\textbf{S}(\textbf{x})\textbf{D}^{R}(\textbf{x}),\label{Eq:wzor3}
\end{equation}

The vector $\textbf{D}^{S}(\textbf{x})$ denotes the aliased coil
image values at a specific location \textbf{x} = $(x,y_{i})$ and
has a length of $L$, $\textbf{S}(\textbf{x})$ is a $L$x$R$ matrix
and represents the sensitivities values for each coil at the $r$~superimposed
positions and $\textbf{D}^{R}(\textbf{x})$ lists the $r$ pixels
from full FOV image to be reconstructed. The closed-form solution
of the problem is as follows: 
\begin{equation}
\widehat{\textbf{D}^{R}(\textbf{x})}=(\textbf{S}^{H}(\textbf{x})\textbf{S}(\textbf{x}))^{-1}\textbf{S}^{H}(\textbf{x})\textbf{D}^{S}(\textbf{x}),\label{Eq:wzor4}
\end{equation}
where $\widehat{\textbf{D}^{R}(\textbf{x})}=[\widehat{D^{R}(x,y_{1})},...,\widehat{D^{R}(x,y_{r})}]^{T}$
and $\textbf{S}^{H}(\textbf{x})$ is the conjugate transpose of the
$\textbf{S}(\textbf{x})$ matrix. The final reconstruction image is
defined as: 
\begin{equation}
M(\textbf{x})=\left|\widehat{\textbf{D}^{R}(\textbf{x})}\right|.\label{Eq:wzor5}
\end{equation}

The `unfolding' process can be performed as long as inversion of $\textbf{S}(\textbf{x})$
matrix is possible. Therefore, we cannot set the value of subsampling
rate exceeding the number of coils $L$. To restore full FOV data,
SENSE algorithm has to be recalled for each pixel in aliased \textbf{x}–space
image.

A regularization approach is defined as an~inversion method that
introduces additional information in order to stabilize the solution.
This method is beneficial as it roughly matches the desired solution
and is less sensitive to perturbations of the data. Tikhonov regularization
is a common approach to obtain an inexact solution to a~linear system
of equations. In particular, the Tikhonov regularized estimate reads
as follows:

\begin{equation}
\widehat{\textbf{D}_{reg}^{R}}=\text{arg}\underset{\textbf{D}^{R}}{\text{min}}\left\{ \left\Vert \textbf{D}^{S}-\textbf{S}\textbf{D}^{R}\right\Vert ^{2}+\lambda^{2}\left\Vert \textbf{A}(\textbf{D}^{R}-\textbf{D})\right\Vert ^{2}\right\} ,\label{Eq:wzor6}
\end{equation}

where $\lambda$ is a regularization parameter ($\lambda>0$) and
$\textbf{D}$ is a~prior image known as a regularization image. Selection
of the parameter $\lambda$ and $\textbf{D}$ can be performed using
different procedures. In this module $\textbf{A}$~is assumed to
be an identity matrix. The first term provides fidelity to the data
and the second introduces prior knowledge (e.x. median filtered initial
guess of LS SENSE) about the expected behaviour of $\textbf{D}^{R}$.
The Tikhonov regularization problem is given by:

\begin{equation}
\widehat{\textbf{D}_{reg}^{R}}=\textbf{D}+(\textbf{S}^{H}\textbf{S}+\lambda\textbf{A}^{H}\textbf{A})^{-1}\textbf{S}^{H}(\textbf{D}^{S}-\textbf{S}\textbf{D}).\label{Eq:wzor7}
\end{equation}

A reasonable value for $\lambda$~can be picked using many technique,
i.e. the L-curve criterion or generalized cross-validation.

\textbf{\emph{Module input}}: Synthetic MR images are brain MRI slices
coming from BrainWeb are normalized to {[}0-255{]} (all with intensity
non-uniformity INU=0). Only T1- and T2-weighted data is used. The
dataset is free of noise and the background areas are set to zero.
The slice thickness equals 1 mm. These images are used then to simulate
synthetic noisy accelerated parallel Cartesian SENSE MRI data according
to following steps (the data simulation is performed with use of eight
receiver coils ($L=8$)): i) simulated sensitivity maps (divided into
the ratio 3:1 for real and imaginary parts, respectively) are added
to fully-sampled \textbf{x}-space data, ii) correlated complex Gaussian
noise with different values of standard deviations is added to each
coil image, iii) 2D FFT and data subsampling with chosen reduction
factor $r$ is performed and iv) 2D iFFT is applied to recover data
in \textbf{x}-space. Then, data reconstruction process is conducted.

\textbf{\emph{Module output}}: The output is full resolution reconstructed
data performed with two different algorithms: SENSE (LSE) and Tikhonov
regularization. \\

\section{Module 2. Intensity inhomogeneity correction}
The aim of this module is to remove intensity inhomogeneity from MR image. Usually after MR imaging there are brighter and darker spots in the image of brain. The bias is very undesirable especially for the next modules such as segmentation and 3D imaging. The expected result of this module is that every kind tissue would be presented by approximately one shade of gray. As we never know how the bias map signal will look like, a method of removing it, which is be able to adapt to any shape of the inhomogeneity map is required. To achieve the goal, the following steps are taken. 
Firstly, the reconstructed image with intensity inhomogeneity is read (Fig. \ref{fig: Module2_1}).
\begin{figure}[H]
\centering{}\includegraphics[scale=0.7]{figures/Module_02/m2_3}\caption{The original image with visible intensity inhomogeneity} 
\label{fig: Module2_1}
\end{figure}
The next step is extracting the background of the image with Gaussian Filter of a size equal 2/3 the size of MRI image. In this step all the details corresponding to high frequency components are filtered out. The gaussian filter used in this part takes two variables. The first one is the size of the bandwidth and the second is the standard deviation sigma. For a picture of size 256x256 the size of the filter is set as 170x170 and the sigma is set as 20. Many tests have shown that too large sigma would reduce too much details from the background while too little sigma would leave too much details (Fig. \ref{fig: Module2_5}). The appropriate background extraction is shown in the Fig. \ref{fig: Module2_2}. 
\begin{figure}[H]
\centering{}\includegraphics[scale=0.7]{figures/Module_02/m2_5}\caption{The filtered image with sigma = 5 (left) and sigma = 150 (right)}
\label{fig: Module2_5}
\end{figure}
\begin{figure}[H]
\centering{}\includegraphics[scale=0.7]{figures/Module_02/m2_2}\caption{The image after Gaussian filtering with sigma = 20}
\label{fig: Module2_2}
\end{figure}
Next there are 150 datapoints randomly selected from the background image. Every datapoint has 3 dimensions: x and y coordinates and z as a level of gray. They are used in next steps to fit a surface to the background. Following 3rd degree polynomial was set as the parametric equation of the surface:
\begin{equation}
\begin{aligned}
f(x, y) = a+(b*x)+(c*y)+(d*x^{2})+(e*y^{2})+(f*x*y)+(g*x^{3})+(h*y^{3})+(i*x^{2}*y)+(i*x*y^{2})
\end{aligned}
\end{equation}
The fitting of the surface was performed with an $curve_fit$ function from the scipy module. $*describe in a few words how the function works*$ The equation is used to generate an image of bias field signal. The result is shown in the fig. \ref{fig: Module2_3}.
\begin{figure}[H]
\centering{}\includegraphics[scale=0.7]{figures/Module_02/m2_1}\caption{The surface fitted to the filtered image}
\label{fig: Module2_3}
\end{figure}
The last step is dividing each element of the original image by the corresponding element of the image of bias field performed with numpy function divide(). To avoid dividing by 0, all the pixels in the bias field image that were equal 0 are changed into value 1. The final result of all the following operations is shown in the fig. \ref{fig: Module2_4}.
\begin{figure}[H]
\centering{}\includegraphics[scale=0.7]{figures/Module_02/m2_4}\caption{The result with intenstity bias reduced}
\label{fig: Module2_4}
\end{figure}

\section{Module 3. Non-stationary noise estimation}

Magnetic Resonance Imaging (MRI) is known to be affected by several
sources of quality deterioration, due to limitations in the hardware,
scanning times, movement of patients, or even the motion of molecules
in the scanning subject. Among them, noise is one source of degradation
that affects acquisitions. The presence of noise over the acquired
MR signal is a problem that affects not only the visual quality of
the images, but also may interfere with further processing techniques
such as registration or tensor estimation in Diffusion Tensor MRI.

The aim of this module is to estimate the spatial dependent pattern
of the variance of noise in SENSE reconstructed images. For this to
work some additional information must be known beforehand, such as
the sensitivity maps of each receiver coil. In the background of a
SENSE MR image, where the SNR is zero, the Rician PDF (Probability
density function) simplifies to a (non-stationary) Rayleigh distribution,
whose second order moment is defined as:

\begin{equation}
\begin{aligned}E\left\{ M^{2}\left(x\right)\right\} =2\cdot\sigma_{R}^{2}\cdot\left(x\right)\end{aligned}
\end{equation}

Since $\sigma_{R}^{2}$ is \textit{x}-dependent, $E\left\{ M^{2}\left(x\right)\right\} $
will also show a different value for each \textit{x} position.

Let us assume that each coil in the \textit{x} -space is initially
corrupted with uncorrelated Gaussian noise with the same variance
$\sigma_{n}^{2}$ and there is a correlation between coils $\rho$
so that matrix $\sigma$ becomes:

\begin{equation}
\begin{aligned}\Sigma=\sigma_{n}^{2}\begin{pmatrix}1 & \rho & \cdots & \rho\\
\rho & 1 & \cdots & \rho\\
\vdots & \vdots & \ddots & \vdots\\
\rho & \rho & \cdots & 1
\end{pmatrix}=\sigma_{n}^{2}\left(I+\rho\left[1-I\right]\right)\end{aligned}
\end{equation}
with \textit{I} the $L\times L$ identity matrix and \textit{1} a
$L\times L$ matrix of 1's. For each \textit{x} value, we define the
global map

\begin{equation}
\begin{aligned}G_{w_{i}}=W_{i}^{*}\left(I-\rho\left[1-I\right]\right)W_{i},\quad i=1,\cdots,r\end{aligned}
\end{equation}

Global map $G_{w}(x)$ can be easily inferred from $G_{w_{i}}$ values.
Note that $G_{w}(x)$ is strongly related fo the g-factor, so the
first equation becomes

\begin{equation}
\begin{aligned}E\left\{ M^{2}\left(x\right)\right\} =2\cdot\sigma_{n}^{2}G_{w}\left(x\right)\end{aligned}
\end{equation}

and

\begin{equation}
\begin{aligned}\sigma_{n}^{2}=\frac{E\left\{ M^{2}\left(x\right)\right\} }{2G_{w}\left(x\right)}\end{aligned}
\end{equation}

By using this regularization we can assure a single $\sigma_{n}^{2}$
value for all the points in the image. We can now define a noise estimator
based on the local sample estimation of the second order moment:

\begin{equation}
\begin{aligned}\left\langle M^{2}(x)\right\rangle _{x}=\frac{1}{\left|\eta(x)\right|}\sum_{p\in\eta(x)}M^{2}(p)\end{aligned}
\end{equation}
with $\eta(x)$ a neighborhood centered in x. $\left\langle M^{2}(x)\right\rangle _{x}$
is know to follow a Gamma distribution whose mode is $\sigma_{n}^{2}(\left|\eta(x)\right|-1)/\left|\eta(x)\right|$.
Then

\begin{equation}
\begin{aligned}mode\left\{ \frac{\left\langle M_{L}^{2}\right\rangle _{x}}{G_{w}(x)}\right\} =2\sigma_{n}^{2}\frac{\left|\eta(x)\right|-1}{\left|\eta(x)\right|}\approx2\sigma_{n}^{2}\end{aligned}
\end{equation}
when $\left|\eta(x)\right|\gg1$. The estimator is the defined as

\begin{equation}
\begin{aligned}\widehat{\sigma_{n}^{2}}=\frac{1}{2}mode\left\{ \frac{\left\langle M_{L}^{2}(x)\right\rangle _{x}}{G_{w}(x)}\right\} \end{aligned}
\end{equation}
and consequently the noise in each pixel is estimated as

\begin{equation}
\begin{aligned}\widehat{\sigma_{R}^{2}}=\frac{1}{2}mode\left\{ \frac{\left\langle M_{L}^{2}(x)\right\rangle _{x}}{G_{w}(x)}\right\} G_{w}(x)\end{aligned}
\end{equation}

This estimator is only vaild over the background pixels. However,
no segmentation of these pixels is needed: the use of the mode allows
us to work with the whole image. On the other hand, to carry out the
estimation, the sensitivity map of each coil and the correlation between
coils must be known beforhand. These parameters are needed for the
SENSE encoding, and thus, they can be easily obtained.

\section{Module 4. Non-stationary noise filtering 1}

The unit tests was implemented to check how the programme behaves in different conditions. The few tests

\begin{itemize}

	\item \textbf{Test\_empty\_input\_data} - test checks how the code behave in case receiving empty dataset,
	\item \textbf{Test\_incorrect\_input\_data} - test checks how the code behave, when incorrect data is passed to the function,
	\item \textbf{Test\_size\_input\_data} - test checks if the size of data is changed on the output of code,
	\item \textbf{Test\_none\_noise\_map} - if code received empty data of noise maps, the function should rise an error,
	\item \textbf{Test\_LMMSE\_function\_changes} - test checks if the input data was changed in the LMMSE-filter function.

\end{itemize}

The results of tests were as it was expected and code behaves in correct way.

\section{Module 5. Non-stationary noise filtering 2}

Unbiased Non-Local Means algorithm was implemented based on \cite{5a1}. Some, it is believed, improvements to original idea were introduced and described below. Implementation of the UNLM algorithm was divided into several steps:
\begin{itemize}
	\item rewriting NLM prototype from MATLAB to Python,
	\item refactoring to UNLM,
	\item testing different Gaussian kernels,
	\item optimization of algorithm in order to reduce execution time,
	\item adjusting the algorithm to fit both types of data (structural and diffusion weighted).
\end{itemize}
Because some works and further improvements were done in parallel it is hard to describe implementation in chronological order.

\subsection*{Comparison of different Gaussian kernels}
Once the main body of algorithm was ready, it was decided to run some tests to examine the influence of various Gaussian kernels on the output of the UNLM function. The main requirement for the Gaussian kernel was the fact that the overall sum of all elements within kernel should be equal to 1. Another, for sake of sanity, requirement built on the shape of the kernel surface - the slope should be gentle in order to avoid 'overfitting' to the central pixel.

\begin{figure}[H]
\centering{}
\includegraphics[scale=0.7]{figures/module05/gk1}
\caption{First tested Gaussian kernel.} 
\end{figure}

\begin{figure}[H]
	\centering{}
	\includegraphics[scale=1]{figures/module05/gk1results}
	\caption{Results obtained for first kernel.} 
\end{figure}

First kernel was presented in the article describing the algorithm \cite{5a1}. However, kernel was considered as not optimal due to its two-level shape. To overcome the problem of wrong shape another attempt took place. Second generated kernel had desired shape, yet values didn't meet the requirement of summing to 1.

\begin{figure}[H]
	\centering{}
	\includegraphics[scale=0.7]{figures/module05/gk2}
	\caption{Second tested Gaussian kernel.} 
\end{figure}

\begin{figure}[H]
	\centering{}
	\includegraphics[scale=0.8]{figures/module05/gk2results}
	\caption{Results obtained for second kernel.} 
\end{figure}  

Third kernel, again was disappointing. It had a shape of first kernel, but represented different values, since it was generated in other manner.

\begin{figure}[H]
	\centering{}
	\includegraphics[scale=0.7]{figures/module05/gk3}
	\caption{Third tested Gaussian kernel.} 
\end{figure}

\begin{figure}[H]
	\centering{}
	\includegraphics[scale=0.8]{figures/module05/gk3results}
	\caption{Results obtained for third kernel.} 
\end{figure}

Fourth kernel, had all properties that were desired. It was decided to use this particular one in implementation of the UNLM algorithm.

\begin{figure}[H]
	\centering{}
	\includegraphics[scale=0.7]{figures/module05/gk4}
	\caption{Fourth, and final, tested Gaussian kernel.} 
\end{figure}

\begin{figure}[H]
	\centering{}
	\includegraphics[scale=1]{figures/module05/gk4results}
	\caption{Results obtained for fourth kernel.} 
\end{figure}

Someone with an eye for detail can spot the differences between results achieved whilst using all kernels. Most eye-catching distinction lies in boundaries of the brain and skull.   


Following code snippet presents the implementation of Gaussian window, that was finally chosen after running some tests. This particular window has a special characteristic of being shallow - the central pixel's value doesn't stand out from values in close vicinity. 

\begin{lstlisting}[language=Python, caption = Code used for Gaussian kernel generation.]
def gk4(rsim):

# function that prepares Gaussian window used to penalize
# pixels based on distance within window
kerlen = 2 * rsim + 1
nsig = 2
interval = (2*nsig+1.)/kerlen
x = np.linspace(-nsig-interval/2., nsig+interval/2., kerlen+1)
kern1d = np.diff(st.norm.cdf(x))
kernel_raw = np.sqrt(np.outer(kern1d, kern1d))
kernel = kernel_raw/kernel_raw.sum()
return kernel
\end{lstlisting}

\subsection*{Reducing execution time}
After line-by-line translation of code from MATLAB to Python filtration of single slice lasted for 180 seconds. That was unacceptable, so few flaws in the code had to be identified. It's worth mentioning that single slice filtering in MATLAB took 19 seconds on average.

The first faced incorrectness was the fact of not fully using the numpy library utilities. Changing 'pure python' functions like \textit{sum} to numpy equivalents allowed reduction of time to 80 seconds per slice. That run time was still insufficient. After code profiling with \textit{cProfile} library it was identified, that nested for loops used to move around image were most time consuming. As a result it was decided to get rid off loops and change tedious loop-based architecture to multi-dimensional matrices operations. That operation resulted in achieving 3.5 seconds on average. Involving matrices operations allows to fully exploit numpy implementation (which itself is based on C, so it is faster than Python).

Furthermore, an attempt to use threads in order to speed up calculations on multi-dimensional data took place. For that test, a synthetic 3D data was generated. Processing each slice in a for loop takes 3.5 x \textit{number of slices} seconds. Involving threads should reduce the time, but unfortunately it was doubled. So a tough decision of processing slices in classic loop has been made. Other possibilities of parallel computing were not taken into consideration due to the compatibility issues with GUI.

\subsection*{Unbiased Non-Local Means algorithm}
Having reduced the computation time and having chosen suitable penalty function (Gaussian kernel) algorithm was considered finished. Some of the most important code features are presented in snippets below.

\begin{lstlisting}[language=Python, caption = Overcoming computation of same matrices in every iteration.]
# precompute possible window2
window2_precomp = np.zeros((m, n, rsearch, rsearch))
for i in range(2, m):
for j in range(2, n):
window2_precomp[i, j, :, :] = img_ext[i - rsim:i + rsim + 1, j - rsim:j + rsim + 1]
\end{lstlisting}

During thorough analysis it was discovered that q pixel windows can be calculated only once. Fact of precomputing windows allowed to save resources, which are really important in computationally complex problems.

Moving widows, which are visualized later in this section are changing positions in each iteration. Code snippet below shows how p pixel window and search space window are determined.

\begin{lstlisting}[language=Python, caption = Calculation of p pixel window and search space area.]
# p pixel loop
for i in range(0, m):
for j in range(0, n):
ii = i + rsim
jj = j + rsim
window1 = np.asarray(img_ext[ii - rsim:ii + rsim + 1, jj - rsim:jj + rsim + 1])

# limit the search space
pmin = max(ii - rsearch-1, rsim)
pmax = min(ii + rsearch - 1, m)
qmin = max(jj - rsearch-1, rsim)
qmax = min(jj + rsearch - 1, n)
\end{lstlisting}

Using precomputed set of all possible q pixel widows and using only those that fit the search space was a smart trick presented below.

\begin{lstlisting}[language=Python, caption = Calculation of q pixel windows within search space area.]
window2_slice = window2_precomp[pmin:pmax, qmin:qmax, :]
\end{lstlisting}

To explain the code and present it in more picturesque way, one can look at the image filtering as moving set of
2D windows on image and calculating some metrics, according to their positions and values. There are two types of windows: neighbourhood and search space. Size of neighbourhood window is indirectly defined by rsim parameter, which was assigned by default to 2. That results in neighbourhood window of size 5x5, becasue 5 = 2 * rsim +1. Search space window is dependent ot rsearch parameter, which was assigned to 5 giving 11x11 window. The values of these parameters were set based on analysis conducted in \cite{5a1}. Authors discovered that these sizes give best results in filtration.

Neighbourhood windows are centered around pixels p and q, which are pixel being filtered and filtering pixel correspondingly. Figure below presents the idea of moving windows.

\begin{figure}[H]
	\centering{}
	\includegraphics[scale=0.7]{figures/module05/m5windows}
	\caption{Graphical representation of moving windows.} 
\end{figure}

Given figure has to be explained. Gray area represents 2D picture, the grid on in resembles pixels. Dark green square (pixel) is the p pixel and the light green area around it is the neighbourhood window defined by rsim. Similarly, purple area and dark purple pixel resembles window of pixel q. The size of the image has to be extended by rsim in each direction to make calculations on edges of original image possible. This scenario is presented in the top left corner of picture. Pixel p is within area of image but its window doesn't fit into original image. As a consequence of extending the image, the orange area is temporarily added to the image. The values in the added space are assigned by symmetrical padding near edges. Last thing that has to be deciphered is blue square. It is the search space window. It narrows the possibilities of q pixel positions taken into consideration whilst calculating distances between p and q windows.

It is worth mentioning that there exist special case, when q pixel is in the same position as p pixel. If that situation would have been treated equally to other, what would introduce some bias towards 'self-distance'. According to the equation used to calculate distance (d) between windows, one has to calculate element-wise difference of windows. Having both windows in the same position would result in distance equal to 0, which further can have negative influence on the output. To overcome this problem, maximum weight found within the window is assigned to a weight calculated based on 'zero' distance.


\subsection*{Adjusting the algorithm for both types of data}
Denoising using UNLM method is used by both data types, structural and diffusion weighted. There are some differences in implementation of the algorithm for both types, however the same code is used for them in this module. It is caused by analysis of diagrams presented in \cite{5a2}. They showed that taking gradient information gives no significant information in filtering result. On this occasion, the only difference in handling both types of data lays in main function of the module. More precisely, after making decision which data is being processed loops work on different dimensions. Following code snippet presents the \textit{run\_module} function, where data is being accurately handled.

\begin{lstlisting}[language=Python, caption = run\_module function.]
def run_module(mri_input, other_arguments=None):

if isinstance(mri_input, smns.mri_diff):
[m, n, slices, grad] = mri_input.diffusion_data.shape
data_out = np.zeros([m, n, slices, grad])

for i in range(slices):
for j in range(grad):
data_out[:, :, i, j] = unlm(mri_input.diffusion_data[:, :, i, j], mri_input.noise_map[:, :, i, j])

mri_input.diffusion_data = data_out

elif isinstance(mri_input, smns.mri_struct):
[m, n, slices] = mri_input.structural_data.shape
data_out = np.zeros([m, n, slices])

for i in range(slices):
data_out[:, :, i] = unlm(mri_input.structural_data[:, :, i], mri_input.noise_map[:, :, i])

mri_input.structural_data = data_out

else:
return "Unexpected data format in module number 5!"

return mri_input
\end{lstlisting}

One decision must be finally justified. The whole bigger product is targeted (theoretically) at physicians that usually have no knowledge of image filtering. On this occasion it was decided to reduce number of parameters responsible for algorithm outcome to 0. That solution should reduce possibilities of getting badly-processed data, as optimal parameters are set by default.


\section{Module 6. Diffusion tensor imaging}

\textbf{Preprocessing and Module I/O}

In order to improve diffusion tensor estimation it is imperative to
remove artifacts. In addition to standard MRI pre-processing, one
needs to correct for artifacts arising the use of diffusion-gradient
pulse sequences and longer acquisition time. While hardware manufacturers
try to proactively diminish some of these effects, software processing
is still mandatory. 
\hfill\\

\textbf{Module Input}:
\begin{itemize}
	\item 
	3D structural data array of shape X x Y x Z, where XY - pixel image intensities, Z - chosen slice, which is the T1- or T2-weighted image corresponding the the given DWI acquisition
	
	\item 
	4D diffusion data array of shape X x Y x Z x M, where XY - pixel image intensities, Z - chosen slice, M - applied diffusion gradient direction
	
	\item 
	b\_value, a scalar value corresponding to applied diffusion gradient sequence magnitude
	
	\item 
	2D gradients matrix of shape M x 3, where each row corresponds to a normalized $(x,y,z)$ components of diffusion gradient sequence vectors
		
	\item 
	optionally - 3D binary mask of shape X x Y x Z, corresponding to the brain area detected by Module 8 (Skull Stripping); if not supplied, DTI is computed on each input data voxel

\end{itemize}
\hfill

\textbf{Module Output}:
\begin{itemize}
	\item
	list of size Z, corresponding to each slice; every list element is a dictionary of biomarker images: MD, RA, FA, VR of shape X x Y, and biomarker FA\_rgb of shape X x Y x 3
\end{itemize}

\subsection{Initialization}

In order to abstract DTI implementation from end-user, all classes and methods other than the main function \texttt{run\_module} are private to module source code script. It is important to note that prior to running the module one has to provide the module with input data object, as well as SOLVER and FIX\_METHOD parameters. SOLVER passed as an argument decides whether to use WLS or NLS estimation, whole FIX\_METHOD decides how to "fix" negative eigenvalues. 

As mentioned in the detailed description chapter, 'ABS' takes absolute value of each eigenvalue, while 'CHOLESKY' ensures that the estimated tensor is positive definite. Eigenvalues of positive definite matrices are always non-negative. 'ABS' is a post-estimation fix, meaning that it does not modify the default estimation algorithm (i.e. it is applied after WLS or NLS computation), while 'CHOLESKY' directly modifies the expressions for WLS and NLS cost function gradients and Hessian matrices.

After passing all required arguments to the \texttt{run\_module} function, they are reshaped internally in order to be compatible with module. Concretely, \texttt{DTISolver} class instance, computing the DTI proper, assumes that input data argument is a concatenated 3D array of both structural and diffusion images, which are stored separately in the original data structure. Moreover, b\_value and gradient fields are reshaped to be lists correpsonding to each slice of the new data array (that is: b\_value is repeated in length while both have zeros appended that correspond to structural images). Finally, all of the above is done separately for each slice and DTI module performs it's computation slice-by-slice due to memory constraints.

\subsection{WLS-ABS estimation}

WLS with the ABS fix method is the fastest yet simple method of module pipeline computation based on diffusion tensor estimation. As such these parameters were set as default for DTI.

Diffusion tensor estimate was computed by implementing the equation:
\begin{equation}
\begin{aligned}
\boldsymbol{\gamma}=\left(\boldsymbol{W}^T\boldsymbol{\omega}^T\boldsymbol{\omega}\boldsymbol{W}\right)^{-1}\boldsymbol{W}^T\boldsymbol{\omega}^T\boldsymbol{\omega y}
\end{aligned}
\label{Eq:m6_impl_eq_1}
\end{equation}

using NumPy matrix broadcasting operations, effectively abstracting away array reshaping. Weights vector $\boldsymbol{\omega}$ is calculated using a separate function in order to avoid changing every piece of code refering to WLS weights in case they change. The following implementation assumes the simplest of models presented in the Detail Description chapter, that is weights being equal to the measured signal.

\subsection{NLS-ABS estimation}

In case of NLS estimation, in addition to implementing gradient and Hessian matrix computation methods:

\begin{equation}
\begin{aligned}
\nabla{f_{NLS}}&=-\boldsymbol{W}^T\boldsymbol{\hat{S}}\boldsymbol{r} \\
\nabla^2{f_{NLS}}&=\boldsymbol{W}^T\left(\boldsymbol{\hat{S}^T\hat{S}-\boldsymbol{R\hat{S}}}\right)\boldsymbol{W}
\end{aligned}
\label{Eq:m6_impl_2}
\end{equation}

It is important to devise an iterative scheme because gradient result depends on NLS diffusion tensor estimate. For that reason an algorithm based on \cite{m6_koay2006a} has been implemented. The method itself is called a Modified Newton's Algorithm and can be summarised as in Fig.\ref{fig:m6_pic_1}.

\begin{figure}[H]
	\includegraphics[width=8cm]{figures/Module_06/mfn_simple}
	\centering
	\caption{Modified Newton's method for iterative computation of NLS estimate \vbox{(based on \cite{m6_koay2006a})}}.
	\label{fig:m6_pic_1}
\end{figure}

The following parameters (collectively known in code as MFN parameters) were set:
\begin{itemize}
	\item 
	MFN\_MAX\_ITER = 3 - iteration limit
	
	\item
	MFN\_ERROR\_EPSILON = 1e-5 - first convergence criterion (error change is small)
	
	\item
	MFN\_GRADIENT\_EPSILON = 1e-5 - second convergence criterion (vanishing gradient)
	
	\item
	MFN\_LAMBDA\_MATRIX\_FUN = 'identity'- regularization matrix added to Hessian matrix
	
	\item 
	MFN\_LAMBDA\_PARAM\_INIT = 1e-4 - initial regularization matrix multiplier
\end{itemize}

Delta estimate is calculated from the following formula:
\begin{equation}
\boldsymbol{\delta}=-\left(\nabla^2{f_{NLS}+\lambda I}\right)^{-1}\nabla{f_{NLS}}
\label{Eq:m6_impl_3}
\end{equation}


\section{Module 8. Skull stripping}

\section{Module 9. Segmentation}

Unity tests are designed to checked if smaller unit of module is running correctly. In this module, the unity tests are using to test each function in module for different conditions. implemented unity test provide information, what happens in module after changing the source code. 

\textbf{\textit{functionality of imHist function}} 
Unity test is checking functionality for different data type, data with only 0 values, empty dataset. Module returns correctly data only for double data type. Unexpected result should be shown to the user.

\textbf{\textit{functionality of gmm function}}
Unity test is checking functionality for different data type, data with only 0 values, empty dataset and missing arguments. Checking the function behavior for different data sizes is also very important. Valid, because of using matrix operation. 
Module returns correctly data only for double, nonzero array data, with the same size of expected values vector (mu), variation vector (v), probability vector (p). Unexpected result should be shown to the user.

\textbf{\textit{functionality of imPart function}}
\begin{itemize}
	\item Checking value of lastPitch, if the value is greater than pitches or smaller than firstPitch, function returns an error.
	\item Checking value of firstPitch, if the value is negative, greater than lastPitch or pitches, function returns an error.
	\item Checking the correct size of image data for the selected part. If the separated part of the data does not have a size equal to [rows x column x difference between the lastPitch and the firstPitch], the function should not return the image and returns an error
\end{itemize}


\textit{\textbf{functionality of segmentation function}}
\begin{itemize}
	\item Loading an image automatically starts the algorithm and calls all functions.
	\item An unexpected action of any of the functions immediately interrupts the program and the function returns an error.
	\item Checking the size of initialized parameters vectors, if are not the same, function returns an error. 
	\item Testing exit condition of while loop in function. If the loop operation repeats over 100 times, the function will be immediately interrupted and return an error. 
	\item Checking the correct size of segmentation image mask. It should be the same as the size of the data after calling the function partIm, in other way function returns an error.
	\item Checking mask contents after segmentation. The mask should contain four data clusters for: background, cerebrospinal fluid,  gray matter, white matter, with values, respectively, 1, 2, 3 and 4, which is corresponding to image histogram.
\end{itemize}



\section{Module 10. Upsampling}

-code

\section{Module 11. Brain 3D}

\indent To prepare tree dimension visualization of the cerebral cortex algorithm of marching cubes
is used.\\
 \indent The input data is multiple 2D slices of MR image. The marching cubes algorithm create a polygonal representation of constant density surfaces from a 3D array of data. To select the cerebral cortex is used output data from segmentation made in module 9. The structure of cortex is represented by value 3 in segmentation mask. 
\indent The space of the image is divided into a regular grid of cubes. In each iteration one cube is considered. At each vertex of cube is determined how the surface intersects this cube. The density value are compared with the limit value - surface constant. If the data value is bigger than suface constant, one is assinged to a cube’s vertex. There are 256 combinations of cube orientation relative to the surface, but we can distinguish 15 basic patterns, that repeat as symmetrical reflections, produces all possibilities (Fig. \ref{fig:figures/Marching cubes}). If all values are less than the constant value, then the cube does not form any polygon. Otherwise, the edges of the polygon are defined (by linear interpolation) at the edges that intersect the surface. Using central differences, a unit normal at each cube vertex is calculated and then normal to each trangle vertex is interpolated. The output of the algorithm is the triangle vertices and vertex normals.

\begin{figure}[H]
\centering{}\includegraphics[scale=0.7]{figures/MarchingCubes}\caption{Triangulation for the 15 patterns. \label{fig:figures/Marching cubes}}
\end{figure}

\indent To visualization the model, obtained by marching cubes, the
VTK library is used, which enables building the three-dimension model.
The VTK is object-oriented library. The classes of VTK is dedicated to processing and visualization data.
\\
 \indent The second part of this module includes visualization of the brain’s cross-section on arbitrarily defined plane. To enable selecting of intersection plane there was used object of VTK class, which allows set plane in elected direction by using computer mouse. When the plane is moved by user, in the real time the  three-dimensional model is clipped in the place of selected plane. To improve quality of visualization the cross-section image, there is also possibility to see the image imposed on the three-dimensional model. \\



\section{Module 12. Oblique imaging}

\indent The literature to this module is as useful as nipples on
men. Everything is about inventing how to realize point 1. of the
list. \\
 \indent Oblique Imaging is a technique to create non-perspective
projections from 3D or multiple 2D images.\\

\indent In order to create oblique image it is essential to: 
\begin{itemize}
\item choose two angles under which the plane will be inclined, 
\item create a matrix of points that this plane consists of, 
\item from existing points pick those, which will be used in the image, 
\item interpolate points that are not existing. 
\end{itemize}
\indent Type of interpolation can vary, but in this project interpolation
based on mean will be used. To interpolate one pixel mean of all pixels
around him with given proximity is taken.

\begin{figure}[H]
\centering{}\includegraphics[scale=0.7]{figures/m11_spherexyz}\caption{Visualization of pixels taken to interpolate}
\label{fig:figures/m11_spherexyz } 
\end{figure}



\chapter{Tests}

\section{Module 1. MRI reconstruction}

The aim of this module is to formulate mathematical algorithm, which
enables proper data reconstruction for images obtained with parallel
MRI scans. The reconstruction is performed with use of Sensitivity
Encoding (SENSE) algorithm in least squares (LS) solution context
and Tikhonov regularization method.

Generally, parallel MRI acquisitions are targeted to diminish time
needed for data sampling. The usage of multiple coils enabled simultaneous
acquisition of signals. A further step, which is acquiring partial
data from \textbf{k}-space, leads to craved time savings, meanwhile
maintaining full spatial resolution as well as contrast at the same
time. However, the approach of omitting lines in acquisition step
results in data aliasing, i.e. folded images that need further data
processing.

To clearly mark out how data is processed in this module, we list
following reconstruction steps: i) the application of 2D Fourier Transform
transform (2D FFT) to \textbf{k}-space data (acquired raw signals)
from multiple coils. The result is a set of \textbf{x}-space images
with folded pixels, ii) the sensitivity maps estimation of coils profiles
(the information is needed to properly unfold subsampled data) and
iii) the proper unfolding data process with usage of SENSE reconstruction
algorithm and its alterations.

The most crucial step in processing is estimation of sensitivity coil
profiles as a successful image reconstruction with use of pMRI algorithms
highly depends on accurate sensitivity coil assessment. As sensitivity
information varies from scan to scan it is impossible to obtain absolute
maps. To obtain reliable knowledge, reference scans have to be conducted
each time an examination is performed. These low-resolution information
helps to estimate coil profiles with use of the many methods i.e.
dividing each component coil image by a 'sum of squares' image.

It basic formulation, SENSE algorithm is applied to Cartesian MRI
data undersampled uniformly by a factor $r$~(i.e. $r=2$ means that
every other line in \textbf{k}-space is skipped). After Fourier transformation,
each pixel in \textbf{x}-space image received in \textit{l}-th coil
can be seen as weighted sum of $r$ pixels from full FOV, each multiplied
by corresponding localized values of maps. The distance between those
'aliased' points in the full FOV is always equal to the desired FOVy
value divided by subsampling rate. Obviously, depending on subsampling
rate the number of folded pixels changes. Basically, the signal in
one pixel at a certain location $(x,y)$ received from $l$-th component
coil image $D_{l}^{S}$ with chosen subsampling rate $r$ can be written
as: 
\begin{equation}
D_{l}^{S}(x,y)=S_{l}(x,y_{1})D^{R}(x,y_{1})+S_{l}(x,y_{2})D^{R}(x,y_{2})+...+S_{l}(x,y_{r})D^{R}(x,y_{r}),\label{Eq:wzor1}
\end{equation}
where index $l$ counts from 1 to $L$ (number of coils) and index
$i$ counts from 1 to $r$. Eq.(\ref{Eq:wzor1}) can be rewritten
as:

\begin{equation}
D_{l}^{S}(x,y)=\sum_{i=1}^{r}S_{l}(x,y_{i})D^{R}(x,y_{i})\quad\text{for}\quad l=1,...,L.\label{Eq:wzor2}
\end{equation}

Including all $L$ coils the above equation can be rewritten in a
matrix form:

\begin{equation}
\textbf{D}^{S}(\textbf{x})=\textbf{S}(\textbf{x})\textbf{D}^{R}(\textbf{x}),\label{Eq:wzor3}
\end{equation}

The vector $\textbf{D}^{S}(\textbf{x})$ denotes the aliased coil
image values at a specific location \textbf{x} = $(x,y_{i})$ and
has a length of $L$, $\textbf{S}(\textbf{x})$ is a $L$x$R$ matrix
and represents the sensitivities values for each coil at the $r$~superimposed
positions and $\textbf{D}^{R}(\textbf{x})$ lists the $r$ pixels
from full FOV image to be reconstructed. The closed-form solution
of the problem is as follows: 
\begin{equation}
\widehat{\textbf{D}^{R}(\textbf{x})}=(\textbf{S}^{H}(\textbf{x})\textbf{S}(\textbf{x}))^{-1}\textbf{S}^{H}(\textbf{x})\textbf{D}^{S}(\textbf{x}),\label{Eq:wzor4}
\end{equation}
where $\widehat{\textbf{D}^{R}(\textbf{x})}=[\widehat{D^{R}(x,y_{1})},...,\widehat{D^{R}(x,y_{r})}]^{T}$
and $\textbf{S}^{H}(\textbf{x})$ is the conjugate transpose of the
$\textbf{S}(\textbf{x})$ matrix. The final reconstruction image is
defined as: 
\begin{equation}
M(\textbf{x})=\left|\widehat{\textbf{D}^{R}(\textbf{x})}\right|.\label{Eq:wzor5}
\end{equation}

The `unfolding' process can be performed as long as inversion of $\textbf{S}(\textbf{x})$
matrix is possible. Therefore, we cannot set the value of subsampling
rate exceeding the number of coils $L$. To restore full FOV data,
SENSE algorithm has to be recalled for each pixel in aliased \textbf{x}–space
image.

A regularization approach is defined as an~inversion method that
introduces additional information in order to stabilize the solution.
This method is beneficial as it roughly matches the desired solution
and is less sensitive to perturbations of the data. Tikhonov regularization
is a common approach to obtain an inexact solution to a~linear system
of equations. In particular, the Tikhonov regularized estimate reads
as follows:

\begin{equation}
\widehat{\textbf{D}_{reg}^{R}}=\text{arg}\underset{\textbf{D}^{R}}{\text{min}}\left\{ \left\Vert \textbf{D}^{S}-\textbf{S}\textbf{D}^{R}\right\Vert ^{2}+\lambda^{2}\left\Vert \textbf{A}(\textbf{D}^{R}-\textbf{D})\right\Vert ^{2}\right\} ,\label{Eq:wzor6}
\end{equation}

where $\lambda$ is a regularization parameter ($\lambda>0$) and
$\textbf{D}$ is a~prior image known as a regularization image. Selection
of the parameter $\lambda$ and $\textbf{D}$ can be performed using
different procedures. In this module $\textbf{A}$~is assumed to
be an identity matrix. The first term provides fidelity to the data
and the second introduces prior knowledge (e.x. median filtered initial
guess of LS SENSE) about the expected behaviour of $\textbf{D}^{R}$.
The Tikhonov regularization problem is given by:

\begin{equation}
\widehat{\textbf{D}_{reg}^{R}}=\textbf{D}+(\textbf{S}^{H}\textbf{S}+\lambda\textbf{A}^{H}\textbf{A})^{-1}\textbf{S}^{H}(\textbf{D}^{S}-\textbf{S}\textbf{D}).\label{Eq:wzor7}
\end{equation}

A reasonable value for $\lambda$~can be picked using many technique,
i.e. the L-curve criterion or generalized cross-validation.

\textbf{\emph{Module input}}: Synthetic MR images are brain MRI slices
coming from BrainWeb are normalized to {[}0-255{]} (all with intensity
non-uniformity INU=0). Only T1- and T2-weighted data is used. The
dataset is free of noise and the background areas are set to zero.
The slice thickness equals 1 mm. These images are used then to simulate
synthetic noisy accelerated parallel Cartesian SENSE MRI data according
to following steps (the data simulation is performed with use of eight
receiver coils ($L=8$)): i) simulated sensitivity maps (divided into
the ratio 3:1 for real and imaginary parts, respectively) are added
to fully-sampled \textbf{x}-space data, ii) correlated complex Gaussian
noise with different values of standard deviations is added to each
coil image, iii) 2D FFT and data subsampling with chosen reduction
factor $r$ is performed and iv) 2D iFFT is applied to recover data
in \textbf{x}-space. Then, data reconstruction process is conducted.

\textbf{\emph{Module output}}: The output is full resolution reconstructed
data performed with two different algorithms: SENSE (LSE) and Tikhonov
regularization. \\

\section{Module 2. Intensity inhomogeneity correction}
The aim of this module is to remove intensity inhomogeneity from MR image. Usually after MR imaging there are brighter and darker spots in the image of brain. The bias is very undesirable especially for the next modules such as segmentation and 3D imaging. The expected result of this module is that every kind tissue would be presented by approximately one shade of gray. As we never know how the bias map signal will look like, a method of removing it, which is be able to adapt to any shape of the inhomogeneity map is required. To achieve the goal, the following steps are taken. 
Firstly, the reconstructed image with intensity inhomogeneity is read (Fig. \ref{fig: Module2_1}).
\begin{figure}[H]
\centering{}\includegraphics[scale=0.7]{figures/Module_02/m2_3}\caption{The original image with visible intensity inhomogeneity} 
\label{fig: Module2_1}
\end{figure}
The next step is extracting the background of the image with Gaussian Filter of a size equal 2/3 the size of MRI image. In this step all the details corresponding to high frequency components are filtered out. The gaussian filter used in this part takes two variables. The first one is the size of the bandwidth and the second is the standard deviation sigma. For a picture of size 256x256 the size of the filter is set as 170x170 and the sigma is set as 20. Many tests have shown that too large sigma would reduce too much details from the background while too little sigma would leave too much details (Fig. \ref{fig: Module2_5}). The appropriate background extraction is shown in the Fig. \ref{fig: Module2_2}. 
\begin{figure}[H]
\centering{}\includegraphics[scale=0.7]{figures/Module_02/m2_5}\caption{The filtered image with sigma = 5 (left) and sigma = 150 (right)}
\label{fig: Module2_5}
\end{figure}
\begin{figure}[H]
\centering{}\includegraphics[scale=0.7]{figures/Module_02/m2_2}\caption{The image after Gaussian filtering with sigma = 20}
\label{fig: Module2_2}
\end{figure}
Next there are 150 datapoints randomly selected from the background image. Every datapoint has 3 dimensions: x and y coordinates and z as a level of gray. They are used in next steps to fit a surface to the background. Following 3rd degree polynomial was set as the parametric equation of the surface:
\begin{equation}
\begin{aligned}
f(x, y) = a+(b*x)+(c*y)+(d*x^{2})+(e*y^{2})+(f*x*y)+(g*x^{3})+(h*y^{3})+(i*x^{2}*y)+(i*x*y^{2})
\end{aligned}
\end{equation}
The fitting of the surface was performed with an $curve_fit$ function from the scipy module. $*describe in a few words how the function works*$ The equation is used to generate an image of bias field signal. The result is shown in the fig. \ref{fig: Module2_3}.
\begin{figure}[H]
\centering{}\includegraphics[scale=0.7]{figures/Module_02/m2_1}\caption{The surface fitted to the filtered image}
\label{fig: Module2_3}
\end{figure}
The last step is dividing each element of the original image by the corresponding element of the image of bias field performed with numpy function divide(). To avoid dividing by 0, all the pixels in the bias field image that were equal 0 are changed into value 1. The final result of all the following operations is shown in the fig. \ref{fig: Module2_4}.
\begin{figure}[H]
\centering{}\includegraphics[scale=0.7]{figures/Module_02/m2_4}\caption{The result with intenstity bias reduced}
\label{fig: Module2_4}
\end{figure}

\section{Module 3. Non-stationary noise estimation}

Magnetic Resonance Imaging (MRI) is known to be affected by several
sources of quality deterioration, due to limitations in the hardware,
scanning times, movement of patients, or even the motion of molecules
in the scanning subject. Among them, noise is one source of degradation
that affects acquisitions. The presence of noise over the acquired
MR signal is a problem that affects not only the visual quality of
the images, but also may interfere with further processing techniques
such as registration or tensor estimation in Diffusion Tensor MRI.

The aim of this module is to estimate the spatial dependent pattern
of the variance of noise in SENSE reconstructed images. For this to
work some additional information must be known beforehand, such as
the sensitivity maps of each receiver coil. In the background of a
SENSE MR image, where the SNR is zero, the Rician PDF (Probability
density function) simplifies to a (non-stationary) Rayleigh distribution,
whose second order moment is defined as:

\begin{equation}
\begin{aligned}E\left\{ M^{2}\left(x\right)\right\} =2\cdot\sigma_{R}^{2}\cdot\left(x\right)\end{aligned}
\end{equation}

Since $\sigma_{R}^{2}$ is \textit{x}-dependent, $E\left\{ M^{2}\left(x\right)\right\} $
will also show a different value for each \textit{x} position.

Let us assume that each coil in the \textit{x} -space is initially
corrupted with uncorrelated Gaussian noise with the same variance
$\sigma_{n}^{2}$ and there is a correlation between coils $\rho$
so that matrix $\sigma$ becomes:

\begin{equation}
\begin{aligned}\Sigma=\sigma_{n}^{2}\begin{pmatrix}1 & \rho & \cdots & \rho\\
\rho & 1 & \cdots & \rho\\
\vdots & \vdots & \ddots & \vdots\\
\rho & \rho & \cdots & 1
\end{pmatrix}=\sigma_{n}^{2}\left(I+\rho\left[1-I\right]\right)\end{aligned}
\end{equation}
with \textit{I} the $L\times L$ identity matrix and \textit{1} a
$L\times L$ matrix of 1's. For each \textit{x} value, we define the
global map

\begin{equation}
\begin{aligned}G_{w_{i}}=W_{i}^{*}\left(I-\rho\left[1-I\right]\right)W_{i},\quad i=1,\cdots,r\end{aligned}
\end{equation}

Global map $G_{w}(x)$ can be easily inferred from $G_{w_{i}}$ values.
Note that $G_{w}(x)$ is strongly related fo the g-factor, so the
first equation becomes

\begin{equation}
\begin{aligned}E\left\{ M^{2}\left(x\right)\right\} =2\cdot\sigma_{n}^{2}G_{w}\left(x\right)\end{aligned}
\end{equation}

and

\begin{equation}
\begin{aligned}\sigma_{n}^{2}=\frac{E\left\{ M^{2}\left(x\right)\right\} }{2G_{w}\left(x\right)}\end{aligned}
\end{equation}

By using this regularization we can assure a single $\sigma_{n}^{2}$
value for all the points in the image. We can now define a noise estimator
based on the local sample estimation of the second order moment:

\begin{equation}
\begin{aligned}\left\langle M^{2}(x)\right\rangle _{x}=\frac{1}{\left|\eta(x)\right|}\sum_{p\in\eta(x)}M^{2}(p)\end{aligned}
\end{equation}
with $\eta(x)$ a neighborhood centered in x. $\left\langle M^{2}(x)\right\rangle _{x}$
is know to follow a Gamma distribution whose mode is $\sigma_{n}^{2}(\left|\eta(x)\right|-1)/\left|\eta(x)\right|$.
Then

\begin{equation}
\begin{aligned}mode\left\{ \frac{\left\langle M_{L}^{2}\right\rangle _{x}}{G_{w}(x)}\right\} =2\sigma_{n}^{2}\frac{\left|\eta(x)\right|-1}{\left|\eta(x)\right|}\approx2\sigma_{n}^{2}\end{aligned}
\end{equation}
when $\left|\eta(x)\right|\gg1$. The estimator is the defined as

\begin{equation}
\begin{aligned}\widehat{\sigma_{n}^{2}}=\frac{1}{2}mode\left\{ \frac{\left\langle M_{L}^{2}(x)\right\rangle _{x}}{G_{w}(x)}\right\} \end{aligned}
\end{equation}
and consequently the noise in each pixel is estimated as

\begin{equation}
\begin{aligned}\widehat{\sigma_{R}^{2}}=\frac{1}{2}mode\left\{ \frac{\left\langle M_{L}^{2}(x)\right\rangle _{x}}{G_{w}(x)}\right\} G_{w}(x)\end{aligned}
\end{equation}

This estimator is only vaild over the background pixels. However,
no segmentation of these pixels is needed: the use of the mode allows
us to work with the whole image. On the other hand, to carry out the
estimation, the sensitivity map of each coil and the correlation between
coils must be known beforhand. These parameters are needed for the
SENSE encoding, and thus, they can be easily obtained.

\section{Module 4. Non-stationary noise filtering 1}

The unit tests was implemented to check how the programme behaves in different conditions. The few tests

\begin{itemize}

	\item \textbf{Test\_empty\_input\_data} - test checks how the code behave in case receiving empty dataset,
	\item \textbf{Test\_incorrect\_input\_data} - test checks how the code behave, when incorrect data is passed to the function,
	\item \textbf{Test\_size\_input\_data} - test checks if the size of data is changed on the output of code,
	\item \textbf{Test\_none\_noise\_map} - if code received empty data of noise maps, the function should rise an error,
	\item \textbf{Test\_LMMSE\_function\_changes} - test checks if the input data was changed in the LMMSE-filter function.

\end{itemize}

The results of tests were as it was expected and code behaves in correct way.

\section{Module 5. Non-stationary noise filtering 2}

Unbiased Non-Local Means algorithm was implemented based on \cite{5a1}. Some, it is believed, improvements to original idea were introduced and described below. Implementation of the UNLM algorithm was divided into several steps:
\begin{itemize}
	\item rewriting NLM prototype from MATLAB to Python,
	\item refactoring to UNLM,
	\item testing different Gaussian kernels,
	\item optimization of algorithm in order to reduce execution time,
	\item adjusting the algorithm to fit both types of data (structural and diffusion weighted).
\end{itemize}
Because some works and further improvements were done in parallel it is hard to describe implementation in chronological order.

\subsection*{Comparison of different Gaussian kernels}
Once the main body of algorithm was ready, it was decided to run some tests to examine the influence of various Gaussian kernels on the output of the UNLM function. The main requirement for the Gaussian kernel was the fact that the overall sum of all elements within kernel should be equal to 1. Another, for sake of sanity, requirement built on the shape of the kernel surface - the slope should be gentle in order to avoid 'overfitting' to the central pixel.

\begin{figure}[H]
\centering{}
\includegraphics[scale=0.7]{figures/module05/gk1}
\caption{First tested Gaussian kernel.} 
\end{figure}

\begin{figure}[H]
	\centering{}
	\includegraphics[scale=1]{figures/module05/gk1results}
	\caption{Results obtained for first kernel.} 
\end{figure}

First kernel was presented in the article describing the algorithm \cite{5a1}. However, kernel was considered as not optimal due to its two-level shape. To overcome the problem of wrong shape another attempt took place. Second generated kernel had desired shape, yet values didn't meet the requirement of summing to 1.

\begin{figure}[H]
	\centering{}
	\includegraphics[scale=0.7]{figures/module05/gk2}
	\caption{Second tested Gaussian kernel.} 
\end{figure}

\begin{figure}[H]
	\centering{}
	\includegraphics[scale=0.8]{figures/module05/gk2results}
	\caption{Results obtained for second kernel.} 
\end{figure}  

Third kernel, again was disappointing. It had a shape of first kernel, but represented different values, since it was generated in other manner.

\begin{figure}[H]
	\centering{}
	\includegraphics[scale=0.7]{figures/module05/gk3}
	\caption{Third tested Gaussian kernel.} 
\end{figure}

\begin{figure}[H]
	\centering{}
	\includegraphics[scale=0.8]{figures/module05/gk3results}
	\caption{Results obtained for third kernel.} 
\end{figure}

Fourth kernel, had all properties that were desired. It was decided to use this particular one in implementation of the UNLM algorithm.

\begin{figure}[H]
	\centering{}
	\includegraphics[scale=0.7]{figures/module05/gk4}
	\caption{Fourth, and final, tested Gaussian kernel.} 
\end{figure}

\begin{figure}[H]
	\centering{}
	\includegraphics[scale=1]{figures/module05/gk4results}
	\caption{Results obtained for fourth kernel.} 
\end{figure}

Someone with an eye for detail can spot the differences between results achieved whilst using all kernels. Most eye-catching distinction lies in boundaries of the brain and skull.   


Following code snippet presents the implementation of Gaussian window, that was finally chosen after running some tests. This particular window has a special characteristic of being shallow - the central pixel's value doesn't stand out from values in close vicinity. 

\begin{lstlisting}[language=Python, caption = Code used for Gaussian kernel generation.]
def gk4(rsim):

# function that prepares Gaussian window used to penalize
# pixels based on distance within window
kerlen = 2 * rsim + 1
nsig = 2
interval = (2*nsig+1.)/kerlen
x = np.linspace(-nsig-interval/2., nsig+interval/2., kerlen+1)
kern1d = np.diff(st.norm.cdf(x))
kernel_raw = np.sqrt(np.outer(kern1d, kern1d))
kernel = kernel_raw/kernel_raw.sum()
return kernel
\end{lstlisting}

\subsection*{Reducing execution time}
After line-by-line translation of code from MATLAB to Python filtration of single slice lasted for 180 seconds. That was unacceptable, so few flaws in the code had to be identified. It's worth mentioning that single slice filtering in MATLAB took 19 seconds on average.

The first faced incorrectness was the fact of not fully using the numpy library utilities. Changing 'pure python' functions like \textit{sum} to numpy equivalents allowed reduction of time to 80 seconds per slice. That run time was still insufficient. After code profiling with \textit{cProfile} library it was identified, that nested for loops used to move around image were most time consuming. As a result it was decided to get rid off loops and change tedious loop-based architecture to multi-dimensional matrices operations. That operation resulted in achieving 3.5 seconds on average. Involving matrices operations allows to fully exploit numpy implementation (which itself is based on C, so it is faster than Python).

Furthermore, an attempt to use threads in order to speed up calculations on multi-dimensional data took place. For that test, a synthetic 3D data was generated. Processing each slice in a for loop takes 3.5 x \textit{number of slices} seconds. Involving threads should reduce the time, but unfortunately it was doubled. So a tough decision of processing slices in classic loop has been made. Other possibilities of parallel computing were not taken into consideration due to the compatibility issues with GUI.

\subsection*{Unbiased Non-Local Means algorithm}
Having reduced the computation time and having chosen suitable penalty function (Gaussian kernel) algorithm was considered finished. Some of the most important code features are presented in snippets below.

\begin{lstlisting}[language=Python, caption = Overcoming computation of same matrices in every iteration.]
# precompute possible window2
window2_precomp = np.zeros((m, n, rsearch, rsearch))
for i in range(2, m):
for j in range(2, n):
window2_precomp[i, j, :, :] = img_ext[i - rsim:i + rsim + 1, j - rsim:j + rsim + 1]
\end{lstlisting}

During thorough analysis it was discovered that q pixel windows can be calculated only once. Fact of precomputing windows allowed to save resources, which are really important in computationally complex problems.

Moving widows, which are visualized later in this section are changing positions in each iteration. Code snippet below shows how p pixel window and search space window are determined.

\begin{lstlisting}[language=Python, caption = Calculation of p pixel window and search space area.]
# p pixel loop
for i in range(0, m):
for j in range(0, n):
ii = i + rsim
jj = j + rsim
window1 = np.asarray(img_ext[ii - rsim:ii + rsim + 1, jj - rsim:jj + rsim + 1])

# limit the search space
pmin = max(ii - rsearch-1, rsim)
pmax = min(ii + rsearch - 1, m)
qmin = max(jj - rsearch-1, rsim)
qmax = min(jj + rsearch - 1, n)
\end{lstlisting}

Using precomputed set of all possible q pixel widows and using only those that fit the search space was a smart trick presented below.

\begin{lstlisting}[language=Python, caption = Calculation of q pixel windows within search space area.]
window2_slice = window2_precomp[pmin:pmax, qmin:qmax, :]
\end{lstlisting}

To explain the code and present it in more picturesque way, one can look at the image filtering as moving set of
2D windows on image and calculating some metrics, according to their positions and values. There are two types of windows: neighbourhood and search space. Size of neighbourhood window is indirectly defined by rsim parameter, which was assigned by default to 2. That results in neighbourhood window of size 5x5, becasue 5 = 2 * rsim +1. Search space window is dependent ot rsearch parameter, which was assigned to 5 giving 11x11 window. The values of these parameters were set based on analysis conducted in \cite{5a1}. Authors discovered that these sizes give best results in filtration.

Neighbourhood windows are centered around pixels p and q, which are pixel being filtered and filtering pixel correspondingly. Figure below presents the idea of moving windows.

\begin{figure}[H]
	\centering{}
	\includegraphics[scale=0.7]{figures/module05/m5windows}
	\caption{Graphical representation of moving windows.} 
\end{figure}

Given figure has to be explained. Gray area represents 2D picture, the grid on in resembles pixels. Dark green square (pixel) is the p pixel and the light green area around it is the neighbourhood window defined by rsim. Similarly, purple area and dark purple pixel resembles window of pixel q. The size of the image has to be extended by rsim in each direction to make calculations on edges of original image possible. This scenario is presented in the top left corner of picture. Pixel p is within area of image but its window doesn't fit into original image. As a consequence of extending the image, the orange area is temporarily added to the image. The values in the added space are assigned by symmetrical padding near edges. Last thing that has to be deciphered is blue square. It is the search space window. It narrows the possibilities of q pixel positions taken into consideration whilst calculating distances between p and q windows.

It is worth mentioning that there exist special case, when q pixel is in the same position as p pixel. If that situation would have been treated equally to other, what would introduce some bias towards 'self-distance'. According to the equation used to calculate distance (d) between windows, one has to calculate element-wise difference of windows. Having both windows in the same position would result in distance equal to 0, which further can have negative influence on the output. To overcome this problem, maximum weight found within the window is assigned to a weight calculated based on 'zero' distance.


\subsection*{Adjusting the algorithm for both types of data}
Denoising using UNLM method is used by both data types, structural and diffusion weighted. There are some differences in implementation of the algorithm for both types, however the same code is used for them in this module. It is caused by analysis of diagrams presented in \cite{5a2}. They showed that taking gradient information gives no significant information in filtering result. On this occasion, the only difference in handling both types of data lays in main function of the module. More precisely, after making decision which data is being processed loops work on different dimensions. Following code snippet presents the \textit{run\_module} function, where data is being accurately handled.

\begin{lstlisting}[language=Python, caption = run\_module function.]
def run_module(mri_input, other_arguments=None):

if isinstance(mri_input, smns.mri_diff):
[m, n, slices, grad] = mri_input.diffusion_data.shape
data_out = np.zeros([m, n, slices, grad])

for i in range(slices):
for j in range(grad):
data_out[:, :, i, j] = unlm(mri_input.diffusion_data[:, :, i, j], mri_input.noise_map[:, :, i, j])

mri_input.diffusion_data = data_out

elif isinstance(mri_input, smns.mri_struct):
[m, n, slices] = mri_input.structural_data.shape
data_out = np.zeros([m, n, slices])

for i in range(slices):
data_out[:, :, i] = unlm(mri_input.structural_data[:, :, i], mri_input.noise_map[:, :, i])

mri_input.structural_data = data_out

else:
return "Unexpected data format in module number 5!"

return mri_input
\end{lstlisting}

One decision must be finally justified. The whole bigger product is targeted (theoretically) at physicians that usually have no knowledge of image filtering. On this occasion it was decided to reduce number of parameters responsible for algorithm outcome to 0. That solution should reduce possibilities of getting badly-processed data, as optimal parameters are set by default.


\section{Module 6. Diffusion tensor imaging}

\textbf{Preprocessing and Module I/O}

In order to improve diffusion tensor estimation it is imperative to
remove artifacts. In addition to standard MRI pre-processing, one
needs to correct for artifacts arising the use of diffusion-gradient
pulse sequences and longer acquisition time. While hardware manufacturers
try to proactively diminish some of these effects, software processing
is still mandatory. 
\hfill\\

\textbf{Module Input}:
\begin{itemize}
	\item 
	3D structural data array of shape X x Y x Z, where XY - pixel image intensities, Z - chosen slice, which is the T1- or T2-weighted image corresponding the the given DWI acquisition
	
	\item 
	4D diffusion data array of shape X x Y x Z x M, where XY - pixel image intensities, Z - chosen slice, M - applied diffusion gradient direction
	
	\item 
	b\_value, a scalar value corresponding to applied diffusion gradient sequence magnitude
	
	\item 
	2D gradients matrix of shape M x 3, where each row corresponds to a normalized $(x,y,z)$ components of diffusion gradient sequence vectors
		
	\item 
	optionally - 3D binary mask of shape X x Y x Z, corresponding to the brain area detected by Module 8 (Skull Stripping); if not supplied, DTI is computed on each input data voxel

\end{itemize}
\hfill

\textbf{Module Output}:
\begin{itemize}
	\item
	list of size Z, corresponding to each slice; every list element is a dictionary of biomarker images: MD, RA, FA, VR of shape X x Y, and biomarker FA\_rgb of shape X x Y x 3
\end{itemize}

\subsection{Initialization}

In order to abstract DTI implementation from end-user, all classes and methods other than the main function \texttt{run\_module} are private to module source code script. It is important to note that prior to running the module one has to provide the module with input data object, as well as SOLVER and FIX\_METHOD parameters. SOLVER passed as an argument decides whether to use WLS or NLS estimation, whole FIX\_METHOD decides how to "fix" negative eigenvalues. 

As mentioned in the detailed description chapter, 'ABS' takes absolute value of each eigenvalue, while 'CHOLESKY' ensures that the estimated tensor is positive definite. Eigenvalues of positive definite matrices are always non-negative. 'ABS' is a post-estimation fix, meaning that it does not modify the default estimation algorithm (i.e. it is applied after WLS or NLS computation), while 'CHOLESKY' directly modifies the expressions for WLS and NLS cost function gradients and Hessian matrices.

After passing all required arguments to the \texttt{run\_module} function, they are reshaped internally in order to be compatible with module. Concretely, \texttt{DTISolver} class instance, computing the DTI proper, assumes that input data argument is a concatenated 3D array of both structural and diffusion images, which are stored separately in the original data structure. Moreover, b\_value and gradient fields are reshaped to be lists correpsonding to each slice of the new data array (that is: b\_value is repeated in length while both have zeros appended that correspond to structural images). Finally, all of the above is done separately for each slice and DTI module performs it's computation slice-by-slice due to memory constraints.

\subsection{WLS-ABS estimation}

WLS with the ABS fix method is the fastest yet simple method of module pipeline computation based on diffusion tensor estimation. As such these parameters were set as default for DTI.

Diffusion tensor estimate was computed by implementing the equation:
\begin{equation}
\begin{aligned}
\boldsymbol{\gamma}=\left(\boldsymbol{W}^T\boldsymbol{\omega}^T\boldsymbol{\omega}\boldsymbol{W}\right)^{-1}\boldsymbol{W}^T\boldsymbol{\omega}^T\boldsymbol{\omega y}
\end{aligned}
\label{Eq:m6_impl_eq_1}
\end{equation}

using NumPy matrix broadcasting operations, effectively abstracting away array reshaping. Weights vector $\boldsymbol{\omega}$ is calculated using a separate function in order to avoid changing every piece of code refering to WLS weights in case they change. The following implementation assumes the simplest of models presented in the Detail Description chapter, that is weights being equal to the measured signal.

\subsection{NLS-ABS estimation}

In case of NLS estimation, in addition to implementing gradient and Hessian matrix computation methods:

\begin{equation}
\begin{aligned}
\nabla{f_{NLS}}&=-\boldsymbol{W}^T\boldsymbol{\hat{S}}\boldsymbol{r} \\
\nabla^2{f_{NLS}}&=\boldsymbol{W}^T\left(\boldsymbol{\hat{S}^T\hat{S}-\boldsymbol{R\hat{S}}}\right)\boldsymbol{W}
\end{aligned}
\label{Eq:m6_impl_2}
\end{equation}

It is important to devise an iterative scheme because gradient result depends on NLS diffusion tensor estimate. For that reason an algorithm based on \cite{m6_koay2006a} has been implemented. The method itself is called a Modified Newton's Algorithm and can be summarised as in Fig.\ref{fig:m6_pic_1}.

\begin{figure}[H]
	\includegraphics[width=8cm]{figures/Module_06/mfn_simple}
	\centering
	\caption{Modified Newton's method for iterative computation of NLS estimate \vbox{(based on \cite{m6_koay2006a})}}.
	\label{fig:m6_pic_1}
\end{figure}

The following parameters (collectively known in code as MFN parameters) were set:
\begin{itemize}
	\item 
	MFN\_MAX\_ITER = 3 - iteration limit
	
	\item
	MFN\_ERROR\_EPSILON = 1e-5 - first convergence criterion (error change is small)
	
	\item
	MFN\_GRADIENT\_EPSILON = 1e-5 - second convergence criterion (vanishing gradient)
	
	\item
	MFN\_LAMBDA\_MATRIX\_FUN = 'identity'- regularization matrix added to Hessian matrix
	
	\item 
	MFN\_LAMBDA\_PARAM\_INIT = 1e-4 - initial regularization matrix multiplier
\end{itemize}

Delta estimate is calculated from the following formula:
\begin{equation}
\boldsymbol{\delta}=-\left(\nabla^2{f_{NLS}+\lambda I}\right)^{-1}\nabla{f_{NLS}}
\label{Eq:m6_impl_3}
\end{equation}


\section{Module 8. Skull stripping}

\section{Module 9. Segmentation}

Unity tests are designed to checked if smaller unit of module is running correctly. In this module, the unity tests are using to test each function in module for different conditions. implemented unity test provide information, what happens in module after changing the source code. 

\textbf{\textit{functionality of imHist function}} 
Unity test is checking functionality for different data type, data with only 0 values, empty dataset. Module returns correctly data only for double data type. Unexpected result should be shown to the user.

\textbf{\textit{functionality of gmm function}}
Unity test is checking functionality for different data type, data with only 0 values, empty dataset and missing arguments. Checking the function behavior for different data sizes is also very important. Valid, because of using matrix operation. 
Module returns correctly data only for double, nonzero array data, with the same size of expected values vector (mu), variation vector (v), probability vector (p). Unexpected result should be shown to the user.

\textbf{\textit{functionality of imPart function}}
\begin{itemize}
	\item Checking value of lastPitch, if the value is greater than pitches or smaller than firstPitch, function returns an error.
	\item Checking value of firstPitch, if the value is negative, greater than lastPitch or pitches, function returns an error.
	\item Checking the correct size of image data for the selected part. If the separated part of the data does not have a size equal to [rows x column x difference between the lastPitch and the firstPitch], the function should not return the image and returns an error
\end{itemize}


\textit{\textbf{functionality of segmentation function}}
\begin{itemize}
	\item Loading an image automatically starts the algorithm and calls all functions.
	\item An unexpected action of any of the functions immediately interrupts the program and the function returns an error.
	\item Checking the size of initialized parameters vectors, if are not the same, function returns an error. 
	\item Testing exit condition of while loop in function. If the loop operation repeats over 100 times, the function will be immediately interrupted and return an error. 
	\item Checking the correct size of segmentation image mask. It should be the same as the size of the data after calling the function partIm, in other way function returns an error.
	\item Checking mask contents after segmentation. The mask should contain four data clusters for: background, cerebrospinal fluid,  gray matter, white matter, with values, respectively, 1, 2, 3 and 4, which is corresponding to image histogram.
\end{itemize}



\section{Module 10. Upsampling}

-code

\section{Module 11. Brain 3D}

\indent To prepare tree dimension visualization of the cerebral cortex algorithm of marching cubes
is used.\\
 \indent The input data is multiple 2D slices of MR image. The marching cubes algorithm create a polygonal representation of constant density surfaces from a 3D array of data. To select the cerebral cortex is used output data from segmentation made in module 9. The structure of cortex is represented by value 3 in segmentation mask. 
\indent The space of the image is divided into a regular grid of cubes. In each iteration one cube is considered. At each vertex of cube is determined how the surface intersects this cube. The density value are compared with the limit value - surface constant. If the data value is bigger than suface constant, one is assinged to a cube’s vertex. There are 256 combinations of cube orientation relative to the surface, but we can distinguish 15 basic patterns, that repeat as symmetrical reflections, produces all possibilities (Fig. \ref{fig:figures/Marching cubes}). If all values are less than the constant value, then the cube does not form any polygon. Otherwise, the edges of the polygon are defined (by linear interpolation) at the edges that intersect the surface. Using central differences, a unit normal at each cube vertex is calculated and then normal to each trangle vertex is interpolated. The output of the algorithm is the triangle vertices and vertex normals.

\begin{figure}[H]
\centering{}\includegraphics[scale=0.7]{figures/MarchingCubes}\caption{Triangulation for the 15 patterns. \label{fig:figures/Marching cubes}}
\end{figure}

\indent To visualization the model, obtained by marching cubes, the
VTK library is used, which enables building the three-dimension model.
The VTK is object-oriented library. The classes of VTK is dedicated to processing and visualization data.
\\
 \indent The second part of this module includes visualization of the brain’s cross-section on arbitrarily defined plane. To enable selecting of intersection plane there was used object of VTK class, which allows set plane in elected direction by using computer mouse. When the plane is moved by user, in the real time the  three-dimensional model is clipped in the place of selected plane. To improve quality of visualization the cross-section image, there is also possibility to see the image imposed on the three-dimensional model. \\



\section{Module 12. Oblique imaging}

\indent The literature to this module is as useful as nipples on
men. Everything is about inventing how to realize point 1. of the
list. \\
 \indent Oblique Imaging is a technique to create non-perspective
projections from 3D or multiple 2D images.\\

\indent In order to create oblique image it is essential to: 
\begin{itemize}
\item choose two angles under which the plane will be inclined, 
\item create a matrix of points that this plane consists of, 
\item from existing points pick those, which will be used in the image, 
\item interpolate points that are not existing. 
\end{itemize}
\indent Type of interpolation can vary, but in this project interpolation
based on mean will be used. To interpolate one pixel mean of all pixels
around him with given proximity is taken.

\begin{figure}[H]
\centering{}\includegraphics[scale=0.7]{figures/m11_spherexyz}\caption{Visualization of pixels taken to interpolate}
\label{fig:figures/m11_spherexyz } 
\end{figure}

\input{"Tests/Application"}


\chapter{Authors}

Authors of this project are students of Biomedical Engineering, AGH
UST, Krakow, Poland. \\

\begin{center}
\begin{tabular}{|c|c|}
\hline
Name  & Role \tabularnewline
\hline
\hline
Sylwia Mól  & Project Manager\tabularnewline
\hline
Jacek Fidos  & Software architect\tabularnewline
\hline
Maciej Gryczan  & GUI engineer\tabularnewline
\hline
Adrian Stopiak  & Vizualization engineer \tabularnewline
\hline
Malwina Molendowska  & 1st module developer \tabularnewline
\hline
Klaudia Gugulska  & 2nd module developer \tabularnewline
\hline
Kacper Turek  & 3rd module developer \tabularnewline
\hline
Magdalena Rychlik  & 4th module developer \tabularnewline
\hline
Alicja Martinek  & 5th module developer \tabularnewline
\hline
Mateusz Pabian  & 6th module developer \tabularnewline
\hline
Anna Grzywa  & 8th module developer \tabularnewline
\hline
Magdalena Kucharska  & 9th module developer \tabularnewline
\hline
Eliza Kowalczyk  & 10th module developer \tabularnewline
\hline
Karolina Gajewska  & 11th module developer \tabularnewline
\hline
Michał Kotarba  & 12th module developer \tabularnewline
\hline
\end{tabular}
\par\end{center}

\newpage{}

\listoffigures

\printbibliography

\begin{comment}
 \bibliographystyle{plain}
\bibliography{bibliografia}
\end{comment}

\end{document}

\grid
\grid
\grid
