%% LyX 2.2.1 created this file.  For more info, see http://www.lyx.org/.
%% Do not edit unless you really know what you are doing.
\documentclass[english]{aghdpl}
\usepackage[T1]{fontenc}
\usepackage[utf8]{inputenc}
\setcounter{secnumdepth}{3}
\setcounter{tocdepth}{3}
\usepackage{babel}
\usepackage{verbatim}
\usepackage{float}
\usepackage{mathtools}
\usepackage{amsmath}
\usepackage{graphicx}
\usepackage[unicode=true,pdfusetitle,
 bookmarks=true,bookmarksnumbered=true,bookmarksopen=true,bookmarksopenlevel=3,
 breaklinks=false,pdfborder={0 0 1},backref=false,colorlinks=false]
 {hyperref}
\usepackage{breakurl}

\makeatletter

%%%%%%%%%%%%%%%%%%%%%%%%%%%%%% LyX specific LaTeX commands.

\newcommand*\LyXZeroWidthSpace{\hspace{0pt}}
%% Because html converters don't know tabularnewline
\providecommand{\tabularnewline}{\\}

%%%%%%%%%%%%%%%%%%%%%%%%%%%%%% User specified LaTeX commands.
\usepackage{polski}




\usepackage{amsfonts}
\usepackage{amsthm}


\usepackage[
style=numeric,
sorting=none,
language=autobib,
autolang=other,
urldate=iso8601,
backref=false,
isbn=true,
url=false,
maxbibnames=3,
backend=biber,
]{biblatex}
\usepackage{csquotes}
\DeclareQuoteAlias{croatian}{polish} 


\addbibresource{bibliografia.bib}

\AtBeginDocument{

}

\author{SIMENS Healthkare}
\shortauthor{SIMENS}
\titlePL{DADM project documentation - draft}
\titleEN{}
\shorttitlePL{DADM project documentation}
\shorttitleEN{DADM project documentation}
\thesistype{}
\supervisor{PhD. Tomasz Pięciak}
\degreeprogramme{Biomedical engineering}
\date{2017}
\department{Department of  Automatics and Biomedical Engineering}
\faculty{AGH University of Science and Technology\protect\\[-2mm]
Faculty of Electrical Engineering, Automatics,\protect\\[-2mm] Computer Science and Biomedical Engineering}
\acknowledgements{}
\setlength{\cftsecnumwidth}{10mm}





\usepackage[english]{babel}

\makeatother

\begin{document}
\titlepages

\fancypagestyle{plain} { \fancyhf{} \global\long\def\headrulewidth{0pt}
 \global\long\def\footrulewidth{0pt}
 }

\setcounter{tocdepth}{2}

\tableofcontents{}

\clearpage{}

\chapter{List of changes}

\begin{tabular}{|c|c|c|}
\hline 
Name  & Date  & Details\tabularnewline
\hline 
\hline 
Sylwia Mól  & 19-Nov-2017  & Document created\tabularnewline
\hline 
Sylwia Mól  & 20-Nov-2017  & Structure changed\tabularnewline
\hline 
Sylwia Mól  & 21-Nov-2017  & Chapter \textquotedbl{}Authors\textquotedbl{} added, in-out table
added\tabularnewline
\hline 
Malwina Molendowska  & 26-Nov-2017  & Description of 1st module added\tabularnewline
\hline 
Eliza Kowalczyk  & 27-Nov-2017  & Description of 9th module added\tabularnewline
\hline 
Karolina Gajewska  & 27-Nov-2017  & Description of 10th module added\tabularnewline
\hline 
Eliza Kowalczyk  & 28-Nov-2017  & Description of 9th module changed\tabularnewline
\hline
Alicja Martinek  & 28-Nov-2017  & Description of 5th module changed\tabularnewline
\hline 
 &  & \tabularnewline
\hline 
\end{tabular}

\chapter{Assumptions}

about the app - the aim, what you can do here etc.

\chapter{Structure}

dependences (tree), modules` descriptions \\
 %
\begin{tabular}{|c|c|c|c|}
\hline 
Module  & Input  & Output  & Before that module\tabularnewline
\hline 
\hline 
1 & \textbf{k}-space signals  &  \textbf{x}-space fully reconstructed data  & ——–\tabularnewline
\hline 
2 &  &  & \tabularnewline
\hline 
3  &  &  & \tabularnewline
\hline 
4 &  &  & \tabularnewline
\hline 
5 & reconstructed, normalized image & Rician noise-free image & reconstruction \tabularnewline
\hline 
6 &  &  & \tabularnewline
\hline 
8 &  &  & \tabularnewline
\hline 
9 & 320x240 image  & 640x480 image  & denoised data \tabularnewline
\hline 
10  & 1)segmentated image 2)defined plane, image  & 1)3D model (e.g. vtkPolyData) 2)cross-section image  & 1)segmentation 2) denoised data\tabularnewline
\hline 
11 &  &  & \tabularnewline
\hline 
\end{tabular}

\chapter{User guide}

\section{Requirements}

what user need to use this app - e.g. windows version etc

\section{Instruction}

instructions for user - GUI screens etc 

\chapter{Detailed description}

\section{Module 1}

The aim of this module is to formulate mathematical algorithm, which
enables proper data reconstruction for images obtained with parallel
MRI scans. The reconstruction is performed with use of Sensitivity
Encoding (SENSE) algorithm in least squares (LS) solution context
and Tikhonov regularization method.

Generally, parallel MRI acquisitions are targeted to diminish time
needed for data sampling. The usage of multiple coils enabled simultaneous
acquisition of signals. A further step, which is acquiring partial
data from \textbf{k}-space, leads to craved time savings, meanwhile
maintaining full spatial resolution as well as contrast at the same
time. However, the approach of omitting lines in acquisition step
results in data aliasing, i.e. folded images that need further data
processing.

To clearly mark out how data is processed in this module, we list
following reconstruction steps: i) the application of 2D Fourier Transform
transform (2D FFT) to \textbf{k}-space data (acquired raw signals)
from multiple coils. The result is a set of \textbf{x}-space images
with folded pixels, ii) the sensitivity maps estimation of coils profiles
(the information is needed to properly unfold subsampled data) and
iii) the proper unfolding data process with usage of SENSE reconstruction
algorithm and its alterations.

The most crucial step in processing is estimation of sensitivity coil
profiles as a successful image reconstruction with use of pMRI algorithms
highly depends on accurate sensitivity coil assessment. As sensitivity
information varies from scan to scan it is impossible to obtain absolute
maps. To obtain reliable knowledge, reference scans have to be conducted
each time an examination is performed. These low-resolution information
helps to estimate coil profiles with use of the many methods i.e.
dividing each component coil image by a 'sum of squares' image.

It basic formulation, SENSE algorithm is applied to Cartesian MRI
data undersampled uniformly by a factor $r$~(i.e. $r=2$ means that
every other line in \textbf{k}-space is skipped). After Fourier transformation,
each pixel in \textbf{x}-space image received in \textit{l}-th coil
can be seen as weighted sum of $r$ pixels from full FOV, each multiplied
by corresponding localized values of maps. The distance between those
'aliased' points in the full FOV is always equal to the desired FOVy
value divided by subsampling rate. Obviously, depending on subsampling
rate the number of folded pixels changes. Basically, the signal in
one pixel at a certain location $(x,y)$ received from $l$-th component
coil image $D_{l}^{S}$ with chosen subsampling rate $r$ can be written
as: 
\begin{equation}
D_{l}^{S}(x,y)=S_{l}(x,y_{1})D^{R}(x,y_{1})+S_{l}(x,y_{2})D^{R}(x,y_{2})+...+S_{l}(x,y_{r})D^{R}(x,y_{r}),\label{Eq:wzor1}
\end{equation}
where index $l$ counts from 1 to $L$ (number of coils) and index
$i$ counts from 1 to $r$. Eq.(\ref{Eq:wzor1}) can be rewritten
as:

\begin{equation}
D_{l}^{S}(x,y)=\sum_{i=1}^{r}S_{l}(x,y_{i})D^{R}(x,y_{i})\quad\text{for}\quad l=1,...,L.\label{Eq:wzor2}
\end{equation}

Including all $L$ coils the above equation can be rewritten in a
matrix form:

\begin{equation}
\textbf{D}^{S}(\textbf{x})=\textbf{S}(\textbf{x})\textbf{D}^{R}(\textbf{x}),\label{Eq:wzor3}
\end{equation}

The vector $\textbf{D}^{S}(\textbf{x})$ denotes the aliased coil
image values at a specific location \textbf{x} = $(x,y_{i})$ and
has a length of $L$, $\textbf{S}(\textbf{x})$ is a $L$x$R$ matrix
and represents the sensitivities values for each coil at the $r$~superimposed
positions and $\textbf{D}^{R}(\textbf{x})$ lists the $r$ pixels
from full FOV image to be reconstructed. The closed-form solution
of the problem is as follows: 
\begin{equation}
\widehat{\textbf{D}^{R}(\textbf{x})}=(\textbf{S}^{H}(\textbf{x})\textbf{S}(\textbf{x}))^{-1}\textbf{S}^{H}(\textbf{x})\textbf{D}^{S}(\textbf{x}),\label{Eq:wzor4}
\end{equation}
where $\widehat{\textbf{D}^{R}(\textbf{x})}=[\widehat{D^{R}(x,y_{1})},...,\widehat{D^{R}(x,y_{r})}]^{T}$
and $\textbf{S}^{H}(\textbf{x})$ is the conjugate transpose of the
$\textbf{S}(\textbf{x})$ matrix. The final reconstruction image is
defined as: 
\begin{equation}
M(\textbf{x})=\left|\widehat{\textbf{D}^{R}(\textbf{x})}\right|.\label{Eq:wzor5}
\end{equation}

The `unfolding' process can be performed as long as inversion of $\textbf{S}(\textbf{x})$
matrix is possible. Therefore, we cannot set the value of subsampling
rate exceeding the number of coils $L$. To restore full FOV data,
SENSE algorithm has to be recalled for each pixel in aliased \textbf{x}–space
image.

A regularization approach is defined as an~inversion method that
introduces additional information in order to stabilize the solution.
This method is beneficial as it roughly matches the desired solution
and is less sensitive to perturbations of the data. Tikhonov regularization
is a common approach to obtain an inexact solution to a~linear system
of equations. In particular, the Tikhonov regularized estimate reads
as follows:

\begin{equation}
\widehat{\textbf{D}_{reg}^{R}}=\text{arg}\underset{\textbf{D}^{R}}{\text{min}}\left\{ \left\Vert \textbf{D}^{S}-\textbf{S}\textbf{D}^{R}\right\Vert ^{2}+\lambda^{2}\left\Vert \textbf{A}(\textbf{D}^{R}-\textbf{D})\right\Vert ^{2}\right\} ,\label{Eq:wzor6}
\end{equation}

where $\lambda$ is a regularization parameter ($\lambda>0$) and
$\textbf{D}$ is a~prior image known as a regularization image. Selection
of the parameter $\lambda$ and $\textbf{D}$ can be performed using
different procedures. In this module $\textbf{A}$~is assumed to
be an identity matrix. The first term provides fidelity to the data
and the second introduces prior knowledge (e.x. median filtered initial
guess of LS SENSE) about the expected behaviour of $\textbf{D}^{R}$.
The Tikhonov regularization problem is given by:

\begin{equation}
\widehat{\textbf{D}_{reg}^{R}}=\textbf{D}+(\textbf{S}^{H}\textbf{S}+\lambda\textbf{A}^{H}\textbf{A})^{-1}\textbf{S}^{H}(\textbf{D}^{S}-\textbf{S}\textbf{D}).\label{Eq:wzor7}
\end{equation}

A reasonable value for $\lambda$~can be picked using many technique,
i.e. the L-curve criterion or generalized cross-validation.

\textbf{\emph{Module input}}: Synthetic MR images are brain MRI slices
coming from BrainWeb are normalized to {[}0-255{]} (all with intensity
non-uniformity INU=0). Only T1- and T2-weighted data is used. The
dataset is free of noise and the background areas are set to zero.
The slice thickness equals 1 mm. These images are used then to simulate
synthetic noisy accelerated parallel Cartesian SENSE MRI data according
to following steps (the data simulation is performed with use of eight
receiver coils ($L=8$)): i) simulated sensitivity maps (divided into
the ratio 3:1 for real and imaginary parts, respectively) are added
to fully-sampled \textbf{x}-space data, ii) correlated complex Gaussian
noise with different values of standard deviations is added to each
coil image, iii) 2D FFT and data subsampling with chosen reduction
factor $r$ is performed and iv) 2D iFFT is applied to recover data
in \textbf{x}-space. Then, data reconstruction process is conducted.

\textbf{\emph{Module output}}: The output is full resolution reconstructed
data performed with two different algorithms: SENSE (LSE) and Tikhonov
regularization. \\


\section{Module 2}

-detailed description of module, algorithm etc, NO CODES, only theoretical! 

\section{Module 3}

-detailed description of module, algorithm etc, NO CODES, only theoretical! 

\section{Module 4}

-detailed description of module, algorithm etc, NO CODES, only theoretical! 

\section{Module 5}
Magnetic Resonance images are endangered of being corrupted by noise and artifacts. Since they are used as a basis for medical diagnosis their quality has to be at highest possible level. Noise can be dealt with by changing the parameters of images acquisition, however it increases the scanning time, which is undesirable in medical imaging. To overcome this obstacle, post-processing methods like filtering are employed for denoising. In domain of MRI denoising many filters may be used, though here emphasis is put on Unbiased Non-Local Means (UNLM) filter, which is an extension of NLM filter. In order to understand Unbiased version of this algorithm, the basic one has to be presented.

Having image \textit{Y}, the NLM algorithm calculates the new value of point \textit{p} accordingly to the equation:

\begin{equation}
\begin{aligned}
NLM(Y(p)) = \sum_{\forall q \in Y}^{}w(p,q)Y(q) \\
0 \le w(p,q) \le 1 , 
\sum_{\forall q \in Y}^{}w(p,q) = 1
\end{aligned}
\label{m5e1}
\end{equation}

It can be seen that value of \textit{p} is calculated as weighted average of pixels in the image (\textit{q}), having fulfilled restrictions from \ref{m5e1}. To determine before mentioned average the similarity between squared neighbourhoods widows centered around pixels \textit{p} and \textit{q} are calculated. The size of the window can determined by the user, defined by parameter $R_{sim}$. Equation \ref{m5e2} shows how to determine this similarity.

\begin{equation}
w(p,q) = \frac{1}{Z(p)}e^{\dfrac{d(p,q)}{h^2}}
\label{m5e2}
\end{equation}

\textit{Z(p)} is the normalizing constant which also uses exponential decay parameter \textit{h} and the weighted Euclidean distance measure for pixels in each neighbourhood, called \textit{d}.

\begin{equation}
Z(p) = \sum_{\forall q}^{} e^{\dfrac{d(p,q)}{h^2}}
\label{m5e5}
\end{equation}

\begin{equation}
d(p,q) = G_{p}||Y(N_{p}) - Y(N_{q})||^{2}_{R_{sim}}
\label{m5e6}
\end{equation}

In above equation $G_p$ stands for a Gaussian weighting function that has a 0 mean and standard deviation usually equal to 1.

Once NLM filter is fully explained, unbiased extension of it can be examined. It builds on the properties of MRI signal. According to \cite{5a1} the magnitude signal of MRI follows a Rician distribution. Furthermore, for low intensity regions the Rician distribution approaches to a Rayleigh one, whilst for high intensity it shifts towards Gaussian. It was investigated that this bias can be handled by filtering the squared MRI image, since it is not longer signal-dependent \cite{5a1}. As a consequence the bias, which equals 2$\sigma^2$ \cite{5a3} can be deleted with ease. The blueprint for UNLM can be summarized in:

\begin{itemize}
	\item noise estimation - which can be done by calculating standard deviation of background in the image, following formula \ref{m5e4}, where $\mu$ is the mean value of background of squared magnitude image. To distinguish background and the body on the MRI scan the Otsu thresholding method \cite{5a4} can be successfully used,  
	\item calculating NLM values for each point of image as in \ref{m5e1},
	\item assessing the unbiased value of each point accordingly to the equation \ref{m5e3}.
	
\end{itemize}

\begin{equation}
UNLM(Y) = \sqrt{NLM(Y)^2 - 2\sigma^2}
\label{m5e3}
\end{equation}

\begin{equation}
\sigma = \sqrt{\frac{\mu}{2}}
\label{m5e4}
\end{equation}

% dwi, if it has to be joint implementation 
UNML implementation for diffusion weighted data becomes a bit less trivial task. Noise estimation is done in different fashion. From distribution of local averages of voxels in the background, the mode has to be calculated and then corrected with a factor of $\sqrt{\frac{2}{\pi}}$ \cite{5a2}. Based on assumption that gradients in similar directions present related behaviours, UNLM for DWI can be formulated as:

\begin{equation}
Y_i(p) = \sqrt{\sum_{j \in \Theta_{i}^{N}}^{}\sum_{q \in N_p}^{} w_i^j(p,q)M_j^2(q)-2\sigma^2}
\end{equation}

Where weights are calculated as for structural data and \textit{M} is a vector containing gray values. Another difference can be found in way the distance between voxels is calculated. It is, the distance in domain of gray levels in the image. 

\begin{equation}
d_{i}^{j}(p,q) = (M_i(N_p)-M_j(N_q))^TG_p(M_i(N_p)-M_j(N_q))
\label{m5e7}
\end{equation}

UNLM in this case not only looks for in the same gradient image \textit{i}, but also searches images \textit{j} near the mainly investigated direction \textit{i}. Working in that manner allows the objective of using joint information (including correlation between channels) to be met \cite{5a2}.



It is worth mentioning that UNLM filter's performance is highly dependent on parameter values. The optimal values of them were examined in \cite{5a1} and same values are adapted in presented implementation. Properly-tuned filter can significantly increase SNR of the scans while preserving body structures.


\emph{\textbf{Module input}}: Previously reconstructed, normalized and corrected data.

\emph{\textbf{Module output}}: Image with deleted Rician noise by unbiased non-local means filter. \\

\section{Module 6}

-detailed description of module, algorithm etc, NO CODES, only theoretical! 

\section{Module 8}

-detailed description of module, algorithm etc, NO CODES, only theoretical! 

\section{Module 9}

Interpolation is a method of constructing new points based on the
existing ones. In other words finding a value of a new point in High
Resolution (HR) image based on points in Low Resolution (LR) pictures.

In classical interpolation techniques pixels in LR data $y$ can be
related to the corresponding $x$ values of HR data: $y_{p}=\frac{1}{N}\sum_{i=1}^{N}x_{i}+n,$
where $y_{p}$ is the pixel of LR image at location $p$, $x_{i}$
is each one of the $N$ High Resolution pixels contained within this
LR pixel and $n$ is some additive noise.

The biggest problem is that to find High Resolution data from the
Low Resolution values. Unfortunately, there is an infinite number
of values that meet that condition. Interpolation methods can be divided
into three basic techniques.

The first group are the most common ones, like linear or spline-based
interpolation. These techniques assume that it is possible to count
the value of a new point by determination some kind of generic function.
The main disadvantage is that they are correct only for images of
homogeneous regions. As is well known, brain consists of grey substance,
white substance and cerebral spinal fluid, so above-mentioned methods
are not appropriate for MRI images interpolation. The second one is
Super Resolution technique. It is commonly used to increase image
resolution on functional MRI (fMRI) and Diffusion Tensor Imaging (DTI).
The method is based on acquisition of multiple Low Resolution (LR)
images of the same object. It is time consuming and not adequate for
clinical applications.

The last but not least method is non-local patch-based technique which
is based on self-similarity of a single image. It is possible to improve
resolution by extracting information from a single image instead of
acquiring several pictures.

The aim of this module is to increase MRI image resolution by upsampling.
The input data is a single 320x240 image. After the processing it
is going to be twice as big. To be more specific 640x480 pixels. The
process can be named as double upsampling.

The first step of the algorithm is to divide every pixel of LR image
into more pixels. Then the patch is designed. It is a rectangle which
will be the area of interest. The pixels that are inside the area
are taken into account during calculation the values of new points.
After that an appropriate estimator is used to correctly calculate
the values of new pixels. Every pixel has to be classified into one
of three groups (grey substance, white substance or cerebral spinal
fluid).

\section{Module 10}

\indent To prepare tree dimension visualization of the cerebral cortex
is used algorithm of marching cubes.\\
 \indent The input data is multiple 2D slices of MR image. The marching
cubes algorithm create a polygonal representation of constant density
surfaces from a 3D array of data. To select the cerebral cortex is
used output data from segmentation made in module 8. The space of
the image is divided into a regular grid of cubes. In each iteration
one cube is considered. At each vertex of cube is determined how the
surface intersects this cube. The density value and compared with
the limit value - surface constant. If the data value is bigger than
suface constant, one is assinged to a cube's vertex. There are 256
combinations of cube orientation relative to the surface, but we can
distinguish 15 basic patterns, that repeat as symmetrical reflections,
produces all possibilities (Fig. \ref{fig:Marching cubes}). If all
values \LyXZeroWidthSpace \LyXZeroWidthSpace are less than the constant
value, then the cube does not form any polygon. Otherwise, the edges
of the polygon are defined (by linear interpolation) at the edges
that intersect the surface. Using central differences, a unit normal
at each cube vertex is calculated and then normal to each trangle
vertex is interpolated. The output of the algorithm is the triangle
vertices and vertex normals.

\begin{figure}[H]
\centering{}\includegraphics[scale=0.7]{figures/MarchingCubes}\caption{Triangulation for the 15 patterns. \label{fig:Marching cubes}}
\end{figure}

\indent To visualization the model, obtained by marching cubes, the
VTK library is used, which enables building the three-dimension model.
\\
 \indent The second part of this module includes visualization of
the brain's cross-section on arbitrarily defined plane. \\
 \indent

There are three primary imaging planes that are performed in medical
imaging: 
\begin{itemize}
\item axial plane, which is any plane that divides the body into superior
and inferior parts, roughly perpendicular to spine. 
\item sagittal plane, which is any imaginary plane parallel to median plane. 
\item coronal plane, which is any vertical plane that divides the body into
anterior and posterior sections. 
\end{itemize}
The MRI produces two-dimensional images that consist of slices of
brain's and is usually performed in axial plane. To receive images
in the remaining planes, linear interpolation is used

\section{Module 11}

-detailed description of module, algorithm etc, NO CODES, only theoretical!

\chapter{Implementation}

\section{Tools}

-python version, libraries

\section{Module 1}

-code

\section{Module 2}

-code 

\section{Module 3}

-code 

\section{Module 4}

-code 

\section{Module 5}

-code 

\section{Module 6}

-code 

\section{Module 8}

-code 

\section{Module 9}

-code 

\section{Module 10}

-code 

\section{Module 11}

-code

\chapter{Tests}

\section{Module 1}

-module 1 tests 

\section{Module 2}

\section{Module 3}

\section{Module 4}

\section{Module 5}

\section{Module 6}

\section{Module 8}

\section{Module 9}

\section{Module 10}

\section{Module 11}

\section{Application}

-whole app tests

\chapter{Authors}

Authors of this project are students of Biomedical Engineering, AGH
UST, Krakow, Poland. \\
 
\begin{center}
\begin{tabular}{|c|c|}
\hline 
Name  & Role \tabularnewline
\hline 
\hline 
Sylwia Mól  & Project Manager\tabularnewline
\hline 
Jacek Fidos  & Software architect\tabularnewline
\hline 
Maciej Gryczan  & GUI engineer\tabularnewline
\hline 
Adrian Stopiak  & Vizualization engineer \tabularnewline
\hline 
Malwina Molendowska  & 1st module developer \tabularnewline
\hline 
Klaudia Gugulska  & 2nd module developer \tabularnewline
\hline 
Kacper Turek  & 3rd module developer \tabularnewline
\hline 
Magdalena Rychlik  & 4th module developer \tabularnewline
\hline 
Alicja Martinek  & 5th module developer \tabularnewline
\hline 
Mateusz Pabian  & 6th module developer \tabularnewline
\hline 
Anna Grzywa  & 8th module developer \tabularnewline
\hline 
Magdalena Kucharska  & 9th module developer \tabularnewline
\hline 
Eliza Kowalczyk  & 9th module developer \tabularnewline
\hline 
Karolina Gajewska  & 10th module developer \tabularnewline
\hline 
Michał Kotarba  & 11th module developer \tabularnewline
\hline 
\end{tabular}
\par\end{center}

\newpage{}

\listoffigures

\begin{comment}
 \bibliographystyle{plain}
\bibliography{bibliografia}
 
\end{comment}

\end{document}
