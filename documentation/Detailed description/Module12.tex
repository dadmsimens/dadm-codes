\section{Module 12. Oblique imaging}

 \indent Oblique Imaging is a technique to create non-perspective
projections from 3D or multiple 2D images. There is not much information about this technique in theoretical sources. \\


\textbf{Algorithm}
\newline\indent Input data is a 3D array, where index is X, Y or Z value and and value of the array in those indices is pixel value of the image. Proposed algorithm involves creating a grid, rotating and translating it, checking which points to interpolate and which can be taken from MRI slices directly and interpolating needed points.
\newline Algorithm step by step:
\begin{itemize}
\item create a grid
\item select 3 angles and rotate this grid relative to the X, Y and Z axes under those angles
\item translate this grid relative to X, Y and Z axes - this grid gives information about x, y and z indices needed for the oblique image 
\item based on created grid check points, that can be taken from MRI slices directly, and which to interpolate
\item create an image base on interpolated points and points from MRI slices
\end{itemize}
\begin{enumerate}
\item \textbf{Creating a grid}
\newline Grid is created based on normal vector $\vec{v1}=[0,0,1]$. Two vectors perpendicular to it are: $\vec{v2}=[1,0,0]$ and $\vec{v3}=[0,1,0]$ are used in grid generation. For given $i = [1,2,\ldots,m] $ and $j = [1,2,\ldots,m] $ every point can be designated from equotation:
\begin{equation}
[x,y,z] = \vec{v2}*i+\vec{v3}*j
\end{equation}
\item \textbf{Rotating grid}
\newline\indent To rotate a grid rotation matrices are used.
\begin{itemize}
\item OX rotation:

\begin{equation}
\left[ \begin{array}{c} X \\ Y \\Z \end{array} \right] = \begin{bmatrix} 1 & 0 & 0 \\ 0 & \cos\Phi & \sin\Phi \\ 0 & -\sin\Phi & \cos\Phi \end{bmatrix}  \left[ \begin{array}{c} X' \\ Y' \\ Z' \end{array} \right] 
\end{equation}

\item OY rotation:

\begin{equation}
\left[ \begin{array}{c} X \\ Y \\Z \end{array} \right] = \begin{bmatrix} \cos\Phi & 0 & -\sin\Phi \\ 0 & 1 & 0 \\ \sin\Phi & 0 & \cos\Phi \end{bmatrix} \left[ \begin{array}{c} X' \\ Y' \\ Z' \end{array} \right]
\end{equation}

\item OZ rotation:
\begin{equation}
\left[ \begin{array}{c} X \\ Y \\Z \end{array} \right] = \begin{bmatrix} \cos\Phi & \sin\Phi & 0 \\ -\sin\Phi & \cos\Phi & 0 \\ 0 & 0 & 1 \end{bmatrix} \left[ \begin{array}{c} X' \\ Y' \\ Z' \end{array} \right] 
\end{equation}
\end{itemize}
\indent In order to save computational time, it is not the created grid itself, that is  being rotated, but two vectors $\vec{v2}$ and $\vec{v3}$

\item \textbf{Translating grid}
\newline\indent To translate every point of grid $i$ units in X direction, $j$ units in Y direction and $k$ units in Z direction given equotation is used:
\begin{equation}
[x,y,z] = x + [1,0,0]*i + y +[0,1,0]*j + [0,0,1]*k
\end{equation}

\item \textbf{Points interpolation}
\indent If after grid rotation and translation any points can not be taken directly from MRI slices it is essential to interpolate them. Maximum number of points from slices taken to interpolate is 8. Their distance to the point to be interpolated is computed using:

\begin{equation}
d = \sqrt{(x_i-x_0)^2+(y_i-y_0)^2+(z_i-z_0)^2}
\end{equation}
\indent Where:
\newline \indent$x_0$, $y_0$, $z_0$ are coordinates of point to be interpolated
\newline \indent$x_i$, $y_i$, $z_i$ are coordinates of the point from surroundings

\indent After that interpolated point value is computed using equotation:

\begin{equation}
p_v = s_p/n_p 
\end{equation}
\indent Where $p_v$ is point value and $s_p$ is sum of values of points in surroundings and $n_p$ is number of points

\item \textbf{Module I/O}
\newline\textbf{- Input -} the input of this module is structural data in form of 3D array $(x,y,slicenumber)$
\newline\textbf{- Output -} the output of this module is one oblique image in form of 2D array

\end{enumerate}
