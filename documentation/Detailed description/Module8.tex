\section{Module 8. Skull stripping}

The aim of this module is to remove pieces of skull from MRI Image
using at pleasure chosen algorithm from the literature. Skull stripping
is performed with use of hybrid approach that combines watershed algorithms
and deformable surface models.

Preliminary processing to isolate the brain form extra-cranial or
non-brain tissues such as e.g. the eye sockets, skin from MRI head
scans is commonly referred as skull stripping. Skull stripping methods
which are available in the literature are broadly classified into
five categories: mathematical morphology-based metods, intensity-based
methods, deformable surface-based methods, atlas-based methods, and
hybrid methods. Each skull stripping method has their own merits and
limitations.

In this module is proposed a hybrid approach to robustly and automatically
segment brain from non-brain tissues in T1-weighted MR image. The
skull stripping consists of series of sequential steps:
\begin{enumerate}
\item {Some relevant parameters are estimated from the input image (the
coordinates of the brain COG, the average brain radius BR, an upper
bound for the intensity of the cerebrospinal fluid CFS – white matter
parameters).} 
\item {Watershed algorithm is performed on the intensity image, with global
minimum initialized within the cerebral white matter.} 
\item {Deformable surface procedure to recover parts of cortex that may
have been erroneously removed in watershed algorithm, using smoothness
constraints on the shape of the skull and atlas information.} 
\end{enumerate}
Watershed algorithms are based on image intensities. Typically, they
attempt to locate the local maxima/minima of the norm of the image
intensity gradient to segment the image into different connected components.
The algorithm proceeds in two steps:
\begin{itemize}
\item {Watershed transform - the sorting all of voxels (of the gray level
inverted image) according to their intensity.} 
\item {Post-watershed correction - assessment the validity of the watershed
segmentation and retrospectively correct it, because the result image
is often inaccurate and nonsmooth, with extracerebral tissues and
CSF frequently remaining.} 
\end{itemize}
Deformable surface algorithm used the watershed segmentation outputs
a segmented volume with most of non-brain tissues removed. A deformable
balloon-like template employes this brain volume. An initial template
deformation is first completed using global parameters regarding the
brain/non-brain border to roughly match the boundary of the brain.
Next, the correctness of resulting surface is verified by an atlas-based
analysis, and if important structures have been removed, it is modified.