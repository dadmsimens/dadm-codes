\section{Module 3. Non-stationary noise estimation}

Magnetic Resonance Imaging (MRI) is known to be affected by several
sources of quality deterioration, due to limitations in the hardware,
scanning times, movement of patients, or even the motion of molecules
in the scanning subject. Among them, noise is one source of degradation
that affects acquisitions. The presence of noise over the acquired
MR signal is a problem that affects not only the visual quality of
the images, but also may interfere with further processing techniques
such as registration or tensor estimation in Diffusion Tensor MRI.

The aim of this module is to estimate the spatial dependent pattern
of the variance of noise in SENSE reconstructed images. For this to
work some additional information must be known beforehand, such as
the sensitivity maps of each receiver coil. In the background of a
SENSE MR image, where the SNR is zero, the Rician PDF (Probability
density function) simplifies to a (non-stationary) Rayleigh distribution,
whose second order moment is defined as:

\begin{equation}
\begin{aligned}E\left\{ M^{2}\left(x\right)\right\} =2\cdot\sigma_{R}^{2}\cdot\left(x\right)\end{aligned}
\end{equation}

Since $\sigma_{R}^{2}$ is \textit{x}-dependent, $E\left\{ M^{2}\left(x\right)\right\} $
will also show a different value for each \textit{x} position.

Let us assume that each coil in the \textit{x} -space is initially
corrupted with uncorrelated Gaussian noise with the same variance
$\sigma_{n}^{2}$ and there is a correlation between coils $\rho$
so that matrix $\sigma$ becomes:

\begin{equation}
\begin{aligned}\Sigma=\sigma_{n}^{2}\begin{pmatrix}1 & \rho & \cdots & \rho\\
\rho & 1 & \cdots & \rho\\
\vdots & \vdots & \ddots & \vdots\\
\rho & \rho & \cdots & 1
\end{pmatrix}=\sigma_{n}^{2}\left(I+\rho\left[1-I\right]\right)\end{aligned}
\end{equation}
with \textit{I} the $L\times L$ identity matrix and \textit{1} a
$L\times L$ matrix of 1's. For each \textit{x} value, we define the
global map

\begin{equation}
\begin{aligned}G_{w_{i}}=W_{i}^{*}\left(I-\rho\left[1-I\right]\right)W_{i},\quad i=1,\cdots,r\end{aligned}
\end{equation}

Global map $G_{w}(x)$ can be easily inferred from $G_{w_{i}}$ values.
Note that $G_{w}(x)$ is strongly related fo the g-factor, so the
first equation becomes

\begin{equation}
\begin{aligned}E\left\{ M^{2}\left(x\right)\right\} =2\cdot\sigma_{n}^{2}G_{w}\left(x\right)\end{aligned}
\end{equation}

and

\begin{equation}
\begin{aligned}\sigma_{n}^{2}=\frac{E\left\{ M^{2}\left(x\right)\right\} }{2G_{w}\left(x\right)}\end{aligned}
\end{equation}

By using this regularization we can assure a single $\sigma_{n}^{2}$
value for all the points in the image. We can now define a noise estimator
based on the local sample estimation of the second order moment:

\begin{equation}
\begin{aligned}\left\langle M^{2}(x)\right\rangle _{x}=\frac{1}{\left|\eta(x)\right|}\sum_{p\in\eta(x)}M^{2}(p)\end{aligned}
\end{equation}
with $\eta(x)$ a neighborhood centered in x. $\left\langle M^{2}(x)\right\rangle _{x}$
is know to follow a Gamma distribution whose mode is $\sigma_{n}^{2}(\left|\eta(x)\right|-1)/\left|\eta(x)\right|$.
Then

\begin{equation}
\begin{aligned}mode\left\{ \frac{\left\langle M_{L}^{2}\right\rangle _{x}}{G_{w}(x)}\right\} =2\sigma_{n}^{2}\frac{\left|\eta(x)\right|-1}{\left|\eta(x)\right|}\approx2\sigma_{n}^{2}\end{aligned}
\end{equation}
when $\left|\eta(x)\right|\gg1$. The estimator is the defined as

\begin{equation}
\begin{aligned}\widehat{\sigma_{n}^{2}}=\frac{1}{2}mode\left\{ \frac{\left\langle M_{L}^{2}(x)\right\rangle _{x}}{G_{w}(x)}\right\} \end{aligned}
\end{equation}
and consequently the noise in each pixel is estimated as

\begin{equation}
\begin{aligned}\widehat{\sigma_{R}^{2}}=\frac{1}{2}mode\left\{ \frac{\left\langle M_{L}^{2}(x)\right\rangle _{x}}{G_{w}(x)}\right\} G_{w}(x)\end{aligned}
\end{equation}

This estimator is only vaild over the background pixels. However,
no segmentation of these pixels is needed: the use of the mode allows
us to work with the whole image. On the other hand, to carry out the
estimation, the sensitivity map of each coil and the correlation between
coils must be known beforhand. These parameters are needed for the
SENSE encoding, and thus, they can be easily obtained.