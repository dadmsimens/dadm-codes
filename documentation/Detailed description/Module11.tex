\section{Module 11. Brain 3D}

\indent To prepare tree dimension visualization of the cerebral cortex
is used algorithm of marching cubes.\\
 \indent The input data is multiple 2D slices of MR image. The marching
cubes algorithm create a polygonal representation of constant density
surfaces from a 3D array of data. To select the cerebral cortex is
used output data from segmentation made in module 8. The space of
the image is divided into a regular grid of cubes. In each iteration
one cube is considered. At each vertex of cube is determined how the
surface intersects this cube. The density value and compared with
the limit value - surface constant. If the data value is bigger than
suface constant, one is assinged to a cube's vertex. There are 256
combinations of cube orientation relative to the surface, but we can
distinguish 15 basic patterns, that repeat as symmetrical reflections,
produces all possibilities (Fig. \ref{fig:Marching cubes}). If all
values \LyXZeroWidthSpace \LyXZeroWidthSpace are less than the constant
value, then the cube does not form any polygon. Otherwise, the edges
of the polygon are defined (by linear interpolation) at the edges
that intersect the surface. Using central differences, a unit normal
at each cube vertex is calculated and then normal to each trangle
vertex is interpolated. The output of the algorithm is the triangle
vertices and vertex normals.

\begin{figure}[H]
\centering{}\includegraphics[scale=0.7,bb = 0 0 200 100, draft, type=eps]{MarchingCubes}\caption{Triangulation for the 15 patterns. \label{fig:Marching cubes}}
\end{figure}

\indent To visualization the model, obtained by marching cubes, the
VTK library is used, which enables building the three-dimension model.
\\
 \indent The second part of this module includes visualization of
the brain's cross-section on arbitrarily defined plane. \\
 \indent

There are three primary imaging planes that are performed in medical
imaging: 
\begin{itemize}
\item axial plane, which is any plane that divides the body into superior
and inferior parts, roughly perpendicular to spine. 
\item sagittal plane, which is any imaginary plane parallel to median plane. 
\item coronal plane, which is any vertical plane that divides the body into
anterior and posterior sections. 
\end{itemize}
The MRI produces two-dimensional images that consist of slices of
brain's and is usually performed in axial plane. To receive images
in the remaining planes, linear interpolation is used