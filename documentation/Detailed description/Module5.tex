\section{Module 5. . Non-stationary noise filtering 2}

Magnetic Resonance images are endangered of being corrupted by noise
and artifacts. Since they are used as a basis for medical diagnosis
their quality has to be at highest possible level. Noise can be dealt
with by changing the parameters of images acquisition, however it
increases the scanning time, which is undesirable in medical imaging.
To overcome this obstacle, post-processing methods like filtering
are employed for denoising. In domain of MRI denoising many filters
may be used, though here emphasis is put on Unbiased Non-Local Means
(UNLM) filter, which is an extension of NLM filter. In order to understand
Unbiased version of this algorithm, the basic one has to be presented.

Having image \textit{Y}, the NLM algorithm calculates the new value
of point \textit{p} accordingly to the equation:

\begin{equation}
\begin{aligned}NLM(Y(p))=\sum_{\forall q\in Y}^ {}w(p,q)Y(q)\\
0\le w(p,q)\le1,\sum_{\forall q\in Y}^ {}w(p,q)=1
\end{aligned}
\label{m5e1}
\end{equation}

It can be seen that value of \textit{p} is calculated as weighted
average of pixels in the image (\textit{q}), having fulfilled restrictions
from \ref{m5e1}. To determine before mentioned average the similarity
between squared neighbourhoods widows centered around pixels \textit{p}
and \textit{q} are calculated. The size of the window can determined
by the user, defined by parameter $R_{sim}$. Equation \ref{m5e2}
shows how to determine this similarity.

\begin{equation}
w(p,q)=\frac{1}{Z(p)}e^{\dfrac{d(p,q)}{h^{2}}}\label{m5e2}
\end{equation}

\textit{Z(p)} is the normalizing constant which also uses exponential
decay parameter \textit{h} and the weighted Euclidean distance measure
for pixels in each neighbourhood, called \textit{d}.

\begin{equation}
Z(p)=\sum_{\forall q}^ {}e^{\dfrac{d(p,q)}{h^{2}}}\label{m5e5}
\end{equation}

\begin{equation}
d(p,q)=G_{p}||Y(N_{p})-Y(N_{q})||_{R_{sim}}^{2}\label{m5e6}
\end{equation}

In above equation $G_{p}$ stands for a Gaussian weighting function
that has a 0 mean and standard deviation usually equal to 1.

Once NLM filter is fully explained, unbiased extension of it can be
examined. It builds on the properties of MRI signal. According to
\cite{5a1} the magnitude signal of MRI follows a Rician distribution.
Furthermore, for low intensity regions the Rician distribution approaches
to a Rayleigh one, whilst for high intensity it shifts towards Gaussian.
It was investigated that this bias can be handled by filtering the
squared MRI image, since it is not longer signal-dependent \cite{5a1}.
As a consequence the bias, which equals 2$\sigma^{2}$ \cite{5a3}
can be deleted with ease. The blueprint for UNLM can be summarized
in:
\begin{itemize}
\item noise estimation - which can be done by calculating standard deviation
of background in the image, following formula \ref{m5e4}, where $\mu$
is the mean value of background of squared magnitude image. To distinguish
background and the body on the MRI scan the Otsu thresholding method
\cite{5a4} can be successfully used, 
\item calculating NLM values for each point of image as in \ref{m5e1}, 
\item assessing the unbiased value of each point accordingly to the equation
\ref{m5e3}.
\end{itemize}
\begin{equation}
UNLM(Y)=\sqrt{NLM(Y)^{2}-2\sigma^{2}}\label{m5e3}
\end{equation}

\begin{equation}
\sigma=\sqrt{\frac{\mu}{2}}\label{m5e4}
\end{equation}

% dwi, if it has to be joint implementation 
UNML implementation for diffusion weighted data becomes a bit less
trivial task. Noise estimation is done in different fashion. From
distribution of local averages of voxels in the background, the mode
has to be calculated and then corrected with a factor of $\sqrt{\frac{2}{\pi}}$
\cite{5a2}. Based on assumption that gradients in similar directions
present related behaviours, UNLM for DWI can be formulated as:

\begin{equation}
Y_{i}(p)=\sqrt{\sum_{j\in\Theta_{i}^{N}}^ {}\sum_{q\in N_{p}}^ {}w_{i}^{j}(p,q)M_{j}^{2}(q)-2\sigma^{2}}
\end{equation}

Where weights are calculated as for structural data and \textit{M}
is a vector containing gray values. Another difference can be found
in way the distance between voxels is calculated. It is, the distance
in domain of gray levels in the image.

\begin{equation}
d_{i}^{j}(p,q)=(M_{i}(N_{p})-M_{j}(N_{q}))^{T}G_{p}(M_{i}(N_{p})-M_{j}(N_{q}))\label{m5e7}
\end{equation}

UNLM in this case not only looks for in the same gradient image \textit{i},
but also searches images \textit{j} near the mainly investigated direction
\textit{i}. Working in that manner allows the objective of using joint
information (including correlation between channels) to be met \cite{5a2}.

It is worth mentioning that UNLM filter's performance is highly dependent
on parameter values. The optimal values of them were examined in \cite{5a1}
and same values are adapted in presented implementation. Properly-tuned
filter can significantly increase SNR of the scans while preserving
body structures.

\textbf{\emph{Module input}}: Previously reconstructed, normalized
and corrected data.

\textbf{\emph{Module output}}: Image with deleted Rician noise by
unbiased non-local means filter. \\