\section{Module 4. Non-stationary noise filtering 1}

The aim of this module is to remove Rician noise from MR images. If
both real and imaginary parts of signal are corrupted with zero-mean
uncorrelated Gaussian noise with equal variance, the envelope of magnitude
signal will follow a Rician distribution. Many processes allow to
remove noise, here the method is used to denoise MR images is the
linear minimum square error estimator (LMMSE). The main purpose of
LMMSE is to find a closed-form estimator for a signal that follows
a Rician distributions. It is more efficient than optimization-based
solutions. The estimator uses information of the sample distribution
of local statistics of the image such as the local mean, the local
variance and the local mean square value. In this method, the true
value for each noisy pixel is estimated by a set of pixels selected
from a local neighborhood.

The LMMSE estimator for a 2-D signal with Rician distribution is defined:

\begin{equation}
\begin{aligned}\widehat{A_{ij}^{2}}=E\{A_{ij}^{2}\}+C_{A_{ij}^{2}M_{ij}^{2}}C_{M_{ij}^{2}M_{ij}^{2}}^{-1}(M_{ij}^{2}-E\{M_{ij}^{2}\})\end{aligned}
\label{m4eq1}
\end{equation}
where $A_{ij}$ is the unknown intensity value in pixel $ij$, $M_{i}j$
the observation vector, $C_{A_{ij}^{2}M_{ij}^{2}}$ the cross-covarience
vector and $C_{M_{ij}^{2}M_{ij}^{2}}$ the covarience matrix. If the
estimator is simplified to be pointwise, vectors and matrics become
scalar values. Then the assuming local ergodicity with a square nighbourhood
around the pixel $ij$, the finally equation for LMMSE is defined:
\begin{equation}
\begin{aligned}\widehat{A_{ij}^{2}}=\langle M_{i}j^{2}\rangle-2\sigma_{n}^{2}+K_{ij}(M_{ij}^{2}-\langle M_{ij}^{2}\rangle)\end{aligned}
\label{m4eq2}
\end{equation}
with $K_{ij}$ 
\begin{equation}
\begin{aligned}K_{ij}^{2}=1-\frac{4\sigma_{n}^{2}(\langle M_{i}j^{2}\rangle-2\sigma_{n}^{2})}{\langle M_{i}j^{4}\rangle-{\langle M_{i}j^{2}\rangle}^{2}}\end{aligned}
\label{m4eq3}
\end{equation}

The LMMSE estimator is related to the quality of the estimate of the
noise variance $\sigma_{n}^{2}$. For the noise estimation the mode
of the sample mean is used:

\begin{equation}
\begin{aligned}\widehat{\sigma_{n}}=\sqrt{\frac{2}{\pi}}mode(\widehat{\mu_{1}}_{ij})\end{aligned}
\label{m4eq4}
\end{equation}
where $\widehat{\mu_{1}}_{ij}$ 
\begin{equation}
\begin{aligned}\widehat{\mu_{1}}_{ij}=\frac{1}{|\eta_{ij}}\sum_{p\colon\eta_{ij}}I_{p}\end{aligned}
\label{m4eq5}
\end{equation}

The use of the LMMSE method should makes the filtering process computationally
far more efficient and easier to implement. Also the use of local
statistics should decrease estimator dependent of parameters such
as the size of window.

\textbf{\emph{Module input}}: Reconstructed, normalized and corrected
data.

\textbf{\emph{Module output}}: Image without Rician noise.