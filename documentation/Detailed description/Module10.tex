\section{Module 10. Upsampling}

Interpolation is a method of constructing new points based on the
existing ones. In other words finding a value of a new point in High
Resolution (HR) image based on points in Low Resolution (LR) pictures.

In classical interpolation techniques pixels in LR data $y$ can be
related to the corresponding $x$ values of HR data: $y_{p}=\frac{1}{N}\sum_{i=1}^{N}x_{i}+n,$
where $y_{p}$ is the pixel of LR image at location $p$, $x_{i}$
is each one of the $N$ High Resolution pixels contained within this
LR pixel and $n$ is some additive noise.

The biggest problem is that to find High Resolution data from the
Low Resolution values. Unfortunately, there is an infinite number
of values that meet that condition. Interpolation methods can be divided
into three basic techniques.

The first group are the most common ones, like linear or spline-based
interpolation. These techniques assume that it is possible to count
the value of a new point by determination some kind of generic function.
The main disadvantage is that they are correct only for images of
homogeneous regions. As is well known, brain consists of grey substance,
white substance and cerebral spinal fluid, so above-mentioned methods
are not appropriate for MRI images interpolation. The second one is
Super Resolution technique. It is commonly used to increase image
resolution on functional MRI (fMRI) and Diffusion Tensor Imaging (DTI).
The method is based on acquisition of multiple Low Resolution (LR)
images of the same object. It is time consuming and not adequate for
clinical applications.

The last but not least method is non-local patch-based technique which
is based on self-similarity of a single image. It is possible to improve
resolution by extracting information from a single image instead of
acquiring several pictures.

The aim of this module is to increase MRI image resolution by upsampling.
The input data is a single 320x240 image. After the processing it
is going to be twice as big. To be more specific 640x480 pixels. The
process can be named as double upsampling.

The first step of the algorithm is to divide every pixel of LR image
into more pixels. Then the patch is designed. It is a rectangle which
will be the area of interest. The pixels that are inside the area
are taken into account during calculation the values of new points.
After that an appropriate estimator is used to correctly calculate
the values of new pixels. Every pixel has to be classified into one
of three groups (grey substance, white substance or cerebral spinal
fluid).

\hfill{}\\
\textbf{List of References}\\
\cite{9art1}