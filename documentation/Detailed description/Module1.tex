\section{Module 1. MRI reconstruction}

The aim of this module is to formulate mathematical algorithm, which
enables proper data reconstruction for images obtained with parallel
MRI scans. The reconstruction is performed with use of Sensitivity
Encoding (SENSE) algorithm in least squares (LS) solution context
and Tikhonov regularization method.

Generally, parallel MRI acquisitions are targeted to diminish time
needed for data sampling. The usage of multiple coils enabled simultaneous
acquisition of signals. A further step, which is acquiring partial
data from \textbf{k}-space, leads to craved time savings, meanwhile
maintaining full spatial resolution as well as contrast at the same
time. However, the approach of omitting lines in acquisition step
results in data aliasing, i.e. folded images that need further data
processing.

To clearly mark out how data is processed in this module, we list
following reconstruction steps: i) the application of 2D Fourier Transform
transform (2D FFT) to \textbf{k}-space data (acquired raw signals)
from multiple coils. The result is a set of \textbf{x}-space images
with folded pixels, ii) the sensitivity maps estimation of coils profiles
(the information is needed to properly unfold subsampled data) and
iii) the proper unfolding data process with usage of SENSE reconstruction
algorithm and its alterations.

The most crucial step in processing is estimation of sensitivity coil
profiles as a successful image reconstruction with use of pMRI algorithms
highly depends on accurate sensitivity coil assessment. As sensitivity
information varies from scan to scan it is impossible to obtain absolute
maps. To obtain reliable knowledge, reference scans have to be conducted
each time an examination is performed. These low-resolution information
helps to estimate coil profiles with use of the many methods i.e.
dividing each component coil image by a 'sum of squares' image.

It basic formulation, SENSE algorithm is applied to Cartesian MRI
data undersampled uniformly by a factor $r$~(i.e. $r=2$ means that
every other line in \textbf{k}-space is skipped). After Fourier transformation,
each pixel in \textbf{x}-space image received in \textit{l}-th coil
can be seen as weighted sum of $r$ pixels from full FOV, each multiplied
by corresponding localized values of maps. The distance between those
'aliased' points in the full FOV is always equal to the desired FOVy
value divided by subsampling rate. Obviously, depending on subsampling
rate the number of folded pixels changes. Basically, the signal in
one pixel at a certain location $(x,y)$ received from $l$-th component
coil image $D_{l}^{S}$ with chosen subsampling rate $r$ can be written
as: 
\begin{equation}
D_{l}^{S}(x,y)=S_{l}(x,y_{1})D^{R}(x,y_{1})+S_{l}(x,y_{2})D^{R}(x,y_{2})+...+S_{l}(x,y_{r})D^{R}(x,y_{r}),\label{Eq:wzor1}
\end{equation}
where index $l$ counts from 1 to $L$ (number of coils) and index
$i$ counts from 1 to $r$. Eq.(\ref{Eq:wzor1}) can be rewritten
as:

\begin{equation}
D_{l}^{S}(x,y)=\sum_{i=1}^{r}S_{l}(x,y_{i})D^{R}(x,y_{i})\quad\text{for}\quad l=1,...,L.\label{Eq:wzor2}
\end{equation}

Including all $L$ coils the above equation can be rewritten in a
matrix form:

\begin{equation}
\textbf{D}^{S}(\textbf{x})=\textbf{S}(\textbf{x})\textbf{D}^{R}(\textbf{x}),\label{Eq:wzor3}
\end{equation}

The vector $\textbf{D}^{S}(\textbf{x})$ denotes the aliased coil
image values at a specific location \textbf{x} = $(x,y_{i})$ and
has a length of $L$, $\textbf{S}(\textbf{x})$ is a $L$x$R$ matrix
and represents the sensitivities values for each coil at the $r$~superimposed
positions and $\textbf{D}^{R}(\textbf{x})$ lists the $r$ pixels
from full FOV image to be reconstructed. The closed-form solution
of the problem is as follows: 
\begin{equation}
\widehat{\textbf{D}^{R}(\textbf{x})}=(\textbf{S}^{H}(\textbf{x})\textbf{S}(\textbf{x}))^{-1}\textbf{S}^{H}(\textbf{x})\textbf{D}^{S}(\textbf{x}),\label{Eq:wzor4}
\end{equation}
where $\widehat{\textbf{D}^{R}(\textbf{x})}=[\widehat{D^{R}(x,y_{1})},...,\widehat{D^{R}(x,y_{r})}]^{T}$
and $\textbf{S}^{H}(\textbf{x})$ is the conjugate transpose of the
$\textbf{S}(\textbf{x})$ matrix. The final reconstruction image is
defined as: 
\begin{equation}
M(\textbf{x})=\left|\widehat{\textbf{D}^{R}(\textbf{x})}\right|.\label{Eq:wzor5}
\end{equation}

The `unfolding' process can be performed as long as inversion of $\textbf{S}(\textbf{x})$
matrix is possible. Therefore, we cannot set the value of subsampling
rate exceeding the number of coils $L$. To restore full FOV data,
SENSE algorithm has to be recalled for each pixel in aliased \textbf{x}–space
image.

A regularization approach is defined as an~inversion method that
introduces additional information in order to stabilize the solution.
This method is beneficial as it roughly matches the desired solution
and is less sensitive to perturbations of the data. Tikhonov regularization
is a common approach to obtain an inexact solution to a~linear system
of equations. In particular, the Tikhonov regularized estimate reads
as follows:

\begin{equation}
\widehat{\textbf{D}_{reg}^{R}}=\text{arg}\underset{\textbf{D}^{R}}{\text{min}}\left\{ \left\Vert \textbf{D}^{S}-\textbf{S}\textbf{D}^{R}\right\Vert ^{2}+\lambda^{2}\left\Vert \textbf{A}(\textbf{D}^{R}-\textbf{D})\right\Vert ^{2}\right\} ,\label{Eq:wzor6}
\end{equation}

where $\lambda$ is a regularization parameter ($\lambda>0$) and
$\textbf{D}$ is a~prior image known as a regularization image. Selection
of the parameter $\lambda$ and $\textbf{D}$ can be performed using
different procedures. In this module $\textbf{A}$~is assumed to
be an identity matrix. The first term provides fidelity to the data
and the second introduces prior knowledge (e.x. median filtered initial
guess of LS SENSE) about the expected behaviour of $\textbf{D}^{R}$.
The Tikhonov regularization problem is given by:

\begin{equation}
\widehat{\textbf{D}_{reg}^{R}}=\textbf{D}+(\textbf{S}^{H}\textbf{S}+\lambda\textbf{A}^{H}\textbf{A})^{-1}\textbf{S}^{H}(\textbf{D}^{S}-\textbf{S}\textbf{D}).\label{Eq:wzor7}
\end{equation}

A reasonable value for $\lambda$~can be picked using many technique,
i.e. the L-curve criterion or generalized cross-validation.

\textbf{\emph{Module input}}: Synthetic MR images are brain MRI slices
coming from BrainWeb are normalized to {[}0-255{]} (all with intensity
non-uniformity INU=0). Only T1- and T2-weighted data is used. The
dataset is free of noise and the background areas are set to zero.
The slice thickness equals 1 mm. These images are used then to simulate
synthetic noisy accelerated parallel Cartesian SENSE MRI data according
to following steps (the data simulation is performed with use of eight
receiver coils ($L=8$)): i) simulated sensitivity maps (divided into
the ratio 3:1 for real and imaginary parts, respectively) are added
to fully-sampled \textbf{x}-space data, ii) correlated complex Gaussian
noise with different values of standard deviations is added to each
coil image, iii) 2D FFT and data subsampling with chosen reduction
factor $r$ is performed and iv) 2D iFFT is applied to recover data
in \textbf{x}-space. Then, data reconstruction process is conducted.

\textbf{\emph{Module output}}: The output is full resolution reconstructed
data performed with two different algorithms: SENSE (LSE) and Tikhonov
regularization. \\