\section{Module 2. Intensity inhomogenity correction}

The Intensity inhomogeneity of the same tissue varies with the location
of the tissue within the image. In other words it refers to the slow,
nonanatomic intensity variations of the same tissue over the image
domain. It can be due to imaging instrumentation (such as radio-frequency
nonuniformity, static field inhomogeneity, etc.) or the patient movement.
This artifact is particularly severe in MR images captured by surface
coils. Although intensity inhomogeneity is usually hardly noticeable
to a human observer, many medical image analysis methods, such as
segmentation and registration, are highly sensitive to the spurious
variations of image intensities.

The aim of this module is correction of intensity inhomogeneity in
MR image using gradient based, surface fitting method. The methods
fit a parametric surface to a set of image features that contain information
on intensity inhomogeneity. The resulting surface, which is usually
polynomial or spline based, represents the multiplicative inhomogeneity
field that is used to correct the input image.

\hfill{}\\
\textbf{List of References}\\
\cite{2a1}, \cite{2a2}