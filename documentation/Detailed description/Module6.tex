\section{Module 6. Diffusion tensor imaging}

The aim of this module is to estimate brain tissue diffusion tensor
from Diffusion Weighted Images (DWI). DWI are obtained using a different
pulse sequence than anatomical MRI images and as such differ in their
information content. Concretely, Diffusion Tensor Imaging (DTI) gives
an insight into tissue microstructure based on DWI, which are obtained by probing tissues using gradient-pulse excited water molecules \cite{m6_soares2013}. This enables indirect measurements of structural orientation and the degree of anisotropy as water molecules diffuse differs between tissues.

In DTI-MRI the measured signal is defined as \cite{m6_koay2006b}: 
\begin{equation}
S\left(b,\boldsymbol{g}\right) =  S_{0} exp\left(-b\boldsymbol{g^TDg}\right)
\label{Eq:m6_eq_1}
\end{equation}

where $S\left(b,\boldsymbol{g}\right)$ is the measured signal, $S_{0}$
is the reference signal without diffusion gradient attenuation, $b$
is the diffusion weight scalar, $\boldsymbol{g}$ is the diffusion
encoding gradient vector of unit length and $\boldsymbol{D}$ is the
diffusion tensor.

The diffusion tensor $\boldsymbol{D}$ is symmetric and describes
molecular mobility along each direction: 
\begin{equation}
\boldsymbol{D}=
\begin{bmatrix}
D_{xx} & D_{xy} & D_{xz} \\
D_{yx} & D_{yy} & D_{yz} \\
D_{zx} & D_{zy} & D_{zz} 
\end{bmatrix}
\label{Eq:m6_eq_2}
\end{equation}

Furthermore, the Eq. (\ref{Eq:m6_eq_1}) can be rewritten as: 
\begin{equation}
ln{S\left(b,\boldsymbol{g}\right)} = ln{S_0}-b\boldsymbol{g^TDg}
\label{Eq:dti_eq_3}
\end{equation}
which established a linear relationship between model parameters and measured quantity.

Estimating $\boldsymbol{D}$ from the above equation can be done using
Weighted Least Squares (WLS) or Nonlinear Least Squares (NLS) algorithms.
In order to correctly estimate the diffusion tensor it is necessary
to acquire at least seven DWI - one for each direction of diffusion
and one in the absence of diffusion gradient.

Let $\boldsymbol{\gamma}$ be the parameters vector of DTI model:
\begin{equation}
\boldsymbol{\gamma}={\lbrack ln{S_0}, D_{xx}, D_{yy}, D_{zz}, D_{xy}, D_{yx}, D_{xz}\rbrack}^T
\label{Eq:m6_eq_4}
\end{equation}

Furthermore let $\boldsymbol{W_i}$ be the design matrix of our model. In this case it represents the diffusion vector in a given direction for $i$-th measurement:
\begin{equation}
\boldsymbol{W_i}=
\begin{bmatrix}
1, -b_i{g_{ix}}^2, -b_i{g_{iy}}^2, -b_i{g_{iz}}^2, -2b_ig_{ix}g_{iy}, -2b_ig_{iy}g_{iz}, -2b_ig_{ix}g_{iz}
\end{bmatrix}
\label{Eq:m6_eq_5}
\end{equation} 

Then equations \ref{Eq:m6_eq_1} oraz \ref{Eq:dti_eq_3} can be rewritten as:
\begin{equation}
S\left(\boldsymbol{\gamma}\right) =  exp\left(\boldsymbol{W_i\gamma}\right)
\label{Eq:m6_eq_6}
\end{equation}
\begin{equation}
ln\left({S\left(\boldsymbol{\gamma}\right)}\right) =  \boldsymbol{W_i\gamma}
\label{Eq:m6_eq_7}
\end{equation}

The aforementioned expressions can be used to estimate the vector $\boldsymbol{\gamma}$ using Least Squares Regression models. It is important to note that tensor built using the elements of $\boldsymbol{\gamma}$ will be symmetric.

\hfill
\subsection{Weighted Least Squares estimation}

For Weighted Least Squares (WLS) model the cost function is defined as \cite{m6_koay2006a}:
\begin{equation}
f_{WLS}\left(\boldsymbol{\gamma}\right)=\frac{1}{2}\sum_{i=1}^{m}{{\omega_i}^2\left(y_i-\sum_{j=1}^{7}{W_{ij}\gamma_{j}}\right)^2}; \quad y_i=ln(S(\gamma)_i)
\label{Eq:m6_eq_8}
\end{equation}
where $m$ is the total number of measurements while $\omega$ is the measurement weight associated with WLS method.

The aim of Least Squares Regression methods is to minimaze the cost function $f_{WLS}$ with respect to $\boldsymbol{\gamma}$, which coincides with the function gradient $\left|\nabla{f_{WLS}}\right|$ equal to zero.

In general, vector-valued function gradient is defined as a vector whose elements can be calculated using the following formula:
\begin{equation}
\left|\nabla{f_{WLS}}\right|_k = \frac{\partial{f_{WLS}}}{\partial{\gamma_k}}
\label{Eq:m6_eq_9}
\end{equation}

For WLS cost function:
\begin{equation}
\begin{aligned}
\frac{\partial{f_{WLS}}}{\partial{\gamma_k}}&=\sum_{i=1}^{m}\frac{\partial}{\partial{\gamma_k}}\,{\omega_i}^2\left(y_i-\sum_{j=1}^{7}W_{ij}\gamma_j\right)\\
& = \sum_{i=1}^{m}\,{\omega_i}^2\left(y_i-\sum_{j=1}^{7}W_{ij}\gamma_j\right)\frac{\partial}{\partial{\gamma_k}}\left(y_i-\sum_{j=1}^{7}W_{ij}\gamma_j\right)\\
& = \sum_{i=1}^{m}\,{\omega_i}^2\left(y_i-\sum_{j=1}^{7}W_{ij}\gamma_j\right)\sum_{j=1}^{7}\left(-W_{ij}\right)\frac{\partial}{\partial{\gamma_k}}\gamma_j\\
& = \sum_{i=1}^{m}\,{\omega_i}^2\left(y_i-\sum_{j=1}^{7}W_{ij}\gamma_j\right)\left(-W_{ik}\right)
\end{aligned}
\label{Eq:m6_eq_10}
\end{equation}

or in matrix form:
\begin{equation}
\nabla{f_{WLS}}=-\boldsymbol{W}^T\boldsymbol{\omega}^T\boldsymbol{\omega}\left(\boldsymbol{y}-\boldsymbol{W\gamma}\right)
\label{Eq:m6_eq_11}
\end{equation}
where $\boldsymbol{\omega}$ is a diagonal matrix with non-zero elements equal to WLS model weights.

The Hessian of vector-valued function is defined as a matrix whose elements computed as:
\begin{equation}
\left|\nabla^2{f_{WLS}}\right|_{lk} = \frac{\partial^2{f_{WLS}}}{\partial{\gamma_l}\partial{\gamma_k}}
\label{Eq:m6_eq_12}
\end{equation}

For WLS cost function:
\begin{equation}
\begin{aligned}
\frac{\partial^2{f_{WLS}}}{\partial{\gamma_l}\partial{\gamma_k}} &= \frac{\partial}{\partial{\gamma_l}}\frac{\partial{f_{WLS}}}{\partial{\gamma_k}}\\
&=\frac{\partial}{\partial{\gamma_l}}\sum_{i=1}^{m}\,{\omega_i}^2\left(y_i-\sum_{j=1}^{7}W_{ij}\gamma_j\right)\left(-W_{ik}\right)\\
&=\sum_{i=1}^{m}\,{\omega_i}^2\frac{\partial}{\partial{\gamma_l}}\left(y_i-\sum_{j=1}^{7}W_{ij}\gamma_j\right)\left(-W_{ik}\right)\\
&=\sum_{i=1}^{m}\,{\omega_i}^2\left(-W_{il}\right)\left(-W_{ik}\right)\\
&=\sum_{i=1}^{m}\,{\omega_i}^2 W_{il}W_{ik}\\
\end{aligned}
\label{Eq:m6_eq_13}
\end{equation}

or in matrix form:
\begin{equation}
\nabla^2{f_{WLS}}=\boldsymbol{W}^T\boldsymbol{\omega}^T\boldsymbol{\omega}\boldsymbol{W}
\label{Eq:m6_eq_14}
\end{equation}

One can notice that Eq. (\ref{Eq:m6_eq_11}) does not depend on model parameters, meaning that it can be solved using it's normal equation:
\begin{equation}
\begin{aligned}
\nabla{f_{WLS}}=0 \iff -\boldsymbol{W}^T\boldsymbol{\omega}^T\boldsymbol{\omega}\left(\boldsymbol{y}-\boldsymbol{W\gamma}\right)=0\\
\boldsymbol{W}^T\boldsymbol{\omega}^T\boldsymbol{\omega}\boldsymbol{W\gamma}=\boldsymbol{W}^T\boldsymbol{\omega}^T\boldsymbol{\omega}\boldsymbol{y}\\
\boldsymbol{\gamma}=\left(\boldsymbol{W}^T\boldsymbol{\omega}^T\boldsymbol{\omega}\boldsymbol{W}\right)^{-1}\boldsymbol{W}^T\boldsymbol{\omega}^T\boldsymbol{\omega y}
\end{aligned}
\label{Eq:m6_eq_15}
\end{equation}

Before continuing, it is important to note that there exists no simple choice of $\boldsymbol{\omega}$ weights. The authors of \cite{m6_koay2006a} suggest using the measured signal $\boldsymbol{S}$, in \cite{m6_salvador2005} one can read that weights should be the Linear Least Squares estimate of signal in each voxel, while the authors \cite{m6_basser1994} argue for the use of signal variance estimation using Rice's noise estimate. As one can see, WLS model complexity varies depending on the choice of weights vector computation method.

\hfill
\subsection{Nonlinear Least Squares estimation}

Nonlinear Least Squares cost function is defined as:
\begin{equation}
f_{NLS}\left(\boldsymbol{\gamma}\right)=\frac{1}{2}\sum_{i=1}^{m}\left(S_i-exp\left(\sum_{j=1}^{7}W_{ij}\gamma_j\right)\right)^2=\frac{1}{2}\sum_{i=1}^{m}r_i^2
\label{Eq:m6_eq_16}
\end{equation}

Same as before, we will derive NLS cost function gradient and Hessian matrix. 
\begin{equation}
\begin{aligned}
\frac{\partial{f_{NLS}}}{\partial{\gamma_k}}&=2\frac{1}{2}\sum_{i=1}^{m}r_i\frac{\partial}{\partial{\gamma_k}}\left(r_i\left(\gamma\right)\right)\\
&=\sum_{i=1}^{m}(-r_i)\frac{\partial}{\partial{\gamma_k}}\left(exp\left[\sum_{j=1}^{7}W_{ij}\gamma_j\right]\right)\\
&=\sum_{i=1}^{m}(-r_i)exp\left(\sum_{j=1}^{7}W_{ij}\gamma_j\right)\frac{\partial}{\partial{\gamma_k}}\left(\sum_{j=1}^{7}W_{ij}\gamma_j\right)\\
&=\sum_{i=1}^{m}(-r_i)exp\left(\sum_{j=1}^{7}W_{ij}\gamma_j\right)\left(\sum_{j=1}^{7}W_{ij}\frac{\partial}{\partial{\gamma_k}}\gamma_j\right)\\
&=\sum_{i=1}^{m}(-r_i)exp\left(\sum_{j=1}^{7}W_{ij}\gamma_j\right)W_{ik}\\
&=\sum_{i=1}^{m}(-r_i)\hat{S_i}W_{ik}
\end{aligned}
\label{Eq:m6_eq_17}
\end{equation}

where $\hat{S_i}$ is the voxel signal estimate of NLS model. Let $\boldsymbol{\hat{S}}$ be a diagonal matrix with non-zero elements set to $\hat{S_i}$. Then $f_{NLS}$ computation can be rewritten in matrix form:
\begin{equation}
\nabla{f_{NLS}}=-\boldsymbol{W}^T\boldsymbol{\hat{S}}\boldsymbol{r}
\label{Eq:m6_eq_18}
\end{equation}
as one can see from the Eq. (\ref{Eq:m6_eq_18}), the value of gradient depends on NLS signal estimate, meaning that contrary to WLS method, it is not possible to compute NLS estimate without resorting to iterative methods.

NLS Hessian matrix elements can be computed as:
\begin{equation}
\begin{aligned}
\frac{\partial^2{f_{NLS}}}{\partial{\gamma_l}\partial{\gamma_k}} &= \frac{\partial}{\partial{\gamma_l}}\frac{\partial{f_{NLS}}}{\partial{\gamma_k}}\\
&=\frac{\partial}{\partial{\gamma_l}}\sum_{i=1}^{m}(-r_i)\hat{S_i}W_{ik}\\
&=-\sum_{i=1}^{m}\left(\hat{S_i}W_{ik}\frac{\partial}{\partial{\gamma_l}}r_i+r_iW_{ik}\frac{\partial}{\partial{\gamma_l}}\hat{S_i}\right)\\
&=-\sum_{i=1}^{m}\left(\hat{S_i}W_{ik}\frac{\partial}{\partial{\gamma_l}}\left[S_i-\hat{S_i}\right]+r_iW_{ik}\frac{\partial}{\partial{\gamma_l}}\hat{S_i}\right)\\
&=-\sum_{i=1}^{m}\left(\hat{S_i}W_{ik}\frac{\partial}{\partial{\gamma_l}}\left[-\hat{S_i}\right]+r_iW_{ik}\frac{\partial}{\partial{\gamma_l}}\hat{S_i}\right)\\
&=-\sum_{i=1}^{m}\left(\hat{S_i}W_{ik}\left[-\hat{S_i}W_{il}\right]+r_iW_{ik}\hat{S_i}W_{il}\right)\\
&=\sum_{i=1}^{m}\left(\hat{S_i}W_{ik}\hat{S_i}W_{il}-r_iW_{ik}\hat{S_i}W_{il}\right)\\
&=\sum_{i=1}^{m}W_{ki}\left(\hat{S_i}^2-r_i\hat{S_i}\right)W_{il}\\
\end{aligned}
\label{Eq:m6_eq_19}
\end{equation}

or in matrix form:
\begin{equation}
\nabla^2{f_{NLS}}=\boldsymbol{W}^T\left(\boldsymbol{\hat{S}^T\hat{S}-\boldsymbol{R\hat{S}}}\right)\boldsymbol{W}
\label{Eq:m6_eq_20}
\end{equation}
where $\boldsymbol{R}$ is the diagonal matrix whose non-zero elements are equal to NLS model residuals ($r_i$).\linebreak

Usually, iterative approaches to function minimization compute the Taylor series of a given function and try to find a vector $\boldsymbol{\delta}$ such that $f_{NLS}\left(\boldsymbol{\gamma}+\boldsymbol{\delta}\right)<f_{NLS}\left(\boldsymbol{\gamma}\right)$. If additionally:
\begin{equation}
\boldsymbol{\delta}=-\left(\nabla^2{f_{NLS}}\right)^{-1}\nabla{f_{NLS}}
\label{Eq:m6_eq_21}
\end{equation}
then this approach is called Newton's method in optimization.

\hfill
\subsection{Diffusion Biomarkers}

Regardless of the chosen estimation method, the estimated vector $\boldsymbol{\hat{\gamma}}$ is reshaped to be a \vbox{symmetric 3x3 matrix $\boldsymbol{D}$}. 

There exists a couple of strategies for visualizing this high-dimmensional
array containing estimated tensor in each voxel of each slice. One of these approaches is to compute the eigenvalue decomposition of $\boldsymbol{D}$ and
compute a series of images called Diffusion Biomarkers which can be used to extract additional information about tissue microstructure. 

Let us define four biomarker images: 
\begin{enumerate}
	\item MD (Mean Diffusivity) is the mean of tensor eigenvalues. MD is an
	inverse measure of membrane density and is very similar for both gray
	(GM) and white matter (WM) and is higher for corticospinal fluid (CSF).
	MD is sensitive to cellularity, edema and necrosis \cite{m6_basser2002}. 
	\begin{equation}
	MD = \dfrac{\lambda_{1}+\lambda_{2}+\lambda_{3}}{3}
	\label{Eq:m6_eq_23}
	\end{equation}
	
	\item RA (Relative Anisotropy) exhibits high degree of contrast between
	anisotropic (WM) and isotropic tissues. 
	\begin{equation}
	RA = \sqrt{\dfrac{\left(\lambda_{1}-MD\right)^2+\left(\lambda_{2}-MD\right)^2+\left(\lambda_{3}-MD\right)^2}{3\,MD}}
	\label{Eq:m6_eq_24}
	\end{equation}
	
	\item FA (Fractional Anisotropy) is a summary measure of microstructural
	integrity. It is highly sensitive to microstructural changes without
	considering the type of change \cite{m6_soares2013}. 
	\begin{equation}
	FA = \sqrt{\dfrac{3}{2}}\sqrt{\dfrac{\left(\lambda_{1}-MD\right)^2+\left(\lambda_{2}-MD\right)^2+\left(\lambda_{3}-MD\right)^2}{\lambda_{1}^2+\lambda_{2}^2+\lambda_{3}^2}}
	\label{Eq:m6_eq_25}
	\end{equation}
	
	\item VR (Volume Ratio) 
	\begin{equation}
	VR = \frac{\lambda_{1}\lambda_{2}\lambda_{3}}{MD\,^3}
	\label{Eq:m6_eq_26}
	\end{equation}
\end{enumerate}

where $\lambda_{1}$, $\lambda_{2}$, $\lambda_{3}$ are the tensor eigenvalues. One can notice that for each biomarker image to be real-valued, eigenvalues must be non-negative. There are a couple of strategies aimed at solving this problem \cite{m6_koay2006b}:
\begin{itemize}
	\item Substituting negative eigenvalues with zero.
	\item Substituting eigenvalues with their absolute value.
	\item Computing Cholesky-parametrization of input vectors to WLS and NLS estimation methods.
\end{itemize}

One can also compute a 3D map using the eigenvectors corresponding to tensor eigenvalue of highest magnitude. The resulting map is a measure of the direction of principal diffusion, which along with it's magnitude (eigenvalue), is the input signal to Tractography.

Additionally, the aforementioned 3D map can be displayed as a RGB image, because eigenvectors obtained from eigenvalue decomposition of a symmetric real matrix are orthogonal. The intensity of resulting color image is then weighted using one of computed biomarker images, usually Fractional Anisotropy.
