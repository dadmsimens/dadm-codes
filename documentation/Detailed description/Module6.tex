\section{Module 6. Diffusion tensor imaging}

The aim of this module is to estimate brain tissue diffusion tensor
from Diffusion Weighted Images (DWI). DWI are obtained using a different
pulse sequence than anatomical MRI images and as such differ in their
information content. Concretely, Diffusion Tensor Imaging (DTI) gives
an insight into tissue microstructure, probing it using gradient-pulse
excited water molecules. This enables indirect measurements of structural
orientation and the degree of anisotropy as water molecules diffuse
differs between tissues.

In DTI-MRI the measured signal is defined as: 
\begin{equation}
S\left(b,\boldsymbol{g}\right)=S_{0}exp\left(-b\boldsymbol{g^{T}Dg}\right)\label{Eq:dti_eq_1}
\end{equation}

where $S\left(b,\boldsymbol{g}\right)$ is the measured signal, $S_{0}$
is the reference signal without diffusion gradient attenuation, $b$
is the diffusion weight scalar, $\boldsymbol{g}$ is the diffusion
encoding gradient vector of unit length and $\boldsymbol{D}$ is the
diffusion tensor.

The diffusion tensor $\boldsymbol{D}$ is symmetric and describes
molecular mobility along each direction: 
\begin{equation}
\boldsymbol{D}=\begin{bmatrix}D_{xx} & D_{xy} & D_{xz}\\
D_{yx} & D_{yy} & D_{yz}\\
D_{zx} & D_{zy} & D_{zz}
\end{bmatrix}\label{Eq:dti_eq_2}
\end{equation}

Furthermore, the Eq. (\ref{Eq:dti_eq_1}) can be rewritten as: 
\begin{equation}
ln\left(\dfrac{S\left(b,\boldsymbol{g}\right)}{S_{0}}\right)=-b\boldsymbol{g^{T}Dg}\label{Eq:dti_eq_3}
\end{equation}

Estimating $\boldsymbol{D}$ from the above equation can be done using
Weighted Least Squares (WLS) or Nonlinear Least Squares (NLS) algorithms.
In order to correctly estimate the diffusion tensor it is necessary
to acquire at least seven DWI - one for each direction of diffusion
and one in the absence of diffusion gradient.

There exists a couple of strategies of visualizing this high-dimmensional
output array containing the estimated tensor in each voxel of the
analyzed slice. One of the approaches is to $\boldsymbol{D}$ and
project the main eigenvector direction into color space, where customarily
$x$ = red, $y$ = green and $b$ = blue. This 2D visualization also
allows to perform a sanity check by viewing the main direction of
diffusion of white matter, corresponding to fiber longitudal axis,
in corpus callosum.

Diagnoalized estimated diffusion tensor can be used to extract additional
information about tissue microstructure. This module calculates the
following biomarker images: 
\begin{enumerate}
\item MD (Mean Diffusivity) is the mean of tensor eigenvalues. MD is an
inverse measure of membrane density and is very similar for both gray
(GM) and white matter (WM) and is higher for corticospinal fluid (CSF).
MD is sensitive to cellularity, edema and necrosis. 
\begin{equation}
MD=\dfrac{\lambda_{1}+\lambda_{2}+\lambda_{3}}{3}\label{Eq:dti_eq_4}
\end{equation}
\item RA (Relative Anisotropy) exhibits high degree of contrast between
anisotropic (WM) and isotropic tissues. 
\begin{equation}
RA=\sqrt{\dfrac{\left(\lambda_{1}-MD\right)^{2}+\left(\lambda_{2}-MD\right)^{2}+\left(\lambda_{3}-MD\right)^{2}}{3\,MD}}\label{Eq:dti_eq_5}
\end{equation}
\item FA (Fractional Anisotropy) is a summary measure of microstructural
integrity. It is highly sensitive to microstructural changes without
considering the type of change. 
\begin{equation}
FA=\sqrt{\dfrac{3}{2}}\sqrt{\dfrac{\left(\lambda_{1}-MD\right)^{2}+\left(\lambda_{2}-MD\right)^{2}+\left(\lambda_{3}-MD\right)^{2}}{\lambda_{1}^{2}+\lambda_{2}^{2}+\lambda_{3}^{2}}}\label{Eq:dti_eq_6}
\end{equation}
\item VR (Volume Ratio) 
\begin{equation}
VR=\frac{\lambda_{1}\lambda_{2}\lambda_{3}}{MD\,^{3}}\label{Eq:dti_eq_7}
\end{equation}
\end{enumerate}
\hfill{}\\
\textbf{DTI pre-processing pipeline}\footnote{Scientific articles enumerated by Mr Pięciak mention these pre-processing
steps as necessary for proper diffusion tensor estimation.}

In order to improve diffusion tensor estimation it is imperative to
remove artifacts. In addition to standard MRI pre-processing, one
needs to correct for artifacts arising from using of diffusion-gradient
pulse sequences and longer acquisition time. While hardware manufacturers
try to proactively diminish some of these effects, software processing
is still mandatory. 
\begin{enumerate}
\item Eddy currents and subject motion removal.\\
This step attempts to realign (register) all images obtained during
diffusion-gradient sequences to one $T_{2}$-weighted reference image
by finding an affine transform for all diffusion-weighted images. 
\item Magnetic susceptibility (local $B_{0}$ inhomogeneity) correction.\\
This step attempts to map the real $B_{0}$ field map using pairs
of gradient-weighted images obtained from gradient sequences along
the same direction but opposite polarity. The estimated field map
can be used to reconstruct images without distortions. 
\item Skull stripping (Module 8). 
\end{enumerate}
\hfill{}\\
\textbf{Module I/O} 
\begin{itemize}
\item \textbf{Input} - DiffusionData object instance containing 3D \textbf{\emph{slice}}
(image pixel intensity varying in time with gradient sequence or lack
thereof), \textbf{\emph{bvalue}} vector corresponding to used diffusion-gradient
sequence intensity for each image, \textbf{\emph{bvector}} array containing
diffusion-gradient sequence unit vector corresponding to \textbf{\emph{bvalue}}
and \textbf{\emph{other}} with general information obtained from data
format (most importantly: voxel physical dimensions and time between
subsequent echo pulses). 
\item \textbf{Output} - same as input DiffusionData object instance with
a new field \textbf{\emph{tensor}} containing estimated diffusion
tensor data. Biomarker images (MD, RA, FA, VR) can be calculated from
\textbf{\emph{tensor}} using implemented methods and then displayed
on a 2x2 grid. 
\end{itemize}
\hfill{}\\
\textbf{List of References}\\
\cite{6_dti_1}, \cite{6_dti_2}, \cite{6_dti_3}, \cite{6_dti_4},
\cite{6_dti_5}, \cite{6_dti_6}, \cite{6_dti_7}, \cite{6_dti_8},
\cite{6_dti_9}, \cite{6_dti_10}